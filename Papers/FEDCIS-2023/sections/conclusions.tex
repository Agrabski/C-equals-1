\section{Conclusions}

CTFEF is a new approach to compiler construction which offers high degree of compiler extensibility, at the cost of development time and performance, compared with conventional compilers.
It places the interpreter as the main component of the compiler and focuses on executing user code at compile time.
The user code has access to the same information as the compiler at compile time and may perform analysis or transformation of the compiled program.
The thesis that introduced this approach \cite{grabski2020} demonstrated numerous practical application: generating bindings for other languages, static analysis and extending semantics of the language.
These goals are significantly easier to accomplish, thanks to the access to semantic model of the program, constructed by the compiler.
Authors of language tools do not need to analyze the user program.

On the other hand, using CTFEF approach has some drawbacks. Implementing a compiler is more difficult.
Since a significant part of the compiler is interpreted, the initial implementation requires working with the target language.
The lack of available tools, such as integrated development environments, debuggers and libraries makes this part of the process significantly more difficult.
On the other hand also means the work on the compiler in the target language may start at the very beginning of compiler bootstrapping process\cite{puntambekar:compiler_design, novillo2007gcc}.

The compiler created for C-=-1 has unacceptable performance, as noted by the original C-=-1 paper \cite{grabski2022compilation}.
Compiling the C-=-1 standard library, containing around 200 lines of code, takes approximately 10 minutes.
Majority of that time is spent on interpreting the compiler interface.
This is a significant barrier to adopting CTFEF approach.
Since the compiler implemented for C-=-1 was made as a research tool with minimal effort, further work is needed to explore the performance issues of CTFEF approach.
