\section{Introduction}

Compile-time function execution is an aspect of a programming language, that allows the programmer to execute code at compile-time.
The goals and capabilities of CTFE depend on the language and is discussed in chapter \ref{related-work}, however it plays a secondary role in the compilation process.

CTFE first builds the compiler around the interpreter component.
The role of the compiler is to construct the semantic model of the program being compiled and pass it as data to the interpreted Compiler-Intreface module.
Compiler-Interface then tranforms the model into the backend's assembly language to be compiled to executable.
Structure of the compiler and its influence on the language design is further discussed in chapter \ref{language-compiler-design}.

Using CTFE First approach carries with it certain unique implementation considerations.
Data structures used within the compiler and the way that the program is modeled is tightly coupled with the compiled language.
These challenges, as well as how they were solved in C-=-1 compiler were described in chapter \ref{implementation}.
