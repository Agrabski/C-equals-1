\section{Introduction}


Compile-time function execution (CTFE) is an aspect of a programming language, that allows the programmer to execute code at compile-time.
It has been gaining popularity with low level programming langauges such as Rust \cite{rust} or C++ \cite{ISO:cpp98}.
The goals and capabilities of CTFE depend on the language and are discussed in chapter \ref{related-work}.

In this paper a new approach to compiler construction that places Compile Time Function Execution capabilities as the first priority is proposed.
This architecture is called CTFEF: Compile Time Function Execution First.
CTFEF builds the compiler around the interpreter component.
The role of the compiler is to construct the semantic model of the program being compiled and pass it as data to the interpreted Compiler-Intreface module.
Compiler-Interface then tranforms the model into the backend's assembly language to be compiled to executable.
This approach was created during implementation of a compiler for C-=-1, a new language that prioritized static metaprogramming.
The design of that language is described in chapter \ref{language-design}.
Chapter \ref{compiler-design} describes the structure of a CTFEF compiler and how its components interact.

Using CTFEF approach carries with it certain unique implementation considerations.
Data structures used within the compiler and the way that the program is modeled is tightly coupled with the compiled language.
These challenges, as well as how they were solved in C-=-1 compiler are described in chapter \ref{implementation}.
