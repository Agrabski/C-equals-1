However, compiler is a very important program not many of us has the chance to implement it. Currently, the chances of implementing a compiler are increasing because many domains languages are proposed, and theirs compilers has to be produced.

In this paper a novel approach to compiler architecture that places the Interpreter as the central component of the compiler is proposed.
Translation of user code into the executable form is done by an Interpreter, written in the target language. The data structures of Interpreter are accessible also to the programmer. Such solution significantly improves flexibility and extensibility of the compiler and enables execution of any code at the compile-time.

Compile Time Function Execution (CTFE) First pattern was designed for low-level, non-garbage-collected, reflection-enabled language C-=-1. This is a new programming language, which allows the programmer to execute any code at the compile time, as well as to analyze and modify the program structure.
The significant flexibility and extensibility of CTFEF caused severe problems in compiler construction. The compiler appeared very complex, its parts, especially Interpreter, were very difficult to debug.
