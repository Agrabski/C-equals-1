\section{C-=-1}

\begin{frame}
    \frametitle{Type System}

    \begin{itemize}
        \item Based on C++\begin{itemize}
            \item Classes have methods and fields with access control
            \item Interfaces are abstract classes with only virtual methods
        \end{itemize}
        \item Generic programming done via templates
        \item Metaprogramming done with attributes\begin{itemize}
            \item Special types similar in concept to C\#
            \item Are notified at compiletime when a program element is used
            \item May report diagnostics or modify code
        \end{itemize}
    \end{itemize}

\end{frame}

\begin{frame}[fragile]
    \frametitle{Metaprogramming}

    \begin{lstlisting}
public att<function> NoDiscard {
    public fn attach(f: functionDescriptor) {}
    public fn onCall(call: functionCallExpression*) {
        if(call._parentStatment != null<IInstruction>())
            raiseError(&(call._pointerToSource), "Return value of a no-discard function is not used", 123);
    }
}
[noDiscard()]
fn noDiscardFunction() -> usize;
fn main() -> usize {
  noDiscardFunction(); // error 123: Return value of
                       // a no-discard function is not used
  let x = noDiscardFunction();     // ok
  let y = x + noDiscardFunction(); // ok
  return noDiscardFunction();      // ok
}
    \end{lstlisting}

\end{frame}


\begin{frame}
    \frametitle{Semantic model}

    \begin{itemize}
        \item Interpreter operates on several types of value\begin{itemize}
            \item Pimitives (string and integer)
            \item Object
            \item Reference to value
            \item Marshalled references (reference to language entity, reference to native object eg: LlvmContext)
        \end{itemize}
        \item Only limited number of language entities have types associated with them in compiler code\begin{itemize}
            \item Classes and their fields
            \item Functions
            \item Generics
        \end{itemize}
        \item The rest are expressed using interpreter data types\begin{itemize}
            \item Instructions
            \item Expressions
            \item Literals
        \end{itemize}
    \end{itemize}
\end{frame}

\begin{frame}[fragile]
    \frametitle{Using compiler data structure}

    \begin{lstlisting}
Type* createType(
    Namespace* target,
    NameResolver& resolver,
    NameResolutionContext& context,
    TypeDeclarationContext* ctx,
    path const& path)
{
 auto name = ::name(ctx);
 auto type = target->get<dataStructures::Type>(name);
 if (type == nullptr)
  type = target->append<dataStructures::Type>(name);
 if (ctx->classContentSequence() != nullptr)
  for (auto member : ctx->classContentSequence()->functionDeclaration())
   createFunction(not_null(type), member, context, path);
 return type;
}
    \end{lstlisting}

\end{frame}

\begin{frame}[fragile]
    \frametitle{Using interpreter data structures}

    \begin{lstlisting}
runtime_value buildVariableReferenceExpressionDescriptor(Variable* var) {
 auto [result, object] = heapAllocateObject(getVariableReferenceExpressionDescriptor());
 object.setValue("_variable", createVariableDescriptor(var));
 object.setValue("_type", createTypeDescriptor(var->type()));
 return std::move(result);
}
    \end{lstlisting}

\end{frame}

\begin{frame}
    \frametitle{Compiler structure}

    

\end{frame}
