\documentclass[10pt,xcolor=svgnames]{beamer} %Beamer
\usepackage{palatino} %font type
\usepackage{caption}
\usepackage{hyperref}
\usepackage[T1]{fontenc}
\usepackage{listings}
\usepackage{csquotes}

\usefonttheme{metropolis} %Type of slides
\usefonttheme[onlymath]{serif} %font type Mathematical expressions
\usetheme[progressbar=frametitle,titleformat frame=smallcaps,numbering=counter]{metropolis} %This adds a bar at the beginning of each section.
\useoutertheme[subsection=false]{miniframes} %Circles in the top of each frame, showing the slide of each section you are at

\usepackage{appendixnumberbeamer} %enumerate each slide without counting the appendix
\setbeamercolor{progress bar}{fg=Maroon!70!Coral} %These are the colours of the progress bar. Notice that the names used are the svgnames
\setbeamercolor{title separator}{fg=DarkSalmon} %This is the line colour in the title slide
\setbeamercolor{structure}{fg=black} %Colour of the text of structure, numbers, items, blah. Not the big text.
\setbeamercolor{normal text}{fg=black!87} %Colour of normal text
\setbeamercolor{alerted text}{fg=DarkRed!60!Gainsboro} %Color of the alert box
\setbeamercolor{example text}{fg=Maroon!70!Coral} %Colour of the Example block text


\setbeamercolor{palette primary}{bg=NavyBlue!50!DarkOliveGreen, fg=white} %These are the colours of the background. Being this the main combination and so one.
\setbeamercolor{palette secondary}{bg=NavyBlue!50!DarkOliveGreen, fg=white}
\setbeamercolor{palette tertiary}{bg=NavyBlue!40!Black, fg= white}
\setbeamercolor{section in toc}{fg=NavyBlue!40!Black} %Color of the text in the table of contents (toc)

%These next packages are the useful for Physics in general, you can add the extras here.
\usepackage{amsmath,amssymb}
\usepackage{slashed}
\usepackage{dirtree}
\usepackage{relsize}
\usepackage{multirow}
\usepackage{caption}
\usepackage{subcaption}
\usepackage{multicol}
\usepackage{booktabs}
\usepackage[scale=2]{ccicons}
\usepackage{pgfplots}
\usepgfplotslibrary{dateplot}
\usepackage{geometry}
\usepackage{xspace}
\usepackage{color}

\lstloadlanguages{C,C++,csh,Java}

\definecolor{red}{rgb}{0.6,0,0}
\definecolor{blue}{rgb}{0,0,0.6}
\definecolor{green}{rgb}{0,0.8,0}
\definecolor{cyan}{rgb}{0.0,0.6,0.6}

\lstset{
language=csh,
basicstyle=\footnotesize\ttfamily,
numbers=left,
numberstyle=\tiny,
numbersep=5pt,
tabsize=2,
extendedchars=true,
breaklines=true,
frame=b,
stringstyle=\color{blue}\ttfamily,
showspaces=false,
showtabs=false,
xleftmargin=17pt,
framexleftmargin=17pt,
framexrightmargin=5pt,
framexbottommargin=4pt,
commentstyle=\color{green},
morecomment=[l]{//}, %use comment-line-style!
morecomment=[s]{/*}{*/}, %for multiline comments
showstringspaces=false,
morekeywords={ abstract, event, new, struct,
as, explicit, null, switch,
base, extern, object, this,
bool, false, operator, throw,
break, finally, out, true,
byte, fixed, override, try,
case, float, params, typeof,
catch, for, private, uint,
char, foreach, protected, ulong,
checked, goto, public, unchecked,
class, if, readonly, unsafe,
const, implicit, ref, ushort,
continue, in, return, using,
decimal, int, sbyte, virtual,
default, interface, sealed, volatile,
delegate, internal, short, void,
do, is, sizeof, while,
double, lock, stackalloc,
else, long, static,
enum, namespace, string},
keywordstyle=\color{cyan},
identifierstyle=\color{red},
backgroundcolor=\color{white},
}
\usepackage{biblatex}
\usepackage{caption}
\DeclareCaptionFont{white}{\color{white}}
\DeclareCaptionFormat{listing}{\colorbox{blue}{\parbox{\textwidth}{\hspace{15pt}#1#2#3}}}
\captionsetup[lstlisting]{format=listing,labelfont=white,textfont=white, singlelinecheck=false, margin=0pt, font={bf,footnotesize}}
\newcommand{\themename}{\textbf{\textsc{bluetemp}\xspace}}%metropolis}}\xspace}

\title{CTFEF }
\subtitle{Building a compiler for metaprogramming languages}
\author[name]{
	mgr. inż. Adam Grabski\\
	dr. hab. inż Ilona Bluemke
}

\institute[uni]{ Department of Electronics and Information Technology \\
Warsaw Univeristy of Technology
}
\date{\today} %Here you can change the date
\begin{document}
{

\setbeamercolor{background canvas}{bg=NavyBlue!50!DarkOliveGreen, fg=white}
\setbeamercolor{normal text}{fg=white}
\maketitle
}%This is the colour of the first slide. bg= background and fg=foreground


\begin{frame}{Agenda}
	\setbeamertemplate{section in toc}[sections numbered] %This is numbering the sections
	\tableofcontents[subsectionstyle=hide/hide/hide] %You can comment this line if you want to show the subsections in the table of contents
\end{frame}

\section{Motivation}


\begin{frame}
	\frametitle{Metaprogramming}

	\begin{itemize}
		\item Metaprogramming - writing programs that analyse or modify themselves\begin{itemize}
			      \item Dynamic - metaprogramming done at run time
			      \item Static - metaprogramming done at compile time
		      \end{itemize}
		\item Makes code more expressive and resilient to change
		\item Boosts productivity
		\item Can improve performance
		\item Mostly present in garbage-collected languages at runtime\begin{itemize}
			      \item C\#
			      \item Java
			      \item Scripting languages such as Javscript or python
		      \end{itemize}
	\end{itemize}

\end{frame}


\begin{frame}
	\frametitle{Static metaprogramming}

	\begin{itemize}
		\item Until recently a niche tool \begin{itemize}
			      \item C style preprocessor macros
			      \item D templates, compile time function execution and string mixins
		      \end{itemize}
		\item Significant developments since 2011\begin{itemize}
			      \item C++ SFINAE template metaprogramming (standardized 2011)
			      \item Rust procedural macros (introduced with the language in 2015)
			      \item C\# source generators (introduced in 2020)
		      \end{itemize}

		\item Significant current interest and more development in the future\begin{itemize}
			      \item Migration of many typical dynamic metaprogramming activities to compile time in C\#
			      \item Introduction of compile time interceptors in C\#12
			      \item Continuous development of generic programming in C++
		      \end{itemize}
	\end{itemize}

\end{frame}


\end{document}
