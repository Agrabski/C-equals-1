%%%%%%%%%%%%%%%%%%%%%%%%%%%%%%%%%%%%%%%%%%%%%%%%%%%%%%%
%% Bachelor's & Master's Thesis Template             %%
%% Copyleft by Artur M. Brodzki & Piotr Woźniak      %%
%% Faculty of Electronics and Information Technology %%
%% Warsaw University of Technology, 2019-2020        %%
%%%%%%%%%%%%%%%%%%%%%%%%%%%%%%%%%%%%%%%%%%%%%%%%%%%%%%%

\documentclass[
    left=2.5cm,         % Sadly, generic margin parameter
    right=2.5cm,        % doesnt't work, as it is
    top=2.5cm,          % superseded by more specific
    bottom=3cm,         % left...bottom parameters.
    bindingoffset=6mm,  % Optional binding offset.
    nohyphenation=false % You may turn off hyphenation, if don't like.
]{eiti/eiti-thesis}

\langpol % Dla języka angielskiego mamy \langeng
\graphicspath{{img/}}             % Katalog z obrazkami.
\addbibresource{bibliografia.bib} % Plik .bib z bibliografią
\usepackage{minted}

\begin{document}

%--------------------------------------
% Strona tytułowa
%--------------------------------------
\MasterThesis % Dla pracy inżynierskiej mamy \EngineerThesis
\instytut{Informatyki}
\kierunek{Informatyka}
\specjalnosc{Sztuczna inteligencja}
\title{
Kompilacja przez interpretację:\\
statyczne metaprogramowanie w języku C-=-1
}
\engtitle{ % Tytuł po angielsku do angielskiego streszczenia
    Unnecessarily long and complicated thesis' title \\
    difficult to read, understand and pronounce
}
\author{Adam Grabski}
\album{283431}
\promotor{dr. hab. inż. Ilona Bluemke}
\date{\the\year}
\maketitle

%--------------------------------------
% Streszczenie po polsku
%--------------------------------------
\cleardoublepage % Zaczynamy od nieparzystej strony
\streszczenie 
Statyczne metaprogramowanie jest nowym i dynamicznie rozwijającym się trendem w językach niskiego poziomu.
Wszystkie tego typu mechanizmy, niosą ze sobą jednak poważne ograniczenia.
C-=-1 proponuje nowe podejście do konstrukcji kompilatora, umożliwiające wprowadzenie do języka bardzo potężnych mechanizmów metaprogramistycznych. 
\slowakluczowe XXX, XXX, XXX

%--------------------------------------
% Streszczenie po angielsku
%--------------------------------------
\newpage
\abstract \kant[1-3]
\keywords Metaprogramowanie, XXX, XXX

%--------------------------------------
% Oświadczenie o autorstwie
%--------------------------------------
\cleardoublepage  % Zaczynamy od nieparzystej strony
\pagestyle{plain}
\makeauthorship

%--------------------------------------
% Spis treści
%--------------------------------------
\cleardoublepage % Zaczynamy od nieparzystej strony
\tableofcontents

%--------------------------------------
% Rozdziały
%--------------------------------------
\cleardoublepage % Zaczynamy od nieparzystej strony
\pagestyle{headings}

\section{Wstęp}
Celem tej pracy jest zaproponowanie nowych mechanizmów statycznego 
metaprogramowania.
Zostały one zaprojektowane z myślą o statycznej analizie kodu oraz prostocie projektowania nowych analizatorów.
Statyczna analiza kodu to technika wyciągania wniosków na temat programu, na podstawie wyłącznie jego kodu źródłowego \cite{survey_of_metaprograming}.

Jednym z ważniejszych celów analizatora jest wykrywanie typowych błędów programistycznych, zanim program zostanie skompilowany.
Ostatnia dekada rozwoju języków programowania i narzędzi z nimi związanych wskazuje, że zapotrzebowanie na statyczną analizę kodu rośnie.
W tym samym czasie doszło do gwałtownego wzrostu zainteresowania metaprogramowaniem. Ten termin obejmuje zarówno refleksję, czyli pozyskiwanie informacji o strukturze programu, jak i modyfikację kodu. W zależności od tego, czy dany mechanizm jest aktywny w czasie kompilacji, czy uruchomienia, nazywamy go odpowiednio statycznym lub dynamicznym.
Meta-program można zdefiniować jako aplikację, która spełnia jedno z następujących wymagań \cite{nielson2004principles}:
\begin{enumerate}
\item Program operuje na innym programie.
\item Program wytwarza inny program jako swój wynik.
\item Program uzyskuje dostęp lub modyfikuje własną strukturę.
\end{enumerate}

Zaproponowane w tej pracy mechanizmy, opisane dokładniej w rozdziale \ref{Attributes_definition}, umożliwiają stworzenie aplikacji jawnie złożonej z dwóch części.
Konwencjonalnego programu, który zostanie skompilowany do formy wykonywalnej oraz meta-programu, który zostanie wykonany w czasie kompilacji.
Jego celem jest wykonanie analizy oraz modyfikacji konwencjonalnej części kodu.
Aby to osiągnąć, zaprojektowany został nowy, uproszczony język programowania zawierający proponowane mechanizmy. To podejście zostało przyjęte, aby ułatwić pisanie kompilatora, poświęcając wygodę programisty.

Kontynuując tradycję zaczętą przez Bjarne Stroustrup-a nazwa, którą nadano temu nowemu językowi, jest żartem programistycznym. C++ swoją nazwę wziął od jednego ze sposobów na inkrementację zmiennej o nazwie C \cite{stroustrup_com}. 
W ten sam sposób, w który C++ jest rozwiniętą wersją C, C-=-1 (wymawiane 'cm1') jest nietypową alternatywą dla C++. C-=-1 zostało nazwane na podstawie C++ ponieważ dzieli z nim podstawy filozoficzne oraz niektóre decyzje projektowe. Te podobieństwa zostały opisane w rozdziale 3.
W ramach prac badawczych zaimplementowano podstawową bibliotekę standardową oraz napisano szereg prostych programów z jej wykorzystaniem. Następnie porównano je z programami, osiągającymi ten sam cel, napisanymi w innych językach programowania, pod kątem złożoności, czytelności i powstałych plików wykonywalnych.
         % Wygodnie jest trzymać każdy rozdział w osobnym pliku.
\section{Istniejące rozwiązania}
Wśród przemysłowo zastosowanych języków programowania istnieje szeroka gama mechanizmów meta-programistycznych. Stosuje się w nich zarówno statyczną jak i dynamiczną refleksję. W popularnie używanych językach, mechanizmy dynamiczne są zwykle dużo bardziej rozbudowane, przyjazne użytkownikowi i potężne.

\subsection{C\#}
Język C\# od początku swojego istnienia wspierał refleksję.
Środowisko uruchomieniowe .Net zostało zaprojektowane z myślą o dynamicznym ładowaniu plików bibliotecznych.
W związku z tym większość mechanizmów metaprogramistycznych w językach opartych na nim jest używana w trakcie uruchomienia.
W ostatnich wersjach Microsoft dodał jednak pewne możliwości do generowania i analizowania kodu w trakcie kompilacji.
\subsubsection{Atrybuty}
W języku C\# atrybuty służą głównie do przechowywania metadanych.
Na rynku, istnieją rozwiązania, które wykorzystują takie adnotacje do modyfikacji programu, jednak takie zastosowanie nie jest proste.
Ponadto, jakiekolwiek zmiany w kodzie aplikacji, będą zachodzić w trakcie uruchomienia, nie kompilacji.
Oznacza to narzuty, które w wielu kontekstach są nieakceptowalne.
\subsubsection{Refleksja}

Informacja o organizacji programu jest integralną częścią skompilowanego pliku C\#.
W czasie uruchomienia aplikacja ma pełny wgląd we własną strukturę.
Ponieważ w C\# występuje dynamiczna refleksja, program ma również dostęp do metadanych w bibliotekach ładowanych w czasie pracy.

Ten mechanizm niesie ze sobą też pewne koszty.
Refleksja w czasie uruchomienia niesie za sobą poważne koszty, które w niektórych kontekstach są nieakceptowalne.
Ponadto, obecność informacji o typach i funkcjach w skompilowanym pliku znacząco ułatwia dekompilację.

Można też argumentować, że dostęp do metadanych programu w trakcie uruchomienia utrudnia optymalizacje.
Program może przeglądać własną strukturę i dynamiczne wywoływać kod na podstawie pozyskanych informacji.
W związku z tym niektóre typy optymalizacji takie jak inlining nie są możliwe.

\subsubsection{Analizatory kodu}

Jednym z wyników publicznego udostępnienia kodu kompilatora Roslyn, było udostępnienie programistom API kompilatora służącego do walidacji kodu.
Programista może więc stworzyć własny analizator w ramach swojego projektu.
Zaletą tego podejścia  jest spójność używanego języka.
Projekt pisany w C\# może być analizowany w C\#, bez żadnych dodatkowych restrykcji.
Unika to problemu tworzenia dodatkowego dialektu języka, jaki występuje w C++.

\subsubsection{Generatory kodu}

C\#9 wprowadził możliwość pisania generatorów kodu. Jest to bardzo podobne rozwiązanie do makr w języku Rust. Generatory kodu to specjalne klasy, których kod jest wykonywany w trakcie kompilacji. 
Mechanizm C\# jest jednak bardziej zaawansowany.
Umożliwia generowanie kodu, używając wszystkich elementów języka, zamiast zapewniać prosty szablon.
Te generatory, tak jak w Rust, nie mają jednak dostępu do informacji zbieranych przez kompilator, poza drzewem składniowym oraz nie mogą modyfikować istniejącego kodu.

Generatory kodu nadal są rozwijane i w najbliższym czasie nie wejdą do stabilnego C\#. Stanowią one jednak ciekawą zmianę w kierunku języka, w stronę większego wsparcia dla statycznego metaprogramowania.

\subsection {Rust}
Ponieważ Rust jest stosunkowo młody, jego projektanci mogli czerpać z doświadczeń innych języków. Wyraźnie widać to w jego składni, zintegrowanym systemie zarządzania zależnościami oraz bardzo przyjaznym kompilatorze.\par
Z tego powodu, nie zawiera on też tradycyjnego preprocesora, takiego jak w C/C++ czy C\#. Makra w Rust są tak naprawdę funkcjami operującymi na drzewie składniowym programu. Jest to zdecydowany postęp względem prostego zastępowania tekstu jak w C/C++, lecz ma też swoje wady.\par
Makra w języku Rust nadal operują na, co prawda strukturyzowanym, ale nadal, tekście. Oznacza to, że nie mają one dostępu do systemu typów ani żadnych innych informacji zgromadzonych przez kompilator.\par
\subsection{C++20}
C++ zawiera szeroką gamę mechanizmów metaprogramistycznych oraz technik do statycznej walidacji kodu. Były one stopniowo wprowadzane do języka przez cały czas jego istnienia, bez spójnego planu.\par
\subsubsection{Szablony}
Szablony w C++ można rozumieć jako oddzielny, funkcyjny, zdolny do symulacji maszyny Turinga, język programowania \cite{template_turing_complete}.
Metaprogramowanie szablonowe powstało nie jako świadomy wysiłek twórców języka, lecz jako naturalna, nieprzewidziana konsekwencja jego zasad.

\subsubsection{Koncepty}
Koncepty zostały wprowadzone w C++20 jako sposób na lepsze dokumentowanie szablonów, oraz czytelniejsze komunikaty o błędzie w wypadku niepoprawnego użycia.
Używa się ich do ograniczania, które typy mogą zostać wykorzystane w danym szablonie.

Koncepty są definiowane jako zbiór wymagań, które typ musi spełnić.
Mogą to być proste wyrażenia logiczne jak i weryfikowanie czy dana metoda istnieje oraz jaki typ zwraca.
W przeciwieństwie do pozostałych omawianych mechanizmów, koncepty bronią programistę przed niezrozumiałymi błędami kompilacji zamiast przed błędami czasu uruchomienia.

\subsubsection{Atrybuty}
Atrybuty zostały oficjalnie wprowadzone do języka w C++11. Umożliwiają one przekazanie kompilatorowi dodatkowych informacji na temat kodu źródłowego. Te dane są potem wykorzystywane w trakcie kompilacji do wydawania lub ignorowania ostrzeżeń. Dwa przykłady warte omówienia to fallthrough oraz no\_discard, ponieważ prezentują obydwa przypadki zastosowania atrybutu.\par
Atrybut fallthrough powstał, aby rozwiązać typowy problem języków zawierających konstrukcję switch. W większości zastosowań, celem programisty jest stworzenie zestawu niezależnych od siebie przypadków, spośród których zostanie wykonany jeden na podstawie wartości jakiejś zmiennej.\par
Niestety domyślnym zachowaniem konstrukcji switch, po wykonaniu bloku jest przejście do następnego (tak zwany fallthrough). Kompilatory zaczęły więc wydawać ostrzeżenie, jeśli blok case nie zostanie zakończony instrukcją break. To tworzy jednak inny problem: czasami programista chce osiągnąć dokładnie takie zachowanie. Niektóre kompilatory zaczęły przez to zwracać uwagę na komentarze, szukając tam wyrażenia fallthrough (w GCC flaga -Wimplicit-fallthrough=3) \cite{gcc_warnings}.\par
W C++11 sformalizowano to zachowanie, tworząc atrybut fallthrough \cite{ISO:2012:III}. Zaaplikowanie go do przypadku konstrukcji switch, powstrzymuje kompilator przed wydaniem ostrzeżenia.\par
Atrybut no\_discard zamiast uciszać ostrzeżenie, generuje je. Jego zadaniem jest wykrywanie sytuacji w których programista ignoruje wartość zwracaną przez funkcję. Można go zaaplikować zarówno do typów (wtedy każda funkcja zwracająca ten typ staje się no\_discard) lub do funkcji.\par
Typowym przykładem zastosowanie tego atrybutu są kody błędów. Ze względu na duży koszt mechanizmu wyjątków w C++, w niektórych projektach się ich nie stosuje. Zamiast tego, funkcje których wykonanie może się nie powieść, zwracają strukturę informującą o błędzie. Ponieważ programista wołający taką funkcję, nie ma obowiązku sprawdzić wyniku tego wywołania, łatwo było o błąd. Stąd konieczność atrybutu no\_discard.\par
Podstawową wadą atrybutów w C++ jest fakt, że są one zachowaniem bezpośrednio wpisanym w kompilator. Nie istnieje możliwość napisania własnego atrybutu a ich lista jest bardzo krótka.\par
\subsubsection{Funkcje constexpr}
Język C++11 wprowadził koncepcję constexpr. Zaaplikowanie tego modyfikatora do zmiennej lub funkcji, informuje kompilator że mogą one być ewaluowane w trakcie kompilacji. Funkcje constexpr mają też nałożony szereg ograniczeń co do swojej struktury. Z każdą kolejną wersją języka są one rozluźniane, jednak nie każda funkcja może być ewaluowana w trakcie kompilacji.

Ponadto, funkcje constexpr nie mają większego dostępu do struktury programu niż normalny program. Nie dają one możliwości modyfikowania istniejącego ani generowania nowego kodu. Stanowią one więc jedynie sposób na optymalizację programu.
W związku z tym, mechanizm \lstinline{constexpr} został skrytykowany za jawne definiowanie które funkcje są wykonywalne w czasie kompilacji.
Autor artykułu argumentował, że wykonywalność funkcji w czasie kompilacji powinna być przejrzysta dla programisty i obsługiwana przez kompilator automatycznie \cite{Klimiankou:contexpr_great_good_wrong_idea}.
    % Umożliwia to również łatwą migrację do nowej wersji szablonu:
\section{Projekt języka}
\label{Language_desig}
%todo: talk about limitations on interpreting and compilation, which functions can be executed, when
Język C-=-1 w przeciwieństwie do większości współczesnych języków programowania jest oparty na udostępnieniu programiście struktur danych tworzonych przez kompilator. Deskryptory typów, funkcji, przestrzeni nazw oraz enumeratorów, razem z reprezentacją pośrednią kodu są udokumentowaną częścią języka (załącznik nr 1).
Aby umożliwić użytkownikowi wykorzystanie tych struktur danych, C-=-1 musi zapewnić sposób na napisanie kodu wykonywanego w czasie kompilacji, który może nimi manipulować. W tym celu postanowiono rozszerzyć koncepcję atrybutu, aby umożliwić inspekcję i modyfikację reprezentacji pośredniej.
C-=-1 został zaprojektowany na bazie założeń C++. Obydwa te języki są statycznie typowane, kompilowane oraz nie wymagają specjalnego środowiska uruchomieniowego.
Przy projektowaniu C-=-1 skorzystano jednak z doświadczeń wyciągniętych z C++ w aspektach takich jak dziedziczenie wielobazowe czy zarządzanie pamięcią. Obydwa te aspekty zostały opisane odpowiednio w rozdziałach \ref{elementy_jezyka} oraz \ref{struktura_paczki}

\subsection{Składnia}

\subsection{Elementy języka}
\label{elementy_jezyka}

\subsubsection{Klasy}
\label{classes_definition}
Klasy w języku C-=-1 działają analogicznie do języków z rodziny C.
Są to zbiory danych (pola klasy) z którymi powiązane są pewne operacje (metody).
Wszystkie elementy takiego typu mogą mieć ograniczenia dostępu.
Dwoma istotnymi elementami klas w C-=-1 są metody specjalne \lstinline{construct} oraz \lstinline{finalize}.
Można je zrozumieć jako odpowiedniki konstruktora oraz destruktora z C++.
Ponieważ C-=-1 w swojej obecnej formie nie wspiera wyjątków, nie gwarantuje wykonania finalizatorów zmiennych lokalnych w wypadku nieoczekiwanego zakończenia programu.

%todo: reference this
\begin{minipage}{\linewidth}
  
	\begin{lstlisting}[
	  numbers=left,
	  firstnumber=0,
	  caption={Fragment gramatyki C-=-1 deklarujący klasę},
	  aboveskip=0pt,
	  label={lst:class_grammar}
	  ]
typeDeclaration:
(attributeSequence)? AccessSpecifier? classTypeSpecifier identifier 
  genericSpecifier? (
  ':' implementedInterfacesSequence
  )? '{' classContentSequence '}';
classTypeSpecifier: ('class' | 'interface' | 'struct');
  \end{lstlisting}
  \end{minipage}

\subsubsection{Interfejsy}
\subsubsection{Funkcje}
W C-=-1 funkcje oraz metody działają na tej samej zasadzie co w C++.
Istotną różnicą jest możliwość ich wykonania w czasie kompilacji.
Domyślnie, funkcje mogą być wykonywane w dowolnym kontekście.
O ile programista nie sprecyzował żadnych ograniczeń, podprogram może zostać wykonany zarówno w czasie uruchomienia, jak i kompilacji.
Wywoływanie funkcji w trakcie kompilacji ma kilka różnych zastosowań.

Optymalizacja programu, historycznie stanowiła podstawową motywację dla wykonywania kodu w czasie kompilacji.
Jeśli wyrażenie zależy wyłącznie od stałych, kompilator może podjąć decyzję o jego ewaluacji do stałej.
C-=-1, w przeciwieństwie do C++, umożliwia optymalizację niemalże dowolnej funkcji w ten sposób.
Ponieważ programista nie musi jawnie precyzować, że procedura jest wykonywalna w czasie kompilacji, język C-=-1 unika problemów powiązanych z modyfikatorem \lstinline{constexpr} z C++ \cite{Klimiankou:contexpr_great_good_wrong_idea}
Taka możliwość tworzy także pewne wyzwania, dokładniej opisane w rozdziale \ref{compile_time_constant_evaluation}.

\subsubsection{Atrybuty}
% todo: stretch importance
\label{Attributes_definition}

Atrybuty są bardzo zbliżone do klas: składają się z metod, pól oraz konstruktorów i mogą implementować interfejsy.
Podobieństwa te kończą się jednak na najbardziej podstawowych tych bytów.
Klasy i atrybuty pełnią w C-=-1 diametralnie różne role i istnieją w odrębnych kontekstach.
Atrybuty mogą być tworzone wyłącznie jako adnotacje do innych elementów programu, a ich instancje istnieją wyłącznie w czasie kompilacji.

Listing \ref{lst:attribute_basic_example} zawiera przykład prostego atrybutu w C-=-1.
Składniowo ta deklaracja jest niemalże identyczna, do deklaracji klasy opisanej w rozdziale \ref{classes_definition}.
Największą różnicą jest użycie słowa kluczowego \lstinline{att} i deklaracja celu atrybutu, zamiast \lstinline{class}.

\begin{minipage}{\linewidth}
  
	\begin{lstlisting}[
	  numbers=left,
	  firstnumber=0,
	  caption={Fragment gramatyki C-=-1 deklarujący atrybut},
	  aboveskip=0pt,
	  label={lst:attribute_grammar}
	  ]
attributeDeclaration: 
  (AccessSpecifier)? 'att' '<' attributeTarget+ '>' 
  identifier (':' implementedInterfacesSequence)?
  '{' classContentSequence '}';

attributeTarget: ('type' | 'variable' | 'function');
  \end{lstlisting}
  \end{minipage}

Listing \ref{lst:attribute_grammar} zawiera fragment gramatyki C-=-1, w notacji EBNF \cite{ebnf}.
Wyraźnie widać w niej podobieństwo do deklaracji klasy z listingu \ref{lst:class_grammar}.
Ciało atrybutu używa wręcz tej samej reguły co ciało klasy.
Istotnym elementem gramatyki atrybutu, jest natomiast wspomniany wcześniej cel atrybutu (reguła \lstinline{attributeTarget}), deklarujący, do których elementów języka można przyłączyć dany atrybut.

W obecnym stanie języka istnieją trzy cele: funkcje, zmienne i typy, zgodnie z linią piątą listingu \ref{lst:attribute_grammar}.
Atrybut może być powiązany z wieloma celami.


\begin{minipage}{\linewidth}
  
	\begin{lstlisting}[
	  numbers=left,
	  firstnumber=0,
	  caption={Przykład atrybutu w C-=-1},
	  aboveskip=0pt,
	  label={lst:attribute_basic_example}
	  ]
  public att<function> SomeAttribute
  {
	private _number: usize;
	public fn construct(number: usize)
	{
		self._number = number;
	}
	public fn attach(f: functionDescriptor)
	{}
  }
  \end{lstlisting}
  \end{minipage}
\subsection{Mechanizm atrybutów}
\label{Attributes_mechanism_cm1}
\subsection{Reprezentacja pośrednia}\label{reprezentacja_posrednia}
Programista C-=-1 może analizować oraz modyfikować reprezentację pośrednią swojego programu w czasie kompilacji. C-=-1 Intermidiate Representation, w skrócie CIR jest nieznacznie uproszczoną wersją języka, reprezentowaną jako struktura danych.
Użytkownik wchodzi w interakcje z CIR za pomocą zestawu interfejsów opisanych w załączniku 1. Wszystkie typy instrukcji oraz wyrażeń mają ze sobą powiązany konkretny typ. Wyjątkiem jest \lstinline{ScopeTerminationStatement} który nie jest powiązany z żadną instrukcją pisaną przez użytkownika.
Instrukcja zakończenia zakresu jest wstawiana przez kompilator na koniec każdej instrukcji złożonej i jest odpowiedzialna za wywołanie destruktorów zmiennych lokalnych (opisane w rozdziale \ref{classes_definition}).

\subsection{Zarządzanie pamięcią}
Zarządzanie pamięcią w C-=-1 jest oparte na C++11. W 2011 do biblioteki standardowej zostały wprowadzone nowe typy 'inteligentnych wskaźników': \lstinline{unique_ptr}, \lstinline{shared_ptr} oraz \lstinline{weak_ptr}\cite{ISO:2012:III}.
Miały one na celu wprowadzenie do języka mechanizmów umożliwiających tanie i automatyczne zarządzanie pamięcią oraz semantyczne podkreślenie relacji między obiektami.
Korzystając z doświadczeń C++, gdzie te wskazania stały się zalecanym sposobem zarządzania pamięcią\cite{cpp:core_guidelines}, inteligentne wskazania są integralną częścią C-=-1.
\subsection{Struktura pakietu}\label{struktura_paczki}
Projekt w języku C-=-1 jest identyfikowany przez plik manifestu (ang: manifest) o rozszerzeniu .mn. %todo: whats in this file? what is it used for
Pliki źródłowe mają rozszerzenia .cm. W tym samym folderze co manifest znajduje się folder \lstinline{src}. Kompilator zakłada, że wszystkie pliki o rozszerzeniu \lstinline{.cm} będące jego potomkami należą do projektu definiowanego przez manifest.
W pliku .mn zdefiniowane są metadane na temat pakietu takie jak autor, wersja czy zależności. Ten aspekt języka jest modelowany na bazie Rust i pliku \lstinline{cargo.toml}.
 % wystarczy podmienić swoje pliki main.tex i eiti-thesis.cls
\section{Implementacja kompilatora i biblioteki standardowej C-=-1}
\label{implementation}
W celu przeprowadzenia prac badawczych na temat konsekwencji zaproponowanych mechanizmów koniecznym było zaimplementowanie podstawowego kompilatora. 
W tym rozdziale opisono implementacja kompilatora, która służyła do testów języka, oraz jej nietypowe aspekty, wynikające z projektu C-=-1.

\subsection{Fazy kompilacji}
\label{Compilation_phases}
Ze względu na nietypową konstrukcję C-=-1, proces kompilacji musi być dłuższy oraz ostrożniej zaprojektowany niż w typowym statycznie typowanym języku programowania.
Ponieważ użytkownik ma pisać kod, który będzie modyfikował wewnętrzne struktury danych kompilatora, musi on wiedzieć, kiedy które jej elementy są gotowe.
Kolejność operacji wykonywanych przez kompilator staje się przez to częścią języka.
Proces kompilacji składa się z następujących etapów:
\begin{enumerate}
  \item Budowa reprezentacji pośredniej atrybutów
  \item Zebranie definicji funkcji i typów
  \item\label{compilation_step:attribute_attachment} Przyłączenie atrybutów
  \item Budowa reprezentacji pośredniej funkcji i typów
  \item Wykonanie meta-funkcji atrybutów
  \item Zastąpienie wyrażeń zawierających statyczną refleksję ich wynikami
  \item Zapisanie reprezentacji pośredniej pakietu
  \item (opcjonalne) Konwersja do postaci pośredniej LLVM i kompilacja do kodu maszynowego
\end{enumerate}

Współczesne kompilatory są typowo skonstruowane z trzech części: front-end, middle-end i back-end \cite{intro_to_compiler_design}.
Zadaniem front-endu jest analiza składni, weryfikacja typów oraz budowa reprezentacji pośredniej kodu. Middle-end, operując niej, dokonuje optymalizacji niezależnych od maszyny docelowej.
Na koniec back-end optymalizuje kod pod kątem konkretnej architektury procesora i generuje kod maszynowy.

Ze względu na nietypowe wymagania C-=-1, zastosowano nowy podział odpowiedzialności.
W przeciwieństwie do typowych języków programowania implementowany kompilator nie może wygenerować reprezentacji pośredniej programu w jednym kroku.
Etap \ref{compilation_step:attribute_attachment} procesu kompilacji może wpływać na wybór przeciążenia funkcji.
Ten aspekt został opisany dokładniej w rozdziale \ref{Attributes_mechanism_cm1}

Kompilator C-=-1 składa się z następujących części:
\begin{enumerate}
  \item Frontend
  \item Interpreter
  \item Optimiser
  \item Serializator
  \item Backend Interface
  \item Backend
\end{enumerate}
Część z nich nazywa się tak samo, jak w klasycznej architekturze. Jest to zabieg celowy, ponieważ pełnią one te same funkcje. 
Dodatkowe komponenty kompilatora, które zostały wydzielone to: \emph{Interpreter}, \emph{Serializator} oraz \emph{Backend Interface}. 
Nazwa Optimiser została wybrana, ponieważ wyrażenie 'middle-end' rzadko występuje w literaturze, a wybrany termin lepiej opisuje ten komponent.

Ponieważ istotnym elementem języka C-=-1 jest możliwość wykonywania kodu w czasie kompilacji, jednym z wydzielonych komponentów jest Interpreter.
Operuje on na CIR omówionej w rozdziale \ref{reprezentacja_posrednia}.
Interpreter stanowi najważniejszy komponent kompilatora, ponieważ większość transformacji odbywa się za jego pomocą. 
Szczegółowe działanie tego komponentu jest opisane w rozdziale \ref{interpreter}.

\lstinline{Optimiser} został dodany do struktury kompilatora, aby zapewnić możliwość dalszego rozwoju i dla kompletności projektu. Nie został jednak zaimplementowany.

\lstinline{Serializator} jest odpowiedzialny za zapisywanie i odczytywanie struktur danych kompilatora z postaci tekstowej.
Serializowany w ten sposób pakiet można dystrybuować za pomocą serwisu takiego jak NPM \cite{npm} oraz używać w innych projektach.
Większość współczesnych języków programowania ma powiązany ze sobą system do zarządzania zależnościami, przechowujący listę publicznie dostępnych pakietów.
Ponieważ zawartością takiego modułu jest serializowane, niezależne od platformy CIR, nie istnieje problem binarnej kompatybilności.
Kompilacja programu używając pakietów C-=-1 jest zbliżona do kompilowania programu C++ używając zależności wyłącznie w formie kodu źródłowego.
Cały program jest kompilowany tym samym narzędziem, na tej samej maszynie i z tą samą konfiguracją.
Serializator został dokładnie opisany w rozdziale \ref{serializer}.

Kompilator C-=-1 używa LLVM \cite{Lattner:MSThesis02} jako back-endu. W związku z tym, koniecznym jest przetłumaczenie CIR na reprezentację pośrednią LLVM (w dalszej części pracy nazywanej LLVMIR). Dlatego do kompilatora dodano element \lstinline{Backend Interface}, który dokonuje tej konwersji. Obydwa te komponenty są opisane w rozdziałach \ref{Backend_Interface} oraz \ref{implementation:backend}.

Diagram \ref{compilation_process_diagram} przedstawia ogólny proces kompilacji.
Kod użytkownika jest dzielony na dwie części: atrybuty i funkcje oraz typy.
Ten podział wynika z konieczności analizy metakodu przed przetworzeniem normalnego kodu programu.


\begin{figure}[]
  \caption{Diagram procesu kompilacji języka C-=-1}
  \label{compilation_process_diagram}
  \includegraphics[width=\textwidth,height=\textheight,keepaspectratio]{img/compilation_process.png}
  \centering
\end{figure}

\subsection{Front-end}
Front-end kompilatora C-=-1 jest odpowiedzialny za analizę tekstu kodu źródłowego. Rysunek \ref{compilation_process_diagram} przedstawia diagram procesu kompilacji.
Front-end gra w nim kluczową rolę w początkowych fazach, przetwarzając zawartość plików źródłowych.
Sposób, w który źródła są odczytywane oraz analizowane jest opisany w rozdziale \ref{implementation:parser}.

Zrealizowanie tego procesu wymagało, aby front-end komponent kompilatora potrafił funkcjonować w różnych trybach w zależności od obecnie wykonywanego kroku.
Ponadto, ponieważ w C-=-1, w przeciwieństwie do C++, kolejność deklaracji nie ma znaczenia, front-end musiał radzić sobie z zależnościami kołowymi.
Ogół działania front-endu jest opisany w rozdziale \ref{implementation:source_processing_phases}.

Frontend kompilatora jest również odpowiedzialny za tworzenie instancji atrybutów, kiedy przetwarza kod funkcji i typów.
Dlatego program użytkownika jest dzielony, przed analizą.
Przetworzenie typów i funkcji wymaga możliwości wykonania kodu atrybutów (kroki i1 oraz i2 na diagramie \ref{compilation_process_diagram}).
Z tego powodu, kompilator musi analizować kod metaprogramu, przed resztą aplikacji.

\subsubsection{Parser}
\label{implementation:parser}
Do analizy tekstu wejściowego, użyto generatora parserów Rose\cite{grabski2020}.
Jest to narzędzie bazujące na Antlr \cite{antlr}, które znacząco ułatwia manipulowanie drzewem składniowym (ang. Abstract Syntax Tree, AST).
Te dodatkowe możliwości są wykorzystywane przy przetwarzaniu szablonów (rozdział \ref{implementation:generics}).
Kompilator trzyma w pamięci tylko jedno drzewo rozbioru na raz, co oznacza, że każdy plik jest wczytywany wielokrotnie.
Ta decyzja ma ograniczyć ilość zużywanej w danym momencie pamięci operacyjnej.

\subsubsection{Fazy przetwarzania plików źródłowych}
\label{implementation:source_processing_phases}

Frontend kompilatora C-=-1 operuje w trzech fazach: \lstinline{create}, \lstinline{confirm} oraz \lstinline{finalize}.
Aby zapewnić, że kolejność deklaracji w programie nie ma znaczenia (w przeciwieństwie do C/C++), fazy są wykonywane dla wszystkich plików.
Najpierw wszystkie pliki przechodzą przez \lstinline{create}, potem przez \lstinline{confirm} a na koniec przez \lstinline{finalize}.

Faza \lstinline{create} zbiera nazwy bytów wewnątrz kodu aplikacji.
Tworzy wszystkie przestrzenie nazw, typy, funkcje oraz atrybuty.
W tym kroku kompilator nie analizuje zawartości tych obiektów.
Nie bierze pod uwagę pól i metod typów oraz parametrów i typów zwracanych funkcji.
Jeżeli w kodzie użytkownika istnieje na przykład funkcja mająca cztery przeciążenia, po fazie \lstinline{create} w modelu semantycznym będą istnieć cztery identyczne kopie tej deklaracji.

W fazie \lstinline{confirm} deklaracje są uzupełniane.
Do deklaracji funkcji dodawane są parametry i typ zwracany.
Typy są uzupełniane o pola oraz implementowane interfejsy.
Ponieważ na tym etapie kompilator zna wszystkie nazwy typów występujących w programie, kolejność deklaracji nie ma znaczenia, a deklaracje zapowiadające nie są konieczne.

W fazie \lstinline{finalize}
analizowane są ciała funkcji oraz metod.
Ten krok wykonywany jest na końcu, aby upewnić się, że kompilator jest w stanie rozwiązać przeciążenia wszystkich funkcji, do których odwołania mogą się pojawić.

\lstinline{Frontend} kompilatora C-=-1 działa w trzech trybach, w zależności od fazy kompilacji.
Na diagramie \ref{compilation_process_diagram} oznaczone zostały jako f1, f2 i f3.
W każdej z nich frontend zachowuje się inaczej.
\begin{itemize}
  \item W trybie f1 kompilator przechodzi przez wszystkie fazy i operuje wyłącznie na kodzie atrybutów.
  Funkcje oraz typy są ignorowane.
  \item W trybie f2 frontend przechodzi tylko przez fazy \lstinline{create} oraz \lstinline{confirm}, tylko na kodzie typów i funkcji.
  Przy okazji, korzystając z zebranych już informacji w trybie f1, tworzone są instancje atrybutów.
  \item W trybie f3 frontend wykonuje wyłącznie krok \lstinline{finalize} na funkcjach oraz metodach
\end{itemize}


\subsubsection{Zarządzanie zależnościami}

Pomimo tego, że możliwość specyfikowania zależności pomiędzy pakietami jest częścią języka C-=-1, zarządzanie zależnościami zostało zaimplementowane w minimalnym stopniu.
W idealnej sytuacji kompilator operowałbym na zserializowanych, do formatu omówionego w rozdziale \ref{serializer}, bibliotekach, które mógłby pobierać ze specjalnego serwisu.
Format pliku manifestu pakietu, opisany w rozdziale \ref{struktura_paczki}, zawiera wszystkie informacje potrzebne do zbudowania drzewa zależności.
Biblioteki są identyfikowane przez nazwę, wersję oraz globalnie unikatowy identyfikator (ang: GUID) \cite{leach2005universally}.
Plik manifestu zawiera też listę zależności, potrzebnych, aby skompilować dany pakiet.
Obecnie, kompilator C-=-1 jest w stanie kompilować kod źródłowy, używając pakietów dostępnych lokalnie.

\subsubsection{Reprezentacja pośrednia}
\label{implementation:intermidiate_representation}
Reprezentacja pośrednia jest istotnym elementem języka, na którym opiera się metaprogramowanie w C-=-1.
W związku, z czym każda jej część jest opisana w dokumentacji.
Wewnątrz kompilatora, reprezentacja pośrednia jest przechowywana w ramach struktur danych interpretera (rozdział \ref{interpreter}), aby umożliwić jego modyfikację z interpretowanego programu.

Ponieważ użytkownik C-=-1 ma operować na CIR (C-=-1 Intermidiate Representation), jej struktura jest bliska językowi, który reprezentuje. 
Poza dodaniem specjalnych typów instrukcji struktura CIR jest niemal identyczna z C-=-1.
Takie podejście ma zarówno wady, jak i zalety. 
Z jednej strony sprawia ono, że błędy w kompilatorze są łatwiejsze do wykrycia, oraz front-end jest łatwiejszy do implementacji.
Przez podobieństwo reprezentacji pośredniej do kodu źródłowego, różnice względem poprawnego rezultatu są dużo bardziej oczywiste od bardziej abstrakcyjnych reprezentacji.
CIR jest jednak dużo trudniejsza do interpretowania.
W przeciwieństwie do języków takich jak CIL (Common Intermediate Language) \cite{ecma:cli}, CIR nie operuje abstrakcyjnej maszynie stosownej.
Szczegółowy opis działania interpretera znajduje się w rozdziale \ref{interpreter}.

Główne różnice w strukturze między C-=-1 a CIR polegają na bardziej dosłownym wyrażeniu programu. W reprezentacji pośredniej wszystkie odniesienia są w pełni kwalifikowane nazwą pakietu i przestrzenią nazw. Destruktory są wywoływane wprost, używając specjalnej instrukcji. Zamiast rozwiązywania przeciążeń funkcji, CIR odnosi się do konkretnego przeciążenia przez unikatowy identyfikator.

\subsubsection{Szablony}
\label{implementation:generics}
Szablony stanowią nietypowy element modelu semantycznego.
W kontekście projektu języka stanowią one źródło ciekawych wyzwań, opisanych w rozdziałach \ref{challenges:generic_instance_placement} oraz \ref{challenges:generic_function_limitations}.
W C-=-1 szablony działają na zasadach analogicznych do C++.
Szablon funkcji, na przykład, nie jest funkcją, a wzorcem, według którego procedura może dopiero powstać po pełnej specjalizacji.

Kompilator C-=-1 przechowuje generyki jako fragmenty drzewa rozkładu, ponieważ w obecnym projekcie reprezentacji pośredniej, nie ma potrzebnych struktur danych.
Frontend analizuje nagłówek szablonu oraz jego parametry i uzupełnia go na żądanie.
Przy odniesieniu do generyka, kompilator najpierw przegląda istniejące instancje szablonów.
Jeśli ta kombinacja parametrów jeszcze nie była używana, kopiuje drzewo rozkładu, zastępuje odniesienia do parametrów generycznych konkretnymi typami i przetwarza je jak zwykły plik źródłowy.

Strategia tworzenia instancji generyków, przyjęta w zaproponowanym kompilatorze wymaga starannego zarządzania kontekstem.
W C-=-1 obiekty pochodzące z innych przestrzeni nazw muszą być jawnie importowane w każdym pliku, który ich używa.
Po uzupełnienia drzewa rozkładu szablonu o wartości parametrów generycznych mogą się w nim znaleźć nazwy, które nie istniały w oryginalnym kontekście.
Z tego powodu, reprezentacja generyków strukturach danych kompilatora, przechowuje listę importów z pliku źródłowego, gdzie zadeklarowano szablon.
Przy jego uzupełnianiu, oryginalny kontekst, oraz kontekst, z którego dochodzi do wypełnienia, są łączone.
W ten sposób, w trakcie kompilacji instancji szablonu, ma dostęp do wszystkich potrzebnych nazw.

\subsection{Interpreter}
\label{interpreter}

Interpreter jest ważnym komponentem kompilatora C-=-1.
Diagram \ref{compilation_process_diagram} pokazuje, że w procesie kompilacji jest używany 3 razy (i1-i3).
Jest używany do konstrukcji atrybutów (i1), wykonywania operacji metaprogramistycznych (i2) oraz do generowania pliku wykonywalnego (i3).

Interpreter C-=-1 używa generycznych struktur danych opisanych w rozdziale \ref{implementation:interpreter:object_representation} zarówno jako danych, jak i reprezentacji wykonywanego programu.
Taka decyzja została podjęta, ponieważ częstym przypadkiem użycia tego komponentu jest modyfikacja programu, który niedługo będzie wykonywany.
Na przykład, jeśli kompilowany pakiet jest częścią większego projektu, kod modyfikowany w kroku i2, może, przy kompilacji następnego pakietu, zostać wykonany.
Wykonywanie kodu przez interpreter zostało dokładnie opisane w rozdziale \ref{implementation:interpreter:code_execution}.

To podejście ma jednak też swoje wady.
Ponieważ interpreter działa wyłącznie na generycznych strukturach danych, silne typowanie języka C++ nie jest w stanie wykryć pewnych błędów.
Ponadto, taka implementacja jest mniej efektywna niż użycie dedykowanych typów.

\subsubsection{Reprezentacja obiektów}
\label{implementation:interpreter:object_representation}

W kontekście działania interpretera C-=-1 jest bardziej podobny do dynamicznie typowanego języka skryptowego takiego jak Python \cite{van1995python} niż statycznie typowanego, kompilowanego języka jak C\# czy C++.
Interpreter działa na generycznych strukturach danych, typowych dla języka interpretowanego.
Rysunek \ref{implementation:data_structures:uml_diagram} przedstawia diagram UML klas używanych przez ten moduł.

\lstinline{ObjectValue}, \lstinline{IntegerValue}, \lstinline{StringValue} oraz \lstinline{ReferenceValue} odpowiadają typowym konstrukcjom występującym w językach obiektowych: obiektowi, liczbie całkowitej, napisowi oraz wskazaniu.
Natomiast \lstinline{GenericRuntimeWrapper} stanowi referencję do natywnego obiektu C++ i jest opisany w rozdziale \ref{implementation:interpreter:cpp_object_references}.


\begin{figure}
  \caption{Diagram UML struktur danych interpretera C-=-1}
  \label{implementation:data_structures:uml_diagram}
  \includegraphics[width=\textwidth]{interpreter_data_structures_uml.png}
\end{figure}

\subsubsection{Wykonanie kodu}
\label{implementation:interpreter:code_execution}

Interpreter C-=-1 jest środowiskiem uruchomieniowym, stanowiącym dużą warstwę abstrakcji od maszyny, na którym działa.
W tym trybie, z perspektywy programisty, nie istnieje stos, a obiekty nie mają adresów, ani wielkości.
Ten sposób wykonania kodu można porównać do sposobu, w który człowiek rozumie program.
Interpreter C-=-1 działa na zmiennych lokalnych, referencjach oraz stercie.

W kompilatorze C-=-1 istnieje globalna sterta obiektów interpretera.
Znajduje się na niej zarówno reprezentacja pośrednia programu, jak i dane alokowane przez interpretowany kod użytkownika.
W obecnej implementacji sterta jest niezarządzana i wymaga, aby programista ręcznie zwolnił pamięć.
To niedociągnięcie wynika z braku czasu i powinno zostać uzupełnione w przyszłej wersji kompilatora.
Dodanie odśmiecania pamięci może też pomóc w odnajdywaniu wycieków pamięci: jeśli ten sam kod jest wykonywany w czasie uruchomienia i kompilacji oraz, w czasie kompilacji wyciekła z niego pamięć, to ten sam efekt prawdopodobnie występuje w trakcie uruchomienia.
\begin{minipage}{\textwidth}
  
  \begin{lstlisting}[
    caption=Nagłówek głównej funkcji interpretera,
    label=interpreter_header
  ]
  std::unique_ptr<IRuntimeValue> execute
  (
    gsl::not_null<Function*> functionDefinition,
    std::vector<std::unique_ptr<IRuntimeValue>>&& arguments
  );
  \end{lstlisting}
  
\end{minipage}
Listing \ref{interpreter_header} zawiera nagłówek głównej funkcji interpretera.
To ona jest odpowiedzialna za wykonywanie kodu w kompilatorze C-=-1, niezależnie od kontekstu.
Ta funkcja przyjmuje dwa parametry: \lstinline{functionDefinition} będący wskazaniem na model semantyczny wykonywanej procedury oraz \lstinline{arguments} zawierający wektor wartości argumentów.

W trakcie wykonywania kodu funkcja \lstinline{execute} z listingu \ref{interpreter_header} zapewnia także obsługę błędów.
Już na wczesnym etapie projektu, zapewnienie użytkownikowi, chociaż tak podstawowej diagnostyki, jak stos wywołań, okazało się konieczne.
W tym celu, poza obsługą podstawowych wyjątków C++, dodano także obsługę przerwań związanych z błędem dostępu do pamięci, przy użyciu strukturyzowanej obsługi wyjątków, specyficznej dla platformy Windows \cite{structured_exception_handling, structured_exception_handling:microsoft}.
Procedura wykonania funkcji jest otoczona blokiem \lstinline{try}, obsługującym wyjątki C++ oraz blokiem \lstinline{__try} obsługującym przerwania systemu operacyjnego.
W razie błędu istnieją dwa scenariusze obsługi: jeśli został rzucony \\\lstinline{cMCompiler::dataStructures::RuntimeException}, nazwa obecnej funkcji i obecnie wykonywana instrukcja są dodawane do \lstinline{stacktrace} wyjątku.
W przeciwnym wypadku rzucany jest \\\lstinline{cMCompiler::dataStructures::RuntimeException} z komunikatem odzwierciedlającym źródło błędu.
Ten wyjątek jest następnie przechwytywany poza interpreterem i informacje w nim zawarte są wyświetlane użytkownikowi.
W ten sposób programista C-=-1 wie, w którym rejonie jego programu wystąpił błąd.

W interpretacji kodu istotną rolę gra typ \lstinline{cMCompiler::compiler::StatementEvaluator}.
Poza obsługą błędów funkcja \lstinline{execute} przygotowuje ten typ do wykonania procedury zdefiniowanej przez \lstinline{functionDefinition}, przechowuje wartości zmiennych lokalnych oraz śledzi obecnie wykonywaną instrukcję.
W celu stworzenia ewaluatora instrukcji najpierw tworzony jest ewaluator wyrażeń: \lstinline{ExpressionEvaluator}.
Ten typ jako zależność przyjmuje w konstruktorze funkcję znajdującą wartości zmiennych lokalnych.
\lstinline{ExpressionEvaluator} jest też używany przy tworzeniu instancji atrybutów do wyliczania wartości parametrów, co zostało opisane w rozdziale \ref{implementation:interpreter:attribute_attachment}.

%todo: more?

\subsubsection{Przyłączanie atrybutów}
\label{implementation:interpreter:attribute_attachment}

Funkcje atrybutów w C-=-1, opisane w rozdziale \ref{design:attributes:special_functions}, w czasie kompilacji, stanowią odpowiednik funkcji \lstinline{main} w czasie uruchomienia.
Stanowią punkty wejściowe dla kodu wykonującego operacje metaprogramistyczne oraz analizę statyczną.
W związku z tym, metody oraz konstruktory są wywoływane w 'próżni'.

W szczególności konstruktory atrybutów są wykonywane w nietypowy sposób.
Tworzenie instancji adnotacji to jedyny przypadek, w którym interpreter C-=-1 ewaluuje wyrażenia poza kontekstem funkcji, bez dostępu do żadnych zmiennych.


\subsubsection{Referencje do obiektów C++}
\label{implementation:interpreter:cpp_object_references}

Umożliwienie C-=-1 modyfikacji modelu semantycznego programu, wymaga udostępnienie referencji do natywnych obiektów języka implementacji kompilatora. %todo: jesus fucking christ, wording
Programista musi mieć na przykład dostęp do reprezentacji funkcji w modelu semantycznym.
Dlatego obiekt typu C-=-1 \lstinline{functionDescriptor} w pamięci kompilatora jest reprezentowany jako \lstinline{RuntimeFunctionDescriptor}, a nie jako \lstinline{ObjectValue}.
Ponieważ elementy modelu semantycznego są zarządzane przez kompilator, obiekty reprezentujące je, posiadają wyłącznie proste wskazanie.
W ten sam sposób obsłużone są referencje do elementów tworzonego przez \emph{backend interface} modułu, opisanego w rozdziale \ref{Backend_Interface}, przy użyciu typu \lstinline{GenericRuntimeWrapper}.
Hierarchia dziedziczenia klas używanych do reprezentowania obiektów C-=-1 została przedstawiona na rysunku \ref{implementation:data_structures:uml_diagram}

Niektóre referencje do obiektów C++ muszą jednak też zarządzać pamięcią.
Do przykładów należą: strumienie powiązane z otwartymi plikami czy kontekst LLVM.
Takie obiekty są zarządzane przez \lstinline{GenericOwningRuntimeWrapper}
Różnica między nim a \\\lstinline{GenericRuntimeWrapper} to typ używanego wskazania.
\lstinline{GenericOwningRuntimeWrapper} zamiast zwykłego wskazania, używa \lstinline{shared_ptr}.
W ten sposób, ten wrapper nie tylko przechowuje referencję na obiekt, ale zarządza też jego czasem życia.

\subsubsection{Biblioteka podstawowa}
\label{implementation:interpreter:basic_library}

Biblioteka podstawowa, w kontekście C-=-1, oznacza pakiet zawierający podstawowe funkcje oraz typy prymitywne.
W każdym języku programowania koniecznym jest specjalne traktowanie pewnego zbioru elementów programu takich jak prymitywne typy liczbowe czy funkcje arytmetyczne na nich.
W typowym kompilatorze wystarczy, że te obiekty zostaną uwzględnione przy generowaniu kodu.
Na przykład zamiast wywołania operatora plus dla dwóch liczb całkowitych, zostanie wstawiona instrukcja \lstinline{add}.

Ponieważ kompilator C-=-1 stanowi też jego pełne środowisko uruchomieniowe, musi zawierać wykonywalne definicje wszystkich operacji prymitywnych.
Ponadto, biblioteka podstawowa zawiera funkcję służące do komunikacji z kompilatorem, takie jak \lstinline{ignoreBody}, \lstinline{excludeAtRuntime}, \lstinline{excludeAtCompiletime}.
Służą one do modyfikowania metadanych funkcji przez atrybuty, co zostało dokładniej opisane w rozdziale \ref{Attributes_mechanism_cm1}.

Wszystkie funkcje z biblioteki podstawowej wymagają ''marshallingu'' - transformacji danych z jednej reprezetnacji w pamięci na drugą, pomiędzy interpretowanym C-=-1 a środowiskiem uruchomieniowym w C++.
Ponieważ interpretowany C-=-1 działa na strukturach danych, opisanych w rozdziale \ref{implementation:interpreter:object_representation}, niekompatybilnych z C++, nie można użyć prostego rzutowania (poza przypadkiem z rozdziału \ref{implementation:interpreter:cpp_object_references}).

\begin{minipage}{\linewidth}
\begin{lstlisting}[
  numbers=left,
  firstnumber=0,
  caption={Przykład budowania funkcji z biblioteki podstawowej C-=-1},
  aboveskip=0pt,
  label={lst:marshalling_cm1}
  ]
void completeBuildingType(gsl::not_null<Type*> type) {
...
  createCustomFunction(
    type
    ->append<Function>("methods")
    ->setReturnType(TypeReference {
      getCollectionTypeFor(TypeReference{
        getFunctionDescriptor(), 0 
      }),
      0 
      }),
    type,
    [](
      map<string, unique_ptr<IRuntimeValue>>&& parameters,
      map<string, not_null<Type*>>genericParameters)
    {
      auto self = dereferenceAs<RuntimeTypeDescriptor>(
        parameters["self"].get())->value();
      auto methods = self.type->methods();
      return convertCollection(methods, TypeReference{
        getFunctionDescriptor(),
        0
      });
    }
  )->setAccessibility(Accessibility::Public);
...
}
\end{lstlisting}
\end{minipage}

Listing \ref{lst:marshalling_cm1} zawiera przykład budowania funkcji będącej częścią biblioteki podstawowej.
\lstinline{createCustomFunction} oznacza wybraną  funkcję jako zaimplementowaną natywnie w C++.
W liniach od trzeciej do dziesiątej, do deskryptora typu, dodawana jest metoda o nazwie \lstinline{methods}, zwracająca kolekcję deskryptorów funkcji.
W liniach od dwunastej do dwudziestej trzeciej tworzony jest obiekt funkcyjny lambda, który będzie służył jako ciało tej funkcji.
Ta procedura ma dwa argumenty: \lstinline{parameters} oraz \lstinline{genericParameters}.

Pierwszy z parametrów zawiera argumenty wywołania, indeksowane po nazwie parametru.
Na przykład dla wywołania \lstinline{f(1 + 1)} funkcji \lstinline{int f(int number)}, parametr \lstinline{parameters} pod kluczem \lstinline{number} zawierałby wartość 2.
W przykładzie z listingu \ref{lst:marshalling_cm1}, ponieważ zastępowane jest ciało metody, dostępna jest zmienna \lstinline{self}, będąca odpowiednikiem \lstinline{this} z C++.
Drugi parametr zawiera parametry generyczne, używane przy implementacji szablonów.

Linie od szesnastej do dwudziestej drugiej listingu \ref{lst:marshalling_cm1} zawierają kod wykonywany zamiast metody \lstinline{methods}.
Jego ogólna struktura jest dosyć typowa dla pod-programów w bibliotece podstawowej.
W liniach szesnastej i siedemnastej dochodzi do konwersji (ang. marshalling)\cite{wiki:Marshalling} ze struktur danych interpretera C-=-1 do natywnego obiektu C++.
W ten sposób pobierany jest obiekt deskryptora typu na, którym została wykonana ta metoda.
Zmienna \lstinline{self} ma więc typ \lstinline{TypeReference}, składający się z poziomu referencji oraz odniesienia do typu w modelu semantycznym.
Następnie, w linii osiemnastej, wywoływana jest metoda \lstinline{methods}, pobierająca wszystkie metody tego typu.
Na koniec, w linii dziewiętnastej dochodzi do konwersji z natywnego \lstinline{vectora} C++ na kolekcję C-=-1.


\subsection{Serializator}
\label{serializer}

Serializator jest komponentem kompilatora, zapisującym i wczytującym reprezentację obiektów do i z tekstu.
Przy implementacji użyto biblioteki Json Nielsa Lohmanna \cite{lohmann}.
Każdy typ, który może być serializowany, ma metodę konwertującą obiekt na słownik klucz-wartość z tego pakietu.

Obiekty reprezentujące referencję (opisane w rozdziale \ref{implementation:interpreter:object_representation}) wewnątrz C-=-1 są traktowane wyjątkowo.
Serializacja odniesień do innych obiektów jest znanym problemem, rozwiązywanym za pomocą technik określanych jako szwindlowanie wskazań (ang: pointer swizzling) \cite{kemper1995swizzling}.
W trakcie generowania pliku JSON każdemu obiektowi jest przypisywany unikatowy identyfikator.
Referencje są serializowane do wartości tych identyfikatorów.

\subsection{Backend Interface}
\label{Backend_Interface}
Rolą interfejsu back-endu jest przetłumaczenie CIR na LLVMIR.
Na tym etapie kompilacji, cała reprezentacja pośrednia C-=-1 jest przechowywana w ramach struktur danych interpretera.
Stworzyło to okazję do napisania części kompilatora w języku docelowym.
To rozwiązanie zapewniło bezpieczeństwo typów w tej części kodu oraz ułatwiło rozwój C-=-1.
Tworzenie kompilatora w języku docelowym jest typowym w konstrukcji tego typu narzędzi.
Programy tak skonstruowane nazywają się 'Bootstrapping Compiler' \cite{puntambekar:compiler_design}. 

Implementacja tej części kompilatora w C-=-1 ułatwiła rozwój tego języka.
Jedną z głównych zalet narzędzia skonstruowanego w ten sposób jest istnienie dużego projektu w języku docelowym.
Wymusza to szybszy rozwój języka oraz umożliwia znajdowanie błędów w kompilatorze.

Razem z plikiem wykonywalnym kompilatora, dystrybuowany jest kod źródłowy pakietu, zawierającej \emph{interfejs backendu}. 
Przed kompilacją kodu użytkownika, kompilator wczytuje ten moduł.
Zaletą tego rozwiązania jest możliwość wymiany tego komponentu, bez rekompilacji całego narzędzia.
Po załadowaniu tego pakiet kompilator szuka w niej funkcji oznaczonej atrybutem \lstinline{compilerEntryPoint}, która służy jako punkt wejścia dla interfejsu. 
Ma ona za zadanie wygenerowanie LLVMIR dla przekazanej jej listy pakietów (LLVMIR została opisana w rozdziale \ref{implementation:backend}).

Punkt wejścia \emph{interfejsu backendu} jako argumenty przyjmuje kolekcję pakietów do skompilowania oraz obiekt \lstinline{CompilationResult}.
Wynik kompilacji odpowiada kontekstowi LLVM, śledzi tworzone moduły oraz globalny stan LLVM.
Poza sygnaturą punktu wejścia, programista ma też obowiązek zapewnić specjalne traktowanie niektórych typów i funkcji z biblioteki podstawowej C-=-1, opisanej w rozdziale \ref{implementation:interpreter:basic_library}

\emph{Interfejs backendu} zaimplementowany w ramach tej pracy jest bardzo prosty oraz niekompletny.
Nie wykorzystuje żadnych bardziej zaawansowanych mechanizmów LLVM oraz nie generuje metadanych dla debuggera.
Jego celem było zademonstrowanie możliwości języka C-=-1.

Najważniejszym przyjętym uproszczeniem jest kompilowanie całego programu jako jeden moduł LLVM.
Zgodnie z zamysłem tej biblioteki, moduł powinien reprezentować jednostkę kompilacji.
Traktowanie wszystkich pakietów, z których składa się program, jako jedną całość znacząco ułatwia jednak implementację.

Centralnym elementem zaproponowanej implementacji \emph{interfejsu backendu} jest typ \lstinline{PackageRegistry}.
Ten obiekt przechowuje mapowanie typów i funkcji reprezentacji pośredniej C-=-1 na typy oraz funkcje LLVMIR.

\emph{Interfejs backendu} generuje reprezentację pośrednią llvm, na podstawie każdej wykonywalnej w czasie uruchomienia funkcji we wszystkich pakietach.
Budowanie llvmir jest rekursywne: tworząc kod dla procedury \lstinline{A},zawierającej wywołanie lstinline{B}, przetłumaczy obydwie funkcje.
Jeśli później napotka \lstinline{B}, to użyje istniejącej już definicji.

Niektóre elementy programu są wyjątkowo traktowane w trakcie generowania LLVMIR.
Typy takie jak \lstinline{array<T>} oraz \lstinline{string} są przez język uznawane jako elementy wbudowane i muszą zostać przetłumaczone na prostsze konstrukty.
Tablice są kompilowane do struktury zawierającej wskazanie na pierwszy element oraz liczby elementów.
Natomiast \lstinline{string} jest reprezentowany w C-=-1 analogicznie do języka C: wskazanie na ciąg znaków, zakończony bajtem \lstinline{0x00}.
\emph{Interfejs backendu} uwzględnia też bardziej typowe elementy wbudowane takie jak typy liczbowe oraz proste operacje arytmetyczne.

Zaimplementowany moduł zawiera też obsługę generowania nagłówków dla innych języków programowania.
W trakcie generowania kodu funkcji, \emph{interfejs backendu} sprawdza czy została opatrzona atrybutem implementującym interfejs \lstinline{IBindingsGenerator} i wywołuje jego metodę \lstinline{generate}.
Generowanie plików nagłówkowych dla innych języków zostało dokładniej opisane w rozdziale \ref{comparison:exporting_to_other_languages}.

\subsubsection{Funkcje eksponowane przez kompilator}

Używając mechanizmów, opisanych w rozdziałach \ref{implementation:interpreter:basic_library} oraz \ref{implementation:interpreter:cpp_object_references}, udostępniono interfejsowi backendu narzędzie do budowy LLVMIR zdefiniowane w LLVM.
W ramach biblioteki backendu istnieje cała gama funkcji oraz typów używanych do generowania reprezentacji pośredniej, co zostało opisane w rozdziale \ref{implementation:backend}.
W języku C-=-1 dostępny jest minimalny ich podzbiór, konieczny do implementacji podstawowego kompilatora.


%todo: finish

\subsection{Backend}
\label{implementation:backend}
Do generowania kodu maszynowego została wykorzystana biblioteka LLVM \cite{Lattner:MSThesis02}.
Ten pakiet jest obecnie używany jako \emph{backend} całej rodziny języków C (C, C++, Objective-C) oraz języka Rust.
Aby obsługiwać tak szeroki zakres języków źródłowych, LLVM definiuje własną reprezentację pośrednią kodu: LLVMIR.

LLVMIR jest formą asemblera dla abstrakcyjnej maszyny stosowej \cite{llvmir}.
Program, zapisany w tej formie, jest dużo prostszy do automatycznej analizy i optymalizacji, oraz może zostać łatwiej skompilowany do kodu maszynowego dowolnego procesora.
Natomiast jego semantyczne podobieństwo do C, sprawia, że, stworzenie frontendu dla podobnego języka jest bardzo naturalne.

Maszyna abstrakcyjna LLVMIR operuje na typach, odpowiadającym strukturom z języka C bez nazwanych pól, oraz na funkcjach.
Funkcje mogą przyjmować nazwane parametry zarówno przez wartość, jak i wskazanie.
W LLVMIR nie istnieją wyrażenia złożone.
Wywołanie \lstinline{f(1 + 1)}, zapisane w C, należy wyrazić, najpierw jako deklaracja stałej tymczasowej \lstinline{%0 = add i32 1, 1} a następnie jako wywołanie \lstinline{call void @f(i32 %0)}, używając tej stałej.

W LLVMIR nie ma zmiennych.
Istnieją wyłącznie stałe, których nie można modyfikować po pierwszym przypisaniu.
Sposobem na zasymulowanie zachowania zmiennej z języka C, jest użycie instrukcji \lstinline{alloca}.
Alokuje ona miejsce na stosie, potrzebne dla danego typu i zwraca na nie wskazanie.
Programista może potem tę lokalizację w pamięci zmieniać oraz wczytywać jej wartość.

LLVM, poza kompilatorami LLVMIR do kodu maszynowego wielu różnych procesorów, zawiera też bibliotekę służącą do budowania LLVMIR.
Ten zestaw narzędzi jest bardzo przydatny, ponieważ dzięki niemu programista może operować na dobrze zdefiniowanej strukturze danych, zamiast na tekstowej reprezentacji.

\subsection{Biblioteka standardowa}

W celu przetestowania możliwości języka, w ramach pracy napisano minimalną bibliotekę standardową.
Zawiera głównie kolekcje, atrybuty oraz funkcje i typy potrzebne do dynamicznego zarządzania pamięcią.
Tworzenie tej biblioteki wymusiło rozwój języka oraz stanowiło test kompilatora.

Biblioteka standardowa C-=-1 zawiera dwa szablony kolekcji: \lstinline{List} oraz \lstinline{Dictionary}.
Pierwszy z nich jest prostą listą jednokierunkową, umożliwiającą dodawanie, zliczanie oraz uzyskiwanie dostępu do elementu po indeksie.
Słownik został zaimplementowany używając tej kolekcji, zamiast drzewa.
Ta decyzja sprawia, że to mapowanie jest znacznie wolniejsze niż podobne typy z bibliotek standardowych innych języków.
Znalezienie elementu po kluczu ma liniową złożoność obliczeniową, w porównaniu z logarytmiczną złożonością \lstinline{std::map} z C++ czy stałą złożonością \lstinline{System.Collections.Generic.Dictionary} z C\#.
Przy tworzeniu kolekcji dla biblioteki standardowej C-=-1, kładziono nacisk na nakład pracy programisty ponad wydajność.

Zarządzanie pamięcią w bibliotece standardowej zostało zbudowane na bazie funkcji języka C \cite{cLangStandard}.
Mechanizmy automatycznego zarządzania zasobami, opisane w rozdziale \ref{memory_management}, używają procedur \lstinline{malloc} oraz \lstinline{free}, zaimportowanych używając technik z rozdziału \ref{external_symbols}.

\subsection{Przyszła praca}

Zaproponowana implementacja kompilatora oraz bibliotek C-=-1, jest bardzo prosta i niekompletna.
Istnieje w niej wiele braków, które należy uzupełnić w ramach przyszłej pracy nad językiem.

Najważniejszym problemem jest brak funkcji do generowania kodu w bibliotece podstawowej.
Wiele z najciekawszych zastosowań C-=-1 polega właśnie na tej funkcjonalności i jej brak stanowi znaczącą przeszkodę przy analizie możliwości języka.

\section{Porównanie z innymi językami}
Ponieważ C-=-1 jest językiem badawczym, nie można go porównywać z językami stosowanymi w przemyśle pod względem wygody użycia.
Warto jednak przeanalizować, w jaki sposób zaproponowane mechanizmy wpływają na jego użyteczność.

\subsection{Wsparcie dla paradygmatów programowania}
Teoretycznie niemalże w każdym języku da się programować w każdym paradygmacie. Jednak w zależności od jego, może być to zadanie prostsze lub trudniejsze.
W niektórych wypadkach wymaga to wręcz użycia zewnętrznego narzędzia, poza kompilatorem \cite{aop:cpp}.
W tym rozdziale zostaną przedstawione paradygmaty programowania wspierane w C-=-1, bez dodatkowych narzędzi i ingerencji w język.

\subsubsection{Aspect Oriented Programming}
Nieoczekiwanym skutkiem zaproponowanych mechanizmów jest wsparcie dla programowania aspektowego (Aspect Oriented Programming, AOP) \cite{aop}.
Ta możliwość wynika z możliwości pisania samomodyfikującego kodu.

Programowanie aspektowe rozszerza typowy model programowania obiektowego o aspekty.
Składniowo, są podobne do klas.
Poza metodami i polami aspekty mogą deklarować również rady, służące do modyfikowania kodu w tak zwanych punktach cięcia.
Rada może wykonać dowolne operacje przed, po oraz zamiast wywołania funkcji zdefiniowanej przez punkt cięcia.

C-=-1 teoretycznie umożliwia tworzenie oprogramowania w paradygmacie AOP.
Jedyna przeszkoda przed praktycznym użyciem programowania aspektowego to niezaimplementowana biblioteka do generacji kodu.
Listing \ref{lst:aop_in_cm1} przedstawia przykład zastosowania AOP w C-=-1, używając hipotetycznych funkcji do generacji kodu.

%todo: make example
%todo: describe example
\begin{minipage}{\linewidth}
  
  \begin{lstlisting}[
    numbers=left,
    firstnumber=0,
    caption={Programowanie Aspektowe w C-=-1},
    aboveskip=0pt,
    label={lst:aop_in_cm1}
    ]
public att<function> NoDiscard
{
  public fn attach(f: functionDescriptor)
  {}
  public fn onCall(call: functionCallExpression*)
  {
  if(call._parentStatment != null<IInstruction>())
    raiseError(
      &(call._pointerToSource), 
      "Return value of a no-discard function is not used",
      123
    );
  }
}
\end{lstlisting}
\end{minipage}

\subsubsection{Obiektowe}
C-=-1 został zaprojektowany z myślą o wsparciu stylu obiektowego. Użytkownik może deklarować typy, metody oraz interfejsy. W przeciwieństwie do C++ czy C\#, w C-=-1 nie ma jednak koncepcji dziedziczenia między konkretnymi klasami.

Klasy i interfejsy mogą implementować inne interfejsy.
To jest jedyny mechanizm dynamicznego polimorfizmu w C-=-1.
Ta decyzja została podjęta częściowo w celu uproszczenia języka, a częściowo, ponieważ dziedziczenie między konkretnymi klasami jest uważane za złą praktykę \cite{gang_of_four:design_patterns}.

\subsection{Analiza kodu}
Podstawowym celem C-=-1 było zbadanie możliwości statycznej analizy kodu, używając zaproponowanych mechanizmów metaprogramowania.
W kolejnych rozdziałach omówione zostaną przykładowe analizatory, które zostały zaimplementowane w ramach biblioteki standardowej bądź których implementacja jest możliwa.
Ma to zademonstrować praktyczne aplikacje udostępnienia programiście modelu semantycznego tworzonego programu.



\subsubsection{Atrybut noDiscard}
\label{no_discard}

Atrybut \lstinline{noDiscard} służy do zapewnienia, że wynik funkcji nie zostanie zignorowany.
Tego typu adnotacja istnieje w C++ od wersji 17\cite{ISO:cpp17}.
Ma on na celu wyeliminować błędy takie jak ignorowanie kodu błędu, czy niepoprawne wywołanie funkcji o mylącej nazwie.

W C-=-1, w przeciwieństwie do większości języków niskiego poziomu, stworzenie analizatora, który zapewnia taki niezmiennik, jest trywialne.
Listing \ref{lst:noDiscardCm1} zawiera kod atrybutu \lstinline{noDiscard} z biblioteki standardowej C-=-1.
Linia czwarta deklaruje specjalną funkcję \lstinline{onCall} (specjalne funkcje atrybutów zostały opisane w rozdziale \ref{Attributes_mechanism_cm1}), która reaguje na wywołanie funkcji, do której został zaaplikowany.

Wewnątrz tego podprogramu, atrybut sprawdza, czy bezpośrednim rodzicem tego wyrażenia jest wyrażenie, czy instrukcja.
Jeśli jest nim instrukcja, oznacza to że wynik wywołania jest ignorowany i trzeba zgłosić błąd, używając funkcji \lstinline{raiseError}.
W przeciwnym wypadku żadne akcje nie są konieczne.

Na Listingu \ref{lst:noDiscardUsageCm1} przedstawiono zastosowanie atrybut \lstinline{noDiscard}.
Zignorowanie wartości zwracanej przez funkcję oznaczoną tą adnotacją, tak jak w linii piątej, powoduje zgłoszenie błędu o kodzie i komunikacie zgodnym z linią siódmą listingu \ref{lst:noDiscardCm1}.
Ponadto, kompilator otrzymuje wskazanie do pliku źródłowego na punkt, który wywołał ten błąd.
Ta informacja może być potem użyta do przekazania programiście dokładniejszej diagnostyki bądź do podkreślenia kodu w zintegrowanym środowisku programistycznym (rozdział \ref{IDE_integration}).

Niniejszy przykład dobrze ilustruje, że mając dostęp do modelu semantycznego programu, implementacja niektórych typów analizy staje się trywialna.
W większości języków programowania taka analiza wymaga modyfikacji kompilatora albo zewnętrznego narzędzia.

\begin{minipage}{\linewidth}
  
  \begin{lstlisting}[
    numbers=left,
    firstnumber=0,
    caption={Atrybut noDiscard w C-=-1},
    aboveskip=0pt,
    label={lst:noDiscardCm1}
    ]
public att<function> NoDiscard
{
  public fn attach(f: functionDescriptor)
  {}
  public fn onCall(call: functionCallExpression*)
  {
  if(call._parentStatment != null<IInstruction>())
    raiseError(
      &(call._pointerToSource), 
      "Return value of a no-discard function is not used",
      123
    );
  }
}
\end{lstlisting}
\end{minipage}


\begin{minipage}{\linewidth}
  
  \begin{lstlisting}[
    numbers=left,
    firstnumber=0,
    caption={Przykład użycia atrybut noDiscard w C-=-1},
    aboveskip=0pt,
    label={lst:noDiscardUsageCm1}
    ]
[noDiscard()]
fn noDiscardFunction() -> usize;

fn main() -> usize
{
  noDiscardFunction(); // error 123: Return value of
                       // a no-discard function is not use
  let x = noDiscardFunction();     // ok
  let y = x + noDiscardFunction(); // ok
  return noDiscardFunction();      // ok
}
\end{lstlisting}
\end{minipage}

\subsubsection{Atrybut const}
\label{const}

Ciekawszym przykładem analizatora, jest atrybut \lstinline{const}.
Odpowiada on modyfikatorowi \lstinline{const} z języka C++.
Zaaplikowanie go do typu oznacza, że do jego instancja jest niemodyfikowalna.
Każda próba modyfikacji, po zainicjalizowaniu, jest błędem.

W obecnym stanie kompilatora oraz języka, możliwa jest implementacja tylko części tej analizy.
Pełne wsparcie dla modyfikatora \lstinline{const}, przy użyciu atrybutu, wymaga od C-=-1 następujących właściwości:
\begin{enumerate}
  \item \label{prop:Attribute_function_overloading} Atrybuty są brane pod uwagę w trakcie wyboru przeciążenia funkcji
  \item \label{prop:Generic_adnotations} Istnieje sposób na modyfikowanie zachowania zmiennych, pól i parametrów w generykach, przy tworzeniu ich instancji
  \item \label{prop:Reference_adnotations} Przy aplikowaniu atrybutu do zmiennej typu referencyjnego, istnieje sposób na rozróżnienie między stałym wskazaniem a wskazaniem na stałą.
\end{enumerate}

Właściwość \ref{prop:Attribute_function_overloading} jest częściowo zaplanowaną, ale niezrealizowaną częścią języka (rozdział \ref{Attributes_definition}).
Atrybuty powiązane z parametrami funkcji mają być traktowane jako wymagania na wartościach przekazywanych przy jej wywołaniu.
Przykładowo, deklaracja \lstinline{fn positive_parameter_fun([positive()] usize paramter)} wymusza na wołającym funckję \lstinline{positive_parameter_fun}, aby przekazywana jej wartość była adnotowana atrybutem \lstinline{positive}.
Otwartym pytaniem pozostaje też, w jaki sposób traktować atrybuty przyjmujące parametr w kontekście wyboru przeciążenia funkcji.
Ten problem został dokładniej opisany w rozdziale \ref{further:adnotated_type_system:attribute_equivalence}
Zapewnienie tej właściwości wymaga ponadto dalszego rozszerzenia systemu atrybutów, tak aby można było aplikować adnotacje do wartości zwracanych z funkcji.

Ta potrzeba wywodzi się z tego, że dokładne znaczenie danej operacji w jest zależne od modyfikatorów zaaplikowanych do używanych wartości.
Listing \ref{lst:cpp_const_example} zawiera przykład użycia modyfikatora \lstinline{const} w C++.
Przeciążenie metody \lstinline{member} jest wybierane, biorąc pod uwagę czy parametr \lstinline{this} jest \lstinline{const}.
Tak więc w linii dziewiątej, wybierane jest przeciążenie nie-\lstinline{const} z linii trzeciej a w linii dziesiątej, przeciążenie \lstinline{const} z linii czwartej.
%todo: wording. example

\begin{minipage}{\linewidth}
  \begin{lstlisting}[
    numbers=left,
    firstnumber=0,
    caption={Przykład modyfikatora const w C++},
    aboveskip=0pt,
    label={lst:cpp_const_example}
    ]
  class Foo {
    int member_;
  public:
    int& member();
    int const& member() const;
  };
  int main() {
    Foo foo;
    Foo const constFoo;
    foo.member() = 3;     // ok
    constFoo.member() = 3;// error
  }
  \end{lstlisting}
\end{minipage}

Zapewnienie właściwości \ref{prop:Generic_adnotations}, jest najtrudniejsze i wymaga dalej idącej pracy teorytycznej.
Zaaplikowanie systemu atrybutów do implementacji modyfikatora \lstinline{const} oznacza śledzenie wartości, nie koniecznie pól, zmiennych, typów czy funkcji.
Listing \ref{lst:const_problem_example} przedstawia problem rozwiązywany przez właściwość \ref{prop:Generic_adnotations}.
W linii drugiej, deklarowana jest zmienna typu \lstinline{usize} z modyfikatorem \lstinline{const}.
Następnie, w linii czwartej, wskazanie na nią jest wstawiane do listy.
W tym miejscu analizator wygeneruje błąd kompilacji, ponieważ tworzymy nie-stałe wskazanie na stałą wartość, co może naruszyć chroniony niezmiennik.
Rozwiązaniem byłoby zadeklarowanie listy jako zawierającej wskazania na stałe.
W obecnej formie języka, nie jest to jednak możliwe.

\begin{minipage}{\linewidth}
  
  \begin{lstlisting}[
    numbers=left,
    firstnumber=0,
    caption={Śledzenie wartości, używając atrybutu const},
    aboveskip=0pt,
    label={lst:const_problem_example}
    ]
fn main() -> usize
{
  [const()] let x = 3;
  let list = List<usize*>();
  list.push(&x);
  [const()] let pointer = &x;
}
\end{lstlisting}
\end{minipage}


Właściwość \ref{prop:Reference_adnotations} jest blisko powiązana z właściowścią \ref{prop:Generic_adnotations}.
Rozróżnienie pomiędzy aplikacją atrybutu do wskazania a wskazywanej wartości, jest konieczne aby zapewnić nienaruszenie niezmienników dotyczących wartości.
Listing \ref{lst:const_problem_example} przedstawia ten problem.
W linii piątej, do zmiennej \lstinline{pointer} przypisywany jest adres zmiennej \lstinline{x}.
Pomimo zaaplikowania atrybutu \lstinline{const} do nowej zmiennej, analizator generuje błąd.
Adnotacja została tutaj zaaplikowana do wartości wskazania, nie do wskazywanej wartości.
W C-=-1 istnieją sposoby na obejście tego ograniczenia, jednak lepsze rozwiązanie powinno powstać w wyniku prac opisanych w rozdziale \ref{further:adnotated_type_system}.%todo: example?

Wszystkie te właściwości wskazują na silne powiązanie pomiędzy systemem adnotacji a typów.
Sposób na połączenie tych aspektów C-=-1 powinno zostać bardziej dogłębnie zbadane w przyszłej pracy.
Szczegóły tych badań zostały opisane dokładniej w rozdziale \ref{further:adnotated_type_system}.


W C/C++ modyfikator \lstinline{const} jest integralną częścią systemu typów.
Typ \lstinline{int*} jest różny od \lstinline{int const*} i konwersja między nimi istnieje tylko w jedną stronę.
Wszystkie problemy, wymienione w tym rozdziale nie mają odzwierciedlenia w C/C++, ponieważ ten modyfikator w nich jest częścią systemu typów, a nie rozszerzeniem języka.

\subsection{Generowanie kodu}

Model semantyczny programu, w języku C-=-1, jest mutowalną strukturą danych.
Oznacza to, że kod atrybutów może ją modyfikować w trakcie kompilacji, dodając, usuwając bądź zmieniając kolejność instrukcji i wyrażeń.
Ten aspekt języka, daje programiście szereg nowych możliwości.

\subsubsection{Atrybut Flags}

Jednym z najprostszych przykładów użycia możliwości generowania kodu, jest zadeklarowanie typu enumeracyjnego jako zbiór flag.
Aby to osiągnąć, wszystkie wartości enumeratora muszą być kolejnymi potęgami liczby 2, tak aby binarna alternatywa dowolnych dwóch opcji była unikatowo identyfikowalna.%todo: wording

W większości języków programowania, nie ma możliwości automatycznego przypisywania wartości enumeratora w ten sposób.
W C++ nie istnieje też żaden analizator który wykrywałby błędną definicję takiej flagi.
Przydzielenie wartości oraz weryfikacja czy są one poprawne, jest opdowiedzialnością programisty.

W C\# istnieje atrybut \lstinline{Flags}.
Służy on do walidowania wartości enumeratora i generuje ostrzerzenia w wypadku wykrycia błędu, ale nie przypisuje ich sam.
Programista nadal musi samodzielnie je ustalić.

%todo: cm1

\subsection{Testowanie kodu}

\subsection{Rozszerzalność języka}
\label{Language_extensibility}
Jedną z konsekwencji zaproponowanych mechanizmów jest możliwość rozszerzania języka bez modyfikacji kompilatora.
Oznacza to że pewne elementy składni, obecne w innych językach, stają się zbędnę w C-=-1.
\subsubsection{Symbole zewnętrzne}
Większość języków programowania zawiera mechanizm umożliwiający import symboli zewnętrznych.
Mogą one pochodzić z kodu napisanego w innym języku lub z już skompilowanej biblioteki.

W wypadku C/C++ wymaga to użycia słowa \lstinline{external}, do zdefiniowania symbolu zewnętrznego oraz przekazania odpowiedniego argumentu do linkera.
W C-=-1 nie ma słowa kluczowego \lstinline{extern}.
Deklaracja symbolu zewnętrznego wymaga stworzenia atrybutu, który zostanie potem obsłużony w interfejsie backendu.

Listing \ref{lst:extern_cm1} przedstawia przykładową implementację tej funkcjonalności.
Interfejs\\ \lstinline{ISymbolNameOverride} ma zwiększyć separację między interfejsem backendu a kodem kontrolującym nazwę symbolu.
Otwiera to drzwi większej ilości możliwych implementacji.

Atrybut \lstinline{MarkAsExternal} ma natomiast dwie role.
Po pierwsze, przechowuje nazwę symbolu, który jest importowany.
W ten sposób użytkownik nie jest przywiązany do nazwy symbolu zadeklarowanej w skompilowanym pliku bibliotecznym.
Umożliwia to też na importowanie funkcji z C++, których nazwy zostały zmieszane i zawierają znaki nieakceptowalne w identyfikatorach.

Po drugie, wywołuje \lstinline{ignoreBody}.
Ta funkcja informuje kompilator, że powinien zignorować ciało podprogramu i analizować wyłącznie nagłówek.
To wywołanie jest konieczne, ponieważ kompilator zapewnia poprawność programu, analizując zawartość funkcji.
Dlatego, jeśli funkcja zwracająca jakąś wartość, nie zawiera instrukcji \lstinline{return}, program zostanie odrzucony.
W wypadku podprogramów importowanych z zewnątrz, ich ciało jest zbędne.

Listing \ref{lst:selected_backend_interface} zawiera wybrane funkcje z domyślnego interfejsu backendu, powiązane z symbolami zewnętrznymi.
Funkcja \lstinline{getFunctionName} ma za zadanie wygenerować nazwę symbolu, kompatybilną z C.
Linie od drugiej do czwartej obsługują atrybuty implementujące \lstinline{ISymbolNameOverride}.
Sprawdzają one, czy z daną funkcją powiązany jest atrybut nadpisujący nazwę symbolu i zwraca wygenerowaną przez niego wartość.

Funkcja \lstinline{buildFunction} ma stworzyć reprezentację pośrednią LLVM funkcji przekazanej jako parametr.
Linie od dwudziestej dziewiątej do trzydziestej trzeciej zawierają kod obsługujący symbole zewnętrzne.
Te instrukcje sprawdzają, czy z funkcją powiązany jest atrybut implementujący \lstinline{ISymbolNameOverride}.
Jeśli tak, ciało procedury jest ignorowane.
W przeciwnym wypadku interfejs backendu buduje funkcję normalnie.
W LLVM, procedury niezawierające ciała, są uznawane za symbole zewnętrzne.

Zaprezentowana implementacja jest bardzo prosta, zmienia jedynie nazwę generowanego symbolu i ignoruje ciało budowanej funkcji.
Mogą jednak istnieć scenariusze, w których użycie symbolu zewnętrznego wymaga bardziej złożonej obsługi.
Przykładowo, jeśli wywołanie funkcji zdefiniowanej w innym języku wymaga marshallingu przekazywanych parametrów, bądź zwracanej wartości.
Po niewielkich modyfikacjach zaproponowanego interfejsu obsługa takich przypadków byłaby możliwa.

\begin{minipage}{\linewidth}
  
  \begin{lstlisting}[
    numbers=left,
    firstnumber=0,
    caption={Atrybut deklarujący funkcję jako symbol zewnętrzny},
    aboveskip=0pt,
    label={lst:extern_cm1}
    ]
public interface ISymbolNameOverride
{
  public fn createSymbolName() -> string;
}

public att<function> MarkAsExternal : ISymbolNameOverride
{
  private _symbolName : string;
  public fn construct(symbolName: string)
  {
    self._symbolName = symbolName;
  }

  public fn attach(f: functionDescriptor)
  {
    ignoreBody(f);
  }

  public fn createSymbolName() -> string
  {
    return self._symbolName;
  }
}
\end{lstlisting}
\end{minipage}

\begin{minipage}{\linewidth}
  
  \begin{lstlisting}[
    numbers=left,
    firstnumber=0,
    caption={Wybrane fragmenty interfejsu backendu},
    aboveskip=0pt,
    label={lst:selected_backend_interface}
    ]

private fn getFunctionName(f: functionDescriptor) -> string {
  let attribute = f.get_attribute<ISymbolNameOverride>();
  if(attribute != null<ISymbolNameOverride>())
    return attribute.createSymbolName();
  let baseName = f.qualifiedName();
  let params = f.parameters();
  for(i in enumerate(0, params.length()))
  {
    baseName = baseName + "__" + (params[i].type().toString());
  }
  return baseName
    .replace(":", "_")
    [...]
    .replace("@", "__at__");
}

private fn buildFunction(
  f: functionDescriptor,
  llvmF: llvmFunction,
  registry: packageRegistry*,
  mod: llvmModule)
{
  let variables = dictionary<variableDescriptor, llvmValue>();
  let params = f.parameters();
  for(i in enumerate(0, params.length()))
    variables.push(params[i], llvmF.getParameter(i));
  let builder = llvmF.getBuilder();
  let attribute = f.get_attribute<ISymbolNameOverride>();
  if(attribute != null<ISymbolNameOverride>())
  {
    let code = f.code();
    build_block(&code, &builder, &variables, registry);
  }
}

\end{lstlisting}
\end{minipage}


\section{Wyzwania}
\subsection{Instancje generyków jako element modelu semantycznego}
Gdzie wsadzić instancję generyka?

\subsection{Funkcje generyczne z ograniczeniami}

Funkcje mogą być wykluczane z użycia w trakcie uruchomienia lub kompilacji. Jeśli generyk takiej funkcji zostanie stworzony z typem dalej ograniczającym wykonywalność tej funkcji, ona może być wykonywalna nigdy.

Na wczesnych etapach implementacji, instancje takich generyków były szczególnie problematyczne, ze względu na to jak rozwiązywanie przeciążeń jest zaimplementowane w C-=-1 (rozdział \ref{Function_overload_resolution}).
W związku z tym, w wypadku użycia funkcji generycznej, kompilator powołuje instancje wszystkich wersji tego szablonu, ze zgodną ilością parametrów.
Niektóre z nich, mogą być niepoprawne w danym kontekście. 

Na przykład operator \lstinline{new unique} został zdefiniowany w dwóch wersjach: na czas uruchomienia oraz kompilacji.
W czasie uruchomienia, odwołuje się on do funckji \lstinline{unsafe_new} która alokuje zadaną ilość bajtów na stercie.
W czasie kompilacji wywołuje funkcję generyczną \lstinline{compiletime_heap_allocate} która zwraca referencję na instancję zadanego typu.
Szczegóły zostały opisane w rozdziale \ref{operator_new}.

Sygnatury tych operatorów niczym się nie różnią.
Język rozpoznaje je jako oddzielne, rozróżnialne byty wyłącznie dlatego, że atrybuty którymi został adnotowane zmieniają ich dostępność w czasie kompilacji i uruchomienia.

\subsection{Operator new}
\label{operator_new}
Operator \lstinline{new} w C-=-1, tak jak w C++ służy do dynamicznej alokacji pamięci na stercie.

\subsection{Wybór przeciążenia funkcji}
\label{Function_overload_resolution}
To czy funkcja jest wykonywalna w czasie kompilacji albo uruchomienia można ustalić bardzo późno w trakcie budowy modelu semantycznego.

Na pytanie "czy ta funkcja może być użyta w tym kontekście", kompilator jest w stanie odpowiedzieć dopiero kiedy wykonanły się na niej funkcje \lstinline{attach} wszystkich powiązanych z nią atrybutów 
\subsection{Wyrażenia literałowe}

Wyrażenie literałowe przedstawia stałą wartość.
Na przykład w języku C++ \lstinline{int a = 4;} deklaruje i inicjalizuje zmienną a literałem 4.
Wyrażenia lietrałowe mogą reprezentować stałe różnych typów wbudowanych.

\subsection{Integracja z back-endem}
Przy pierwszym podejściu do budowy reprezentacji pośredniej, użyto dedykowanych struktur danych. Było to podejście najbardziej naturalne i dające najwięcej bezpieczeństwa dzięki silnemu typowaniu. Wszystkie struktury danych opisane w rozdziale \ref{reprezentacja_posrednia} miały powiązane ze sobą klasy C++.


To podejście tworzy jednak duży problem. Reprezentacja pośrednia musi zostać wyeksponowana użytkownikowi. Ponieważ nie istniała możliwość stworzenia binarnego interfejsu między kompilatorem a interpretowanym kodem, struktury danych CIR musiały być dodatkowo reprezentowane za pomocą struktur danych interpretera.


Użytkownik może dokonywać modyfikacji w CIR, co oznacza że może dojść do rozbieżności między strukturami danych interpretera a kompilatora. Utrzymywanie tych dwóch reprezentacji stanowiło poważne wyzwanie, dlatego postanowiono zmienić podejście. Użycie wyłącznie struktur danych interpretera do reprezentacji CIR usunęło ten problem, kosztem bezpieczeństwa kodu.


Ponieważ w nowym podejściu, z perspektywy C++ niemalże wszystkie obiekty miały ten sam typ, kompilator stracił możliwość statycznej weryfikacji kodu.
Aby zapewnić poprawność programu, koniecznym stało się dodawanie walidacji argumentów do wszystkich funkcji.

\section{Wnioski z implementacji kompilatora C-=-1}

Celem C-=-1 nie było jednak zastąpić C++, Rust czy C\#, a zbadać jak priorytetyzacja wsparcia dla statycznego metaprogramowania wpływa na język oraz pisany w nim kod. 
Pod tym względem, można go uznać za sukces.
C-=-1, z oczywistych powodów, pod wieloma względami jest gorszy od tych języków.
Wiele udogodnień, traktowanych już jako standard, takich jak dedukcja parametrów generycznych czy semantyka przenoszenia nie zostało zaimplementowanych z powodu braku czasu.
Ponadto wsparcie dla C-=-1 jest minimalne: nie istnieje edytor, który koloruje jego składnie, a kompilator oferuje w najlepszym wypadku bardzo podstawową diagnostykę.
Na C-=-1 i narzędziami z nim związanymi, poświęcono niecałe dwa roboczolata, co jest bardzo małym nakładem pracy jak na język programowania.

Wiele aspektów tego języka, zostało niestety omówione wyłącznie na poziomie teoretycznym, ze względu na brak wsparcia kompilatora.
Stworzenie nowego języka programowania jest zadaniem dużo bardziej złożonym i czasochłonnym niż zakładano na początku tej pracy.
Z tego powodu, zabrakło czasu, aby zrealizować projekt C-=-1 w pełni.


\subsection{Kategoryzacja języka}

C-=-1 od samego początku był definiowany jako kompilowany język programowania ze wsparciem dla statycznego metaprogramowania i wykonywania kodu w czasie kompilacji.
Oryginalny projekt implementacji zakładał więc, że kompilator będzie zawierał w sobie moduł interpretujący reprezentację pośrednią programu.
Po wykonaniu transformacji na kodzie użytkownika kompilator miał przejąć kontrolę i dokonać kompilacji do kodu maszynowego.

Jednak w wyniku zmian opisanych w rozdziale \ref{Backend_Interface}, kompilacja do kodu maszynowego została zaimplementowana w interpretowanym C-=-1.
Oznacza to, że kompilator składa się wyłącznie z frontendu oraz środowiska uruchomieniowego które, udostępnia interfejs biblioteki llvm \cite{lattner2008llvm}.
Z tego powodu, w ogólnym przypadku, C-=-1 można zaklasyfikować jako silnie typowany język interpretowany.

Należy jednak pamiętać, że ten język zawiera w sobie rozróżnienie między kodem wykonywalnym w czasie uruchomienia a kompilacji.
Ponadto, jego projekt zawiera semantykę typową dla języków kompilowanych: jawną alokację obiektów na stercie, ręczne zarządzanie pamięcią oraz wskazania. 
Ten aspekt został opisany w rozdziale \ref{Language_desig}.
Tak więc najlepszym sposobem na zkategoryzowanie C-=-1 jest uznanie go za hybrydę pomiędzy językiem kompilowanym a interpretowanym.


\subsection{Rozszerzalność języka}

Umożliwienie programiście modyfikowanie kodu oraz tworzenie własnych adnotacji sprawia, że język można bardzo łatwo rozszerzyć o nowe mechanizmy.
W rozdziale \ref{Language_extensibility} opisano przykłady składni niewspieranej przez C-=-1, a którą można dodać przy użyciu proponowanego mechanizmu atrybutów.

Wykorzystując go, programista może zawęzić zbiór programów akceptowanych przez kompilator, przy użyciu analizy statycznej.
W rozdziałach \ref{no_discard} oraz \ref{const} przedstawiono przykłady adnotacji istniejących w C++ jako część języka, które można dodać do C-=-1.
Po dodaniu wyjątków trywialną będzie też implementacja atrybutu \lstinline{noexcept}, oznaczającego funkcje, która nigdy nie rzuca.

\subsection{Dyskusja rozwiązania}
\label{diadvantages}

Zaproponowane podejście niesie ze sobą pewne wady.
Być możne najpoważniejszą z nich jest utrudniony rozwój kompilatora.
Ponieważ model programu musi edytowalny przez interpretowany kod, należy go zbudować ze struktur danych interpretera.
Dokładne uzasadnienie i opis implementacji znajduje się w rozdziale \ref{backend_integration}.
Opisane tam wyzwania stanowią argument przeciwko konstruowaniu języków programowania w ten sposób. 
Chwilowo nie ma alternatywnej implementacji, która zachowuje silne typowanie w kodzie kompilatora.

Inny problem dotyczy samej konstrukcji języka.
Zaproponowane w tej pracy mechanizmy metaprogramistyczne, mogą prowadzić do zwiększenia poziomu złożoności tworzonego kodu.
Ponieważ atrybuty w C-=-1 mogą nadpisywać kod użytkownika, nadużycie tego mechanizmu może prowadzić do trudno przewidywalnego zachowania.
Ten problem staje się jeszcze poważniejszy, jeśli dany fragment programu jest modyfikowany przez więcej niż jeden atrybut.
W takiej sytuacji kolejność modyfikacji będzie wpływać na wynik kompilacji.
Jest to jeden z powodów, przez które generatory kodu w C\# nie mogą analizować wygenerowanego kodu \cite{roslyn:source_generators}.
Zasadność tej krytyki powinna być tematem przyszłych badań, używając bardziej kompletnej implementacji kompilatora.



\newpage % Rozdziały zaczynamy od nowej strony
\section{Summatio}          % Można też pisać rozdziały w jednym pliku.


%--------------------------------------------
% Literatura
%--------------------------------------------
\cleardoublepage % Zaczynamy od nieparzystej strony
\printbibliography

%--------------------------------------------
% Spisy (opcjonalne)
%--------------------------------------------
\newpage
\pagestyle{plain}

% Wykaz symboli i skrótów.
% Pamiętaj, żeby posortować symbole alfabetycznie
% we własnym zakresie. Ponieważ mało kto używa takiego wykazu,
% uznałem, że robienie automatycznie sortowanej listy
% na poziomie LaTeXa to za duży overkill.
% Makro \acronymlist generuje właściwy tytuł sekcji,
% w zależności od języka.
% Makro \acronym dodaje skrót/symbol do listy,
% zapewniając podstawowe formatowanie.
% //AB
\vspace{0.8cm}
\acronymlist
\acronym{EiTI}{Wydział Elektroniki i Technik Informacyjnych}
\acronym{PW}{Politechnika Warszawska}
\acronym{WEIRD}{ang. \emph{Western, Educated, Industrialized, Rich and Democratic}}

\listoffigurestoc     % Spis rysunków.
\vspace{1cm}          % vertical space
\listoftablestoc      % Spis tabel.
\vspace{1cm}          % vertical space
\listofappendicestoc  % Spis załączników

% Załączniki
\newpage
\appendix{Nazwa załącznika 1}


\newpage
\appendix{Nazwa załącznika 2}


% Używając powyższych spisów jako szablonu,
% możesz tu dodać swój własny wykaz bądź listę,
% np. spis algorytmów.

\end{document} % Dobranoc.
