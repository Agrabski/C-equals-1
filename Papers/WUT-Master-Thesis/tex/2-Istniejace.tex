\section{Istniejące rozwiązania}

Wśród przemysłowo stosowanych języków programowania istnieje szeroka gama mechanizmów metaprogramistycznych. Stosuje się w nich zarówno statyczną, jak i dynamiczną refleksję. W popularnie używanych językach mechanizmy dynamiczne są zwykle dużo bardziej rozbudowane, przyjazne użytkownikowi i potężne.

W tym rozdziale zostaną przedstawione wybrane języki programowania oraz dostępne w nich mechanizmy metaprogramistyczne.
Te istniejące rozwiązania będą stanowić podstawę dla analizy możliwości C-=-1.



\subsection{C\#}
Język C\# od początku swojego istnienia wspierał refleksję.
Środowisko uruchomieniowe DotNet zostało zaprojektowane z myślą o dynamicznym ładowaniu plików bibliotecznych.
W związku z tym większość mechanizmów metaprogramistycznych w językach opartych na nim jest używana w trakcie uruchomienia.
W ostatnich wersjach Microsoft dodał jednak pewne możliwości do generowania i analizowania kodu w trakcie kompilacji.
\subsubsection{Atrybuty}
W języku C\# atrybuty służą głównie do przechowywania metadanych.
Na rynku, istnieją rozwiązania, które wykorzystują takie adnotacje do modyfikacji programu, jednak takie zastosowanie nie jest proste.
Ponadto, jakiekolwiek zmiany w kodzie aplikacji, będą zachodzić w trakcie uruchomienia, nie kompilacji.
Oznacza to narzuty, które w wielu kontekstach są nieakceptowalne.
\subsubsection{Refleksja}

Informacja o organizacji programu jest integralną częścią skompilowanego pliku C\#.
W czasie uruchomienia aplikacja ma pełny wgląd we własną strukturę.
Ponieważ w C\# występuje dynamiczna refleksja, program ma również dostęp do metadanych w bibliotekach ładowanych w czasie pracy.

Ten mechanizm niesie ze sobą też pewne koszty.
Refleksja w czasie uruchomienia niesie za sobą poważne koszty, które w niektórych kontekstach są nieakceptowalne.
Ponadto, obecność informacji o typach i funkcjach w skompilowanym pliku znacząco ułatwia dekompilację.

Można też argumentować, że dostęp do metadanych programu w trakcie uruchomienia utrudnia optymalizacje.
Program może przeglądać własną strukturę i dynamiczne wywoływać kod na podstawie pozyskanych informacji.
W związku z tym niektóre typy optymalizacji takie jak \lstinline{inlining} nie są możliwe.

\subsubsection{Analizatory kodu}

Jednym z wyników publicznego udostępnienia kodu kompilatora Roslyn\cite{roslyn}, było udostępnienie programistom API kompilatora służącego do analizy kodu.
Programista może więc stworzyć własny analizator w ramach swojego projektu.
Zaletą tego podejścia  jest spójność używanego języka.
Projekt pisany w C\# może być analizowany w C\#, bez żadnych dodatkowych restrykcji.
Unika to problemu tworzenia dodatkowego dialektu języka, jaki występuje w C++.

\subsubsection{Generatory kodu}

C\#9 wprowadził możliwość pisania generatorów kodu. Jest to bardzo podobne rozwiązanie do makr w języku Rust. Generatory kodu to specjalne klasy, których kod jest wykonywany w trakcie kompilacji. 
Mechanizm C\# jest jednak bardziej zaawansowany.
Umożliwia generowanie kodu, używając wszystkich elementów języka, zamiast zapewniać prosty szablon.
Te generatory, tak jak w Rust \cite{rust}, nie mają jednak dostępu do informacji zbieranych przez kompilator, poza drzewem składniowym oraz nie mogą modyfikować istniejącego kodu.

Generatory kodu nadal są rozwijane i stanowią już część stabilnego C\#.
Stanowią one jednak ciekawą zmianę w kierunku rozwoju języka. 
Generatory kodu są pierwszym przejawem statycznego metaprogramowania w C\#

\subsection {Rust}
Ponieważ Rust \cite{rust} jest stosunkowo młody, jego projektanci mogli czerpać z doświadczeń innych języków. Wyraźnie widać to w jego składni, zintegrowanym systemie zarządzania zależnościami oraz bardzo przyjaznym kompilatorze.\par
Z tego powodu, nie zawiera on też tradycyjnego preprocesora, takiego jak w C/C++ czy C\#. Makra w Rust są tak naprawdę funkcjami operującymi na drzewie składniowym programu. Jest to zdecydowany postęp względem prostego zastępowania tekstu jak w C/C++, lecz ma też swoje wady.\par
Makra w języku Rust nadal operują na, co prawda strukturyzowanym, ale nadal, tekście. Oznacza to, że nie mają one dostępu do systemu typów ani żadnych innych informacji zgromadzonych przez kompilator.\par
\subsection{C++20}
C++ \cite{ISO:cpp20} zawiera szeroką gamę mechanizmów metaprogramistycznych oraz technik do statycznej analizy kodu. Były one stopniowo wprowadzane do języka przez cały czas jego istnienia, bez spójnego planu.\par
\subsubsection{Szablony}
Szablony w C++ można rozumieć jako oddzielny, funkcyjny, zdolny do symulacji maszyny Turinga, język programowania \cite{template_turing_complete}.
Metaprogramowanie szablonowe powstało nie jako świadomy wysiłek twórców języka, lecz jako naturalna, nieprzewidziana konsekwencja jego zasad.

\subsubsection{Koncepty}
Koncepty zostały wprowadzone w C++20 \cite{ISO:cpp20} jako sposób na lepsze dokumentowanie szablonów oraz czytelniejsze komunikaty o błędzie w wypadku niepoprawnego użycia.
Używa się ich do ograniczania, które typy mogą zostać wykorzystane w danym szablonie.

Koncepty są definiowane jako zbiór wymagań, które typ musi spełnić.
Mogą to być proste wyrażenia logiczne, jak i weryfikowanie czy dana metoda istnieje oraz jaki typ zwraca.
W przeciwieństwie do pozostałych omawianych mechanizmów koncepty bronią programistę przed niezrozumiałymi błędami kompilacji zamiast przed błędami czasu uruchomienia.

\subsubsection{Atrybuty}
Atrybuty zostały oficjalnie wprowadzone do języka w C++11\cite{ISO:2012:III}. 
Umożliwiają one przekazanie kompilatorowi dodatkowych informacji na temat kodu źródłowego.
Te dane są potem wykorzystywane w trakcie kompilacji do wydawania lub ignorowania ostrzeżeń.
Dwa przykłady warte omówienia to \lstinline{fallthrough} oraz \lstinline{no\_discard}, ponieważ prezentują obydwa przypadki zastosowania atrybutu.

Atrybut \lstinline{fallthrough} powstał, aby rozwiązać typowy problem języków zawierających konstrukcję \lstinline{switch}. W większości zastosowań celem programisty jest stworzenie zestawu niezależnych od siebie przypadków, spośród których zostanie wykonany jeden na podstawie wartości jakiejś zmiennej.\par
Niestety domyślnym zachowaniem konstrukcji \lstinline{switch}, po wykonaniu bloku jest przejście do następnego (tak zwany \lstinline{fallthrough}). Kompilatory zaczęły więc wydawać ostrzeżenie, jeśli blok case nie zostanie zakończony instrukcją \lstinline{break}. To tworzy jednak inny problem: czasami programista chce osiągnąć dokładnie takie zachowanie. Niektóre kompilatory zaczęły przez to zwracać uwagę na komentarze, szukając tam wyrażenia \lstinline{fallthrough} (w GCC flaga -Wimplicit-fallthrough=3) \cite{gcc_warnings}.\par
W C++11 sformalizowano to zachowanie, tworząc atrybut \lstinline{fallthrough} \cite{ISO:2012:III}. Zaaplikowanie go do przypadku konstrukcji \lstinline{switch}, powstrzymuje kompilator przed wydaniem ostrzeżenia.\par
Atrybut no\_discard, zamiast uciszać ostrzeżenie, generuje je. Jego zadaniem jest wykrywanie sytuacji, w których programista ignoruje wartość zwracaną przez funkcję. Można go zaaplikować zarówno do typów (wtedy każda funkcja zwracająca ten typ staje się no\_discard) lub do funkcji.\par
Typowym przykładem zastosowania tego atrybutu są kody błędów. Ze względu na duży koszt mechanizmu wyjątków w C++, w niektórych projektach się ich nie stosuje. Zamiast tego, funkcje których wykonanie może się nie powieść, zwracają strukturę informującą o błędzie. Ponieważ programista wołający taką funkcję, nie ma obowiązku sprawdzić wyniku tego wywołania, łatwo było o błąd. Stąd konieczność atrybutu \lstinline{no\_discard}.

Podstawową wadą atrybutów w C++ jest fakt, że są one zachowaniem bezpośrednio wpisanym w kompilator. Nie istnieje możliwość napisania własnego atrybutu a ich lista, jest bardzo krótka.\par
\subsubsection{Funkcje constexpr}
Język C++11 wprowadził koncepcję \lstinline{constexpr}. Zaaplikowanie tego modyfikatora do zmiennej lub funkcji, informuje kompilator, że mogą one być ewaluowane w trakcie kompilacji. Funkcje \lstinline{constexpr} mają też nałożony szereg ograniczeń co do swojej struktury. Z każdą kolejną wersją języka są one rozluźniane, jednak nie każda funkcja może być ewaluowana w trakcie kompilacji.

Ponadto, funkcje \lstinline{constexpr} nie mają większego dostępu do struktury programu niż normalny program. Nie dają one możliwości modyfikowania istniejącego ani generowania nowego kodu. Stanowią one więc jedynie sposób na optymalizację programu.
W związku z tym mechanizm \lstinline{constexpr} został skrytykowany za jawne definiowanie które funkcje są wykonywalne w czasie kompilacji.
Yauhen Klimiankou argumentował \cite{Klimiankou:contexpr_great_good_wrong_idea}, że wykonywalność funkcji w czasie kompilacji powinna być przejrzysta dla programisty i obsługiwana przez kompilator automatycznie.
