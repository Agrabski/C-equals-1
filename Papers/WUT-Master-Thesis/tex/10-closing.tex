\section{Słowa końcowe}

Prace C-=-1 oraz jego dokumentacją trwały od czternastego grudnia 2019, do trzydziestego stycznia 2022.
Ten język powstał jako rozwój wcześniejszego projektu Aergia \cite{grabski:aergia}, mającego na celu stworzenie preprocesora dla C++ dodającego statyczne metaprogramowanie.
Złożoność C++ znacząco utrudniała stworzenie kompletnego modelu programu w tym języku.
W związku z tym, aby móc skupić się na badaniu nowych technik metaprogramistycznych, postanowiono stworzyć nowy, prostszy język.
Jako pierwszy krok tego procesu, zmodyfikowano generator parserów ANTLR \cite{grabski2020}, aby ułatwić budowanie kompilatora.

Pomimo ponad dwóch lat pracy nad C-=-1 oraz siedmiu miesięcy przy Aergii (pierwszy kwietnia 2019 do siedemnastego listopada 2019), pozostało jeszcze wiele pracy do wykonania.
Generacja kodu w zaimplementowanym kompilatorze nie jest możliwa, w języku brakuje części konstruktów oraz wiele możliwości optymalizacji programu nie zostało zbadane.
Projekt kompilatora C-=-1 zawiera w przybliżeniu siedemdziesiąt dwa tysiące linii kodu, z których około dziesięciorga tysięcy zostało wygenerowanych przez ANTLR.



