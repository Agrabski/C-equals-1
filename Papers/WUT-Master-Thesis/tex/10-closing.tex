\section{Uwagi końcowe}

Prace C-=-1 oraz jego dokumentacją trwały od czternastego grudnia 2019, do trzydziestego stycznia 2022.
Ten język powstał jako rozwój wcześniejszego projektu Aergia \cite{grabski:aergia}, mającego na celu stworzenie preprocesora dla C++ dodającego statyczne metaprogramowanie.
Złożoność C++ znacząco utrudniała stworzenie kompletnego modelu programu w tym języku.
W związku z tym, aby móc skupić się na badaniu nowych technik metaprogramistycznych, postanowiono stworzyć nowy, prostszy język.
Jako pierwszy krok tego procesu, zmodyfikowano generator parserów ANTLR \cite{grabski2020}, aby ułatwić budowanie kompilatora.

Pomimo ponad dwóch lat pracy nad C-=-1 oraz siedmiu miesięcy przy Aergii (pierwszy kwietnia 2019 do siedemnastego listopada 2019), pozostało jeszcze wiele pracy do wykonania.
Generacja kodu w zaimplementowanym kompilatorze nie jest możliwa, w języku brakuje części konstruktów oraz wiele możliwości optymalizacji programu nie zostało zbadane.
Projekt kompilatora C-=-1 zawiera w przybliżeniu siedemdziesiąt dwa tysiące linii kodu, z których około dziesięciorga tysięcy zostało wygenerowanych przez ANTLR.

W ramach tej pracy zaimplementowano nowy język programowania, do przetestowania nowych mechanizmów metaprogramistycznych.
Atrybuty w C-=-1 stanowią nowy element języka, na równi z klasami i funkcjami, używany do adnotowania elementów programu.
Te adnotacje mogą następnie analizować i modyfikować powiązane z nimi fragmenty programu.
Atrybuty zostały dokładniej opisane w rozdziałach \ref{Attributes_definition} i \ref{Attributes_mechanism_cm1}.
Dla C-=-1 stworzono podstawowy kompilator oraz bibliotekę standardową, opisane w rozdziałach \ref{Language_desig} i \ref{implementation}.
Używając tych narzędzi, przeprowadzono następnie badania porównawcze z innymi językami programowania.

Pomimo relatywnie niewielkiego, jak na język programowania, nakładu pracy, poniżej dwóch roboczolat, C-=-1 oferuje nowe, ciekawe możliwości.
Jako język zbudowany wokół metaprogramowania, statyczna analiza kodu, oraz integracja z innymi językami jest dużo łatwiejsza.
Przy większym nakładzie pracy, możliwa byłaby też modyfikacja kodu w trakcie kompilacji, co znacząco zwiększyłoby możliwości języka.
Nawet w obecnej formie, zaproponowane mechanizmy metaprogramistyczne są użyteczne, co zademonstrowano w rozdziale \ref{comparison}.

Przy implementacji kompilatora przyjęto podejście oparte na użyciu interpretera C-=-1 do wykonania dużej części pracy.
Oznacza to, że komponent kompilatora został napisany w kompilowanym języku.
To podejście ma też jednak pewne wady.
Implementacja kompilatora staje się trudniejsza, przez porzucenie silnego typowania, co zostało dokładniej opisane w rozdziale \ref{diadvantages}.
Ponadto, czas kompilacji w obecnej wersji narzędzi C-=-1 jest długi, sięgając dziesięciu minut przy kompilowaniu biblioteki standardowej.
Te problemy wynikają w dużej mierze z braku czasu na dopracowanie języka oraz kompilatora.

C-=-1 wskazuje na użyteczność zaproponowanych mechanizmów metaprogramistycznych.
Język, pomimo braku dojrzałości, jest zdatny do użytku, a atrybuty oferują funkcjonalność przydatną w rozwiązywaniu praktycznych problemów.
Pełne zbadanie możliwości C-=-1 wymaga dalszej pracy nad kompilatorem.

%todo: jeszcze raz co zrobiono, wady, zalety
