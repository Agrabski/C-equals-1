\section{Projekt języka}
\label{Language_desig}
%todo: talk about limitations on interpreting and compilation, which functions can be executed, when
Język C-=-1 w przeciwieństwie do większości współczesnych języków programowania jest oparty na udostępnieniu programiście struktur danych tworzonych przez kompilator. Deskryptory typów, funkcji, przestrzeni nazw oraz enumeratorów, razem z reprezentacją pośrednią kodu są udokumentowaną częścią języka (załącznik nr 1).
Aby umożliwić użytkownikowi wykorzystanie tych struktur danych, C-=-1 musi zapewnić sposób na napisanie kodu wykonywanego w czasie kompilacji, który może nimi manipulować. W tym celu postanowiono rozszerzyć koncepcję atrybutu, aby umożliwić inspekcję i modyfikację reprezentacji pośredniej.
C-=-1 został zaprojektowany na bazie założeń C++. Obydwa te języki są statycznie typowane, kompilowane oraz nie wymagają specjalnego środowiska uruchomieniowego.
Przy projektowaniu C-=-1 skorzystano jednak z doświadczeń wyciągniętych z C++ w aspektach takich jak dziedziczenie wielobazowe czy zarządzanie pamięcią. Obydwa te aspekty zostały opisane odpowiednio w rozdziałach \ref{elementy_jezyka} oraz \ref{struktura_paczki}

\subsection{Składnia}

\subsection{Elementy języka}
\label{elementy_jezyka}

\subsubsection{Klasy}\label{klasy}
Klasy w języku C-=-1 działają analogicznie do języków z rodziny C.
Są to zbiory danych (pola klasy) z którymi powiązane są pewne operacje (metody).
Wszystkie elementy takiego typu mogą mieć ograniczenia dostępu.
Dwoma istotnymi elementami klas w C-=-1 są metody specjalne \lstinline{construct} oraz \lstinline{finalize}.
Można je zrozumieć jako odpowiedniki konstruktora oraz destruktora z C++.
Ponieważ C-=-1 w swojej obecnej formie nie wspiera wyjątków, nie gwarantuje wykonania finalizatorów zmiennych lokalnych w wypadku nieoczekiwanego zakończenia programu.
\subsubsection{Interfejsy}
\subsubsection{Funkcje}
W C-=-1 funkcje oraz metody działają na tej samej zasadzie co w C++.
Istotną różnicą jest możliwość ich wykonania w czasie kompilacji.
Domyślnie, funkcje mogą być wykonywane w dowolnym kontekście.
O ile programista nie sprecyzował żadnych ograniczeń, podprogram może zostać wykonany zarówno w czasie uruchomienia, jak i kompilacji.
Wywoływanie funkcji w trakcie kompilacji ma kilka różnych zastosowań.

Optymalizacja programu, historycznie stanowiła podstawową motywację dla wykonywania kodu w czasie kompilacji.
Jeśli wyrażenie zależy wyłącznie od stałych, kompilator może podjąć decyzję o jego ewaluacji do stałej.
C-=-1, w przeciwieństwie do C++, umożliwia optymalizację niemalże dowolnej funkcji w ten sposób.
Ponieważ programista nie musi jawnie precyzować, że procedura jest wykonywalna w czasie kompilacji, język C-=-1 unika problemów powiązanych z modyfikatorem \lstinline{constexpr} z C++ \cite{Klimiankou:contexpr_great_good_wrong_idea}
Taka możliwość tworzy także pewne wyzwania, dokładniej opisane w rozdziale \ref{compile_time_constant_evaluation}.

\subsubsection{Atrybuty}
\label{Attributes_definition}

Atrybuty, pod względem 
\subsection{Mechanizm atrybutów}
\label{Attributes_mechanism_cm1}
\subsection{Reprezentacja pośrednia}\label{reprezentacja_posrednia}
Programista C-=-1 może analizować oraz modyfikować reprezentację pośrednią swojego programu w czasie kompilacji. C-=-1 Intermidiate Representation, w skrócie CIR jest nieznacznie uproszczoną wersją języka, reprezentowaną jako struktura danych.
Użytkownik wchodzi w interakcje z CIR za pomocą zestawu interfejsów opisanych w załączniku 1. Wszystkie typy instrukcji oraz wyrażeń mają ze sobą powiązany konkretny typ. Wyjątkiem jest \lstinline{ScopeTerminationStatement} który nie jest powiązany z żadną instrukcją pisaną przez użytkownika.
Instrukcja zakończenia zakresu jest wstawiana przez kompilator na koniec każdej instrukcji złożonej i jest odpowiedzialna za wywołanie destruktorów zmiennych lokalnych (opisane w rozdziale \ref{klasy}).
\subsection{Zarządzanie pamięcią}
Zarządzanie pamięcią w C-=-1 jest oparte na C++11. W 2011 do biblioteki standardowej zostały wprowadzone nowe typy 'inteligentnych wskaźników': \lstinline{unique_ptr}, \lstinline{shared_ptr} oraz \lstinline{weak_ptr}\cite{ISO:2012:III}.
Miały one na celu wprowadzenie do języka mechanizmów umożliwiających tanie i automatyczne zarządzanie pamięcią oraz semantyczne podkreślenie relacji między obiektami.
Korzystając z doświadczeń C++, gdzie te wskazania stały się zalecanym sposobem zarządzania pamięcią\cite{cpp:core_guidelines}, inteligentne wskazania są integralną częścią C-=-1.
\subsection{Struktura pakietu}\label{struktura_paczki}
Projekt w języku C-=-1 jest identyfikowany przez plik manifestu o rozszerzeniu .mn. Pliki źródłowe mają rozszerzenia .cm. W tym samym folderze co manifest znajduje się folder \lstinline{src}. Kompilator zakłada, że wszystkie pliki o rozszerzeniu \lstinline{.cm} będące jego potomkami należą do projektu definiowanego przez manifest.
W pliku .mn zdefiniowane są metadane na temat pakietu takie jak autor, wersja czy zależności. Ten aspekt języka jest modelowany na bazie Rust i pliku \lstinline{cargo.toml}.
