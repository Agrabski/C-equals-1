\section{Projekt języka}
Język C-=-1 w przeciwieństwie do większości współczesnych języków programowania jest oparty na eksponowaniu programiście struktur danych tworzonych przez kompilator. Deskryptory typów, funkcji, przestrzeni nazw oraz enumeratorów, razem z reprezentacją pośrednią kodu są udokumentowaną częścią języka (załącznik nr. 1).
Aby umożliwić użytkownikowi wykorzystanie tych struktur danych, C-=-1 musi zapewnić sposób na napisanie kodu wykonywanego w czasie kompilacji, który może nimi manipulować. W tym celu, postanowiono rozszerzyć koncepcję atrybutu aby umożliwić inspekcję i modyfikację reprezentacji pośredniej.
C-=-1 został zaprojektowany na bazie założeń C++. Obydwa te języki są statycznie typowane, kompilowane oraz nie wymagają specjalnego środowiska uruchomieniowego. Przy projektowaniu C-=-1 skorzystano jednak z doświadczeń wyciągniętych z C++ w aspektach takich jak dziedziczenie wielobazowe czy zarządzanie pamięcią. Obydwa te aspekty zostały opisane odpowiednio w rozdziałach \ref{elementy_jezyka} oraz \ref{struktura_paczki}
\subsection{Składnia}
\subsection{Elementy języka}\label{elementy_jezyka}
\subsubsection{Klasy}\label{klasy}
Klasy w języku C-=-1 działają analogicznie do języków z rodziny C. Są to zbiory danych (pola klasy) z którymi powiązane są pewne operacje (metody). Wszystkie elementy takiego typu mogą mieć ograniczenia dostępu.
Dwoma istotnymi elementami klas w C-=-1 są metody specjalne construct oraz finalize. Można je zrozumieć jako odpowiedniki konstruktora oraz destruktora z C++. Ponieważ C-=-1 w swojej obecnej formie nie wspiera wyjątków, nie gwarantuje wykonania finalizatorów zmiennych lokalnych w wypadku nieoczekiwanego zakończenia programu.
\subsubsection{Interfejsy}
\subsubsection{Funkcje}
\subsubsection{Atrybuty}
\subsection{Mechanizm atrybutów}
\subsection{Reprezentacja pośrednia}\label{reprezentacja_posrednia}
Programista C-=-1 ma możliwość analizować oraz modyfikować reprezentację pośrednią swojego programu w czasie kompilacji. C-=-1 Intermidiate Representation, w skrócie CIR jest nieznacznie uproszczoną wersją języka, reprezentowaną jako struktura danych.
Użytkownik wchodzi w interakcje z CIR za pomocą zestawu interfejsów opisanych w załączniku 1. Wszystkie typy instrukcji oraz wyrażeń mają ze sobą powiązany konkretny typ. Wyjątkiem jest ScopeTerminationStatement które nie jest powiązane z żadną instrukcją pisaną przez użytkownika. Instrukcja zakończenia zakresu jest wstawiana przez kompilator na koniec każdej instrukcji złożonej i jest odpowiedzialna za wywołanie finalizatorów zmiennych lokalnych (opisane w rozdziale \ref{klasy}).
\subsection{Zarządzanie pamięcią}
Zarządzanie pamięcią w C-=-1 jest oparte na C++11. W 2011 do biblioteki standardowej zostały wprowadzone nowe typy „inteligentnych wskaźników”: unique\_ptr, shared\_ptr oraz weak\_ptr [7]. Miały one na celu wprowadzenie do języka mechanizmów umożliwiających tanie i automatyczne zarządzanie pamięcią oraz semantyczne podkreślenie relacji między obiektami.
Korzystając z doświadczeń C++, gdzie te wskazania stały się zalecanym sposobem zarządzania pamięcią [8], inteligentne wskazania są integralną częścią C-=-1.
\subsection{Struktura paczki}\label{struktura_paczki}
Projekt w języku C-=-1 jest identyfikowany przez plik manifestu o rozszerzeniu .mn. Pliki źródłowe mają rozszerzenia .cm. W tym samym folderze co manifest, znajduje się folder src. Kompilator zakłada że wszystkie pliki o rozszerzenia .cm będące jego potomkami należą do projektu definiowanego przez manifest.
W pliku .mn zdefiniowane są metadane na temat paczki takie jak autor, wersja czy zależności. Ten aspekt języka jest modelowany na bazie Rust i pliku cargo.toml.
