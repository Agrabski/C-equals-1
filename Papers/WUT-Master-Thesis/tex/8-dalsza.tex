\section{Rozwój języka}

Ta praca oferuje szereg ciekawych wniosków na temat zaproponowanego podejścia do projektowania języków programowania.
Pozostawia jednak też kilka otwartych pytań.
Odpowiedź na nie jest kluczowa dla dalszego rozwoju tej idei i stanowi barierę przed praktyczną jej aplikacją.

\subsection{Obsługa rozszerzeń języka}
\label{further:language_extension}
Pierwszy poważny problem jest powiązany z możliwością rozszerzania języka, opisaną w rozdziale \ref{Language_extensibility}.
Programista, tworzący własne adnotacje zapewniające statyczną analizę, nie ma obecnie możliwości adnotowania kodu bibliotecznego.
Problem, który z tego wynika można zilustrować na przykładzie atrybutu \lstinline{const}.

Załóżmy, że atrybut \lstinline{const} nie istnieje w bibliotece standardowej C-=-1.
Programista, pracujący nad projektem A, tworzy taką adnotację, aby poprawić jakość produkowanego kodu.
Problem pojawia się, kiedy próbuje użyć biblioteki standardowej.
Funkcja \lstinline{indexof}, na przykład, przyjmuje wskazanie na tablicę oraz na szukany element.
Ponieważ atrybut \lstinline{const} został zdefiniowany w bibliotece A, funkcje z biblioteki standardowej nie są nim adnotowane.

Oznacza to, że programista pracujący nad biblioteką A ma obecnie następujące alternatywy:
\begin{enumerate}
	\item\label{option:abandon_attribute} Porzucenie atrybutu \lstinline{const} i dalszą pracę bez analizy statycznej.
	\item\label{option:embed_in_attribute} Zaszycie które elementy biblioteki standardowej powinny mieć tą adnotację w kodzie atrybutu.
	\item\label{option:std_library_fascade} Stworzenie 'nakładki' na bibliotekę standardową.
	\item\label{option:analize_function_bodies} Implementacja atrybutu w taki sposób aby analizował też ciała wołanych funkcji.
	\item\label{option:own_std_lib} Implementacja własnej biblioteki standardowej.
\end{enumerate}
Alternatywa \ref{option:abandon_attribute} jest oczywiście najprostsza, ale sprawia, że proponowany mechanizm atrybutów staje bezużyteczny. 
Użytkownicy języka powinni mieć możliwość definiowania własnych analizatorów i ich zastosowanie bez ograniczeń.

Możliwości \ref{option:embed_in_attribute} oraz \ref{option:std_library_fascade} zapewniają używalność stworzonego atrybutu, ale wymagają dużej ilości pracy.
Programista musi uwzględnić w swoim kodzie, każdą funkcję ze wszytkich używanych bibliotek.
Poza nakładem pracy, te rozwiązania wymuszają ścisłe powiązanie pomiędzy biblioteką A a wszystkimi jej zależnościami.
Prowadzi to do drastycznego spatku utrzymwywalności projektu i zwiększenia kosztu dołączania kolejnych zależności.

Alternatywa \ref{option:analize_function_bodies} wydaje się być najlepsza. 
Wymaga ona bardzo dużego nakładu pracy, aby ustalić czy wywołanie danej funkcji wpływa na badany niezmiennik.
Koniecznośc przeprowadzenia dużo trudniejszej analizy, nie jest optymalna ale w obecnej formie mechanizmu atrybutów jest najlepszą możliwością.

W ramach opcji \ref{option:own_std_lib} programista może postanowić zaimplementować własną bibliotekę standardową.
To podejście, z oczywistych powodów w większości przypadków jest nieakceptowalne.
Nakład pracy, wymagana wiedza oraz dodatkowe możliwości popełnienia błędu niemalże gwarantują, że żaden zespół nie podejmie takiej decyzji.

Analogiczny problem powstał z wprowadzeniem nie-nullowalnych typów referencyjnych do języka C\# w wersji 8 \cite{wagner_2021, strauss2019new_cs_new_features}.
Zespół Microsoft podjął decyzję, aby traktować biblioteki niewspierające tej analizy w sposób pesymistyczny.
Wszystkie referencje pochodzące z takiego pliku dll, są uznawane za potencjalnie przyjmujące wartość null.

Warto zauważyć, że przypadek występujący w C\# jest jednym z prostszych.
Brakująca analiza dotyczy wyłącznie danych 'wypływających' z biblioteki.
Brak odpowiednich adnotacji nie wpływa natomiast na dane 'wpływające' do niej.
Dla porównania adnotacja \lstinline{const} wymaga analizy danych 'wpływających' do biblioteki (w języku C++ wartość typu \lstinline{int const*} jest nieużywalna jako parametr funkcji przyjmującej typ \lstinline{int*}).

\subsection{Modyfikowanie kodu}

Jednym z celów tej pracy było zbadanie, jak umożliwienie modyfikowania kodu w trakcie kompilacji, wpływa na tworzone oprogramowanie.
Niestety ze względu na błędną wycenę wielkości tego projektu, ten aspekt został pominięty.

Większość potrzebnych prac została już wykonana.
Model semantyczny programu jest trzymany w strukturach danych, które można modyfikować z interpretowanego kodu.
Proces kompilacji został zaprojektowany z uwzględnieniem zmian w reprezentacji pośredniej.

Aby skutecznie skorzystać z tej możliwości, należy jeszcze rozszerzyć bibliotekę podstawową języka (rozdział \ref{implementation:interpreter:basic_library}).
Obecnie programista może zamieniać albo usuwać istniejące elementy modelu semantycznego.
Aby umożliwić użytkownikom pełne wykorzystanie potencjału C-=-1, do biblioteki podstawowej należy dodać możliwość dynamicznego tworzenia elementów modelu.
W przyszłości, do języka trzeba będzie dodać funkcje do tworzenia nowych funkcji, instrukcji oraz wyrażeń.

\subsection{Ewaluacja stałych wyrażeń w trakcie kompilacji}
\label{compile_time_constant_evaluation}
Jednym z celów C-=-1 było umożliwienie ewaluacji dowolnie złożonego wyrażenia do stałej oraz wykonywania obliczeń i generowania dodatkowych plików w trakcie kompilacji.
Z tego powodu, większość funkcji można wykonać zarówno w czasie kompilacji, jak i uruchomienia.

Wśród nich są też procedury, mające efekty uboczne, na przykład interakcja z systemem plików czy nawet modyfikacja zmiennej globalnej.
Programista musi wiedzieć w którym etapie życia programu, kompilacji czy uruchomienia, te funkcje zostaną wykonane.
Jednak z punktu widzenia modelu semantycznego one w żaden sposób się nie wyróżniają.

Ewaluacja wyrażeń, zawierających taki kod, w czasie kompilacji, prowadziłaby do błędnego zachowania programu.
W czasie kompilacji dochodziłoby do zbędnych operacji na plikach, bądź nawet błędów kompilacji.
Natomiast w czasie uruchomienia, program pomijałby pewne operacje, operując na przykład na plikach dostępnych w czasie kompilowania.

Aby poprawnie zrealizować tą funkcjonalność, koniecznym jest wprowadzenie do języka wyróżnienia funkcji, które wpływają na, bądź zależą od globalnego stanu.
W tym celu można zastosować koncepcję 'czystej' funkcji z funkcyjnych języków programowania \cite{McLoughlin1989ImperativeEF}.
Dzięki temu, kompilator będzie w stanie ustalić, które wyrażenia może ewaluować jako część optymalizacji a które muszą pozostać w skompilowanym programie.

Takie rozwiązanie omija problem powstały z wprowadzeniem słowa kluczowego \lstinline{constexpr} w C++\cite{Klimiankou:contexpr_great_good_wrong_idea}.
Czystość funkcji niesie za sobą dużo więcej konsekwencji niż to, że funkcja może być częścią procesu optymalizacji.
Oznacza to, że programista nie musi być świadomy transformacji, które wykonuje kompilator.

\subsection{Integracja ze środowiskiem programistycznym}
\label{IDE_integration}

Language Server Protocol (LSP) \cite{bunder2019decoupling_lsp} to protokół umożliwiający rozwój narzędzi wspomagających programistę, niezależnie od zintegrowanego środowiska programistycznego.
Serwer Języka ma zapewniać usługi takie jak podświetlanie składni, uzupełnianie kodu, refaktoring czy wspomaganie nawigacji.

W wypadku większości języków programowania implementacja takiego serwera wymaga stworzenia dedykowanego oprogramowania.
Ponieważ kompilator C-=-1 stanowi środowisko urchomieniowe dla tego języka, z wbudowanymi narzędziami do analizy kodu źródłowego, można go wykorzystać w tym celu.
Takie zastosowanie wymaga pewnych zmian w kompilatorze, ale sam język jest do tego dobrze przystosowany.

\subsection{System typów a atrybuty}
\label{further:adnotated_type_system}

Wyzwania opisane w rozdziale \ref{const} sugerują, że atrybuty są silnie powiązane z systemem typów.
Podstawowym celem adnotacji aplikowanych do zmiennych, jest wymuszenie zachowania danych inwariantów.
W paradygmacie obiektowym ten cel jest zwykle osiągany za pomocą klasy \cite{oop_paradignm}.
W niektórych językach wspierających ten styl istnieją jednak też modyfikatory typów takie jak \lstinline{const} czy \lstinline{viotile} z C++ bądź \lstinline{mut} z Rust \cite{klabnik2019rust}.
Te adnotacje są częścią systemu typów i niosą ze sobą informacje semantyczne bądź wymuszają pewne inwarianty, których nie da się wyrazić za pomocą systemu typów.

Atrybuty z C-=-1, aplikowane do zmiennych i wartości, stanowią odpowiednik tych modyfikatorów, pisany przez użytkownika.
Przenoszą informacje semantyczne i wymuszają inwarianty.
Powinny więc stanowić rozszerzenie systemu typów, tak samo, jak modyfikatory \lstinline{const} czy \lstinline{viotile}.
Takie rozszerzenie języka, wymaga poważnego przeprojektowania modelu semantycznego programu oraz samej składni.
W szczególności należy rozwiązać problem przedstawiony w rozdziale \ref{const} oraz w listingu \ref{lst:const_problem_example}.
Obecnie, wskazania są reprezentowane jako para typ-licznik poziomu referencji.
Listing \ref{lst:further:type_reference_semantic_model} zawiera typ używany do reprezentacji wskazania w kompilatorze C-=-1.
Model semantyczny dostępny programiście C-=-1 zawiera bardzo podobną strukturę danych.

\begin{lstlisting}[
	numbers=left,
	firstnumber=0,
	caption={Obiekt reprezentujący odniesienie do typu w kompilatorze C-=-1},
	aboveskip=0pt,
	tabsize=2,
	label={lst:further:type_reference_semantic_model}
	]
struct TypeReference
{
	Type* type = nullptr;
	size_t referenceCount = 0;
	std::string typeName() const;
	bool isVoidType() const noexcept;
	TypeReference reference() const noexcept;
	TypeReference dereference() const noexcept;
	TypeReference baseType() const noexcept;
	TypeReference() = default;
	TypeReference(Type* t) noexcept;
	TypeReference(Type* t, size_t refCount) noexcept;
};
\end{lstlisting}

Prosta reprezentacja wskazań, taka jak w listingu \ref{lst:further:type_reference_semantic_model} uniemożliwia skuteczne wprowadzenie modyfikatorów typów takich jak \lstinline{const} z C++.
W tym podejściu typ \lstinline{int*} jest traktowany jako specjalny przypadek typu \lstinline{int}.
Z kolei \lstinline{int**} oraz \lstinline{int*} są postrzegane jako niemalże ten sam typ.
Aby umożliwić tworzenie modyfikatorów takich jak \lstinline{const}, typy \lstinline{int}, \lstinline{int*} oraz \lstinline{int**} powinny być traktowane na równi w modelu semantycznym.

\subsubsection{Równoważność atrybutów przy wybieraniu przeciążenia}
\label{further:adnotated_type_system:attribute_equivalence}

Zmiany w języku zaproponowane w rozdziale \ref{further:language_extension} wymagają dobrego zdefiniowania, w jaki sposób porównywać adnotacje aplikowane do parametrów.
Atrybuty w C-=-1 mogą być konstruowane z parametrami.
Przykład z listingu \ref{lst:further:attribute_comparison} przedstawia ten problem.
Atrybut \lstinline{GreaterThan} zapewnia, że zmienna, do której został przyłączony, jest większa od zadanej stałej.
Funkcja \lstinline{externalFunction} oczekuje natomiast parametru liczbowego o wartości większej niż pięć.
Jednak w funkcji \lstinline{function} przekazywana mu jest wartość większa niż siedem.
W wypadku tego wywołania inwariant \lstinline{GreaterThan} jest oczywiście zachowany, ale nie każdy sposób porównywania adnotacji doprowadzi kompilator do takiego wniosku.

Najprostszy sposób porównywania instancji atrybutów sprawdzanie ich typów.
Według tego algorytmu wszystkie wartości adnotowane jako \lstinline{GreaterThan} byłyby akceptowalnym argumentem wywołania funkcji \lstinline{externalFunction} z listingu \ref{lst:further:attribute_comparison}.
To podejście sprawia, że parametryzowane adnotacje zmiennych tracą jakikolwiek sens.
Niewykluczone, że możliwość definiowania takich atrybutów w praktycznym zastosowaniu jest zbędne.
Zasadność takiej konstrukcji powinna być przedmiotem przyszłych badań.

\begin{minipage}{\textwidth}
	
	\begin{lstlisting}[
		numbers=left,
		firstnumber=0,
		caption={Przykład atrybutu w C-=-1},
		aboveskip=0pt,
		label={lst:further:attribute_comparison}
		]
	att<variable> GreaterThan {
		private _limit: usize;
		public fn construct(limit: usize)
		...
	}
	public fn greaterThan<limit: usize>(value: usize) -> 
		[GreaterThan(limit)] usize {
		assert(value > limit);
		return value;
	}
	public fn getParameter() -> usize;
	public fn externalFunction([GreaterThan(5)]parameter: usize);
	
	public fn function() {
		let parameter = getParameter();
		externalFunction(greaterThan<7>(parameter));
	}
	\end{lstlisting}
	
\end{minipage}
Alternatywnym podejściem do problemu porównywania atrybutów jest rozszerzenie zbioru ich metod o \lstinline{satisfiesInvariant(value: valueDescriptor*) -> bool}.
Ta funkcja byłaby wywoływana na atrybucie powiązanym z parametrem wywoływanej procedury.
Ta metoda ma za zadanie zweryfikować czy dana wartość spełnia wymagania narzucone przez daną adnotację.
Takie rozwiązanie daje duże możliwości przy definiowaniu zasad, według których działa dany inwariant.
Może to jednak prowadzić do nieprzewidywalnego wyboru przeciążenia funkcji oraz do sytuacji, w których tego wyboru nie da się dokonać jednoznacznie.
Zapewnienie programiście dobrej diagnostyki, zawierającej informacje, dlaczego dane wywołanie jest niepoprawne.

Listing \ref{lst:further:semantic_model_extension} zawiera deklaracje typów, o które należy rozszerzyć model semantyczny C-=-1, aby dodać do atrybutów metodę \lstinline{satisfiesInvariant}.
Typ \lstinline{valueDescriptor} opisuje wartość wyrażenia na etapie kompilacji.
Zawiera informacje o tym, czy jest to stała, na podstawie jakiego wyrażenia powstała oraz jakie atrybuty są z nią powiązane.
Instancje tego typu będą tworzone przez kompilator, w celu wywołania metody \lstinline{satisfiesInvariant}, w trakcie wyboru przeciążenia funkcji.
Klasa \lstinline{attributeInstance} zawiera natomiast informacje o instancji atrybutu.
Składa się z referencji do typu atrybutu oraz wskazanie na obiekt będący jego instancją.
Adnotacje mogą użyć \lstinline{attributeInstance} aby łatwiej identyfikować, które atrybuty powiązane z wartością analizować.

\begin{minipage}{\textwidth}
	
\begin{lstlisting}[
	numbers=left,
	firstnumber=0,
	caption={Proponowane rozszerzenie modelu semantycznego C-=-1},
	aboveskip=0pt,
	label={lst:further:semantic_model_extension}
	]
public class attributeInstance {
	public attribute: attributeDescriptor;
	public instance: void*;
}
public class valueDescriptor {
	public type: typeReference;
	public attributes: attributeInstance[];
	public isConstant: bool;
	public constantValue: void*;
	public sourceExpression: IExpression*;
}
\end{lstlisting}
	
\end{minipage}

Ten aspekt języka jest ciekawym, ale także trudnym problemem do rozwiązania.
Należy znaleźć balans, pomiędzy zapewnieniem użytecznego mechanizmu do dostosowywania algorytmu wyboru przeciążeń a zdolnością programisty do zrozumienia tworzonego kodu oraz generowanych przez kompilator diagnostyk.

\subsection{Reprezentacja pośrednia generyków}
\label{further:generic_intermidiate_representation}

W obecnej specyfikacji języka programista ma bardzo ograniczony wgląd w strukturę typów i funkcji generycznych.
Wynika to z konfliktu pomiędzy precyzją reprezentacji pośredniej C-=-1 a niepewnością istniejącą w szablonach.
Listing \ref{lst:further:generic_imprecision} zawiera przykład demonstrujący ten problem.
Kod tego generyka jest poprawny, jednak instrukcja w linii pierwszej nie zawiera dostatecznie dużo informacji, aby określić która metoda zostanie wywołana.
Kompilator wie jedynie, że wywoływana metoda nazywa się \lstinline{foo}.
W normalnej reprezentacji pośredniej C-=-1 taka sytuacja jest niedozwolona.
\begin{lstlisting}[
	numbers=left,
	firstnumber=0,
	caption={Przykład niedoprecyzowania generyka w C-=-1},
	aboveskip=0pt,
	label={lst:further:generic_imprecision}
	]
public fn genericFunction<T>(value: T*, text: string) -> usize {
	value.foo();
	return text.length();
}

\end{lstlisting}

W celu ograniczenia zakresu pracy, w obecnej definicji języka, programista ma dostęp jedynie do nazwy generyka oraz jego listy parametrów.
Implementacja kompilatora trzyma szablony jako metadane oraz drzewo AST, które jest modyfikowane przy tworzeniu instancji, a następnie analizowane jak normalny plik źródłowy.

Zapewnienie programiście możliwości analizy kodu generyków wymaga rozszerzenia specyfikacji języka o model kodu generycznego.
Opis szablonów wymaga mniej precyzyjnych struktur danych.
Na przykład powinien powstać odpowiednik typu \\\lstinline{functionCallExpression}, \lstinline{genericFunctionCallExpression}, który zamiast wskazywać na konkretną funkcję, przechowuje jedynie jej nazwę.
W przykładzie z listingu \ref{lst:further:generic_imprecision} zostałyby wykorzystane obydwa typy wyrażeń.
\lstinline{genericFunctionCallExpression} w linii pierwszej i \lstinline{functionCallExpression} w linii drugiej.

Obecnie, ponieważ w modelu semantycznym istnieje bardzo ograniczona ilość informacji o generykach, nie można przyłączyć atrybutu do szablonu.
Wszystkie adnotacje są uznawane za dotyczące konkretnej instancji.
Rozszerzenie modelu o reprezentację kodu generyka, wymaga dodania możliwości adnotowania samych szablonów.
Wymaga to dodania dwóch nowych celów dla atrybutów: \lstinline{genericFunction} oraz \lstinline{genericClass}.
Ponadto, należy zaproponować składnię umożliwiającą rozróżnienie adnotacji aplikowanej do generyka a do jego instancji.
