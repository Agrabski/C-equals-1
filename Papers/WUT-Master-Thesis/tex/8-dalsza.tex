\section{Dalsza praca}

Ta praca oferuje szereg ciekawych wniosków na temat zaproponowanego podejścia do projektowania języków programowania.
Pozostawia jednak też kilka otwartych pytań.
Odpowiedź na nie jest kluczowa dla dalszego rozwoju tej idei i stanowi barierę przed praktyczną jej aplikacją.

\subsection{Obsługa rozszerzeń języka}
Pierwszy poważny problem jest powiązany z możliwością rozszerzania języka, opisaną w rozdziale \ref{Language_extensibility}.
Programista, tworzący własne adnotacje zapewniające statyczną analizę, nie ma obecnie możliwości adnotowania kodu bibliotecznego.
Problem, który z tego wynika można zilustrować na przykładzie atrybutu \lstinline{const}.

Załóżmy, że atrybut \lstinline{const} nie istnieje w bibliotece standardowej C-=-1.
Programista, pracujący nad projektem A, tworzy taką adnotację, aby poprawić jakość produkowanego kodu.
Problem pojawia się, kiedy próbuje użyć biblioteki standardowej.
Funkcja \lstinline{indexof}, na przykład, przyjmuje wskazanie na tablicę oraz na szukany element.
Ponieważ atrybut \lstinline{const} został zdefiniowany w bibliotece A, funkcje z biblioteki standardowej nie są nim adnotowane.

Oznacza to, że programista pracujący nad biblioteką A ma obecnie następujące alternatywy:
\begin{enumerate}
	\item\label{option:abandon_attribute} Porzucenie atrybutu \lstinline{const} i dalszą pracę bez analizy statycznej.
	\item\label{option:embed_in_attribute} Zaszycie które elementy biblioteki standardowej powinny mieć tą adnotację w kodzie atrybutu.
	\item\label{option:std_library_fascade} Stworzenie 'nakładki' na bibliotekę standardową.%todo: lepsze słowo% 
	\item\label{option:analize_function_bodies} Implementacja atrybutu w taki sposób aby analizował też ciała wołanych funkcji.
	\item\label{option:own_std_lib} Implementacja własnej biblioteki standardowej.
\end{enumerate}
Alternatywa \ref{option:abandon_attribute} jest oczywiście najprostsza, ale sprawia, że proponowany mechanizm atrybutów staje bezużyteczny. 
Użytkownicy języka powinni mieć możliwość definiowania własnych analizatorów i ich zastosowanie bez ograniczeń.

Możliwości \ref{option:embed_in_attribute} oraz \ref{option:std_library_fascade} zapewniają używalność stworzonego atrybutu, ale wymagają dużej ilości pracy.
Programista musi uwzględnić w swoim kodzie, każdą funkcję ze wszytkich używanych bibliotek.
Poza nakładem pracy, te rozwiązania wymuszają ścisłe powiązanie pomiędzy biblioteką A a wszystkimi jej zależnościami.
Prowadzi to do drastycznego spatku utrzymwywalności projektu i zwiększenia kosztu dołączania kolejnych zależności.

Alternatywa \ref{option:analize_function_bodies} wydaje się być najlepsza. 
Wymaga ona bardzo dużego nakładu pracy, aby ustalić czy wywołanie danej funkcji wpływa na badany niezmiennik.
Koniecznośc przeprowadzenia dużo trudniejszej analizy, nie jest optymalna ale w obecnej formie mechanizmu atrybutów jest najlepszą możliwością.

W ramach opcji \ref{option:own_std_lib} programista może postanowić zaimplementować własną bibliotekę standardową.
To podejście, z oczywistych powodów w większości przypadków jest nieakceptowalne.
Nakład pracy, wymagana wiedza oraz dodatkowe możliwości popełnienia błędu niemalże gwarantują, że żaden zespół nie podejmie takiej decyzji.

Analogiczny problem powstał z wprowadzeniem nie-nullowalnych typów referencyjnych do języka C\# w wersji 8 \cite{wagner_2021}.
Zespół Microsoft podjął decyzję aby traktować biblioteki nie wspierające tej analizy w sposób pesymistyczny.
Wszystkie referencje pochodzące z takiego pliku dll, są uznawane jako potencjalnie przyjmujące wartość null.

Warto zauważyć, że przypadek występujący w C\# jest jednym z prostszych.
Brakująca analiza dotyczy wyłącznie danych 'wypływających' z biblioteki.
Brak odpowiednich adnotacji nie wpływa natomiast na dane 'wpływające' do niej.
Dla porównania, adnotacja \lstinline{const} wymaga analizy danych 'wpływających' do biblioteki (w języku C++ wartość typu \lstinline{int const*} jest nieużywalna jako parametr funkcji przyjmującej typ \lstinline{int*}).
%todo: cite C#8 nullable reference types

\subsection{Adnotacje a generyki}
\label{Generic_adnotations}
trzeba dorzucić możliwość aplikacji atrybutu const do parametrów generyka, np list< [const] usize>

\subsection{Modyfikowanie kodu}

Jednym z celów tej pracy było zbadanie, jak umożliwienie modyfikowania kodu w trakcie kompilacji, wpływa na tworzone oprogramowanie.
Niestety ze względu na błędną wycenę wielkości tego projektu, ten aspekt został pominięty.

Większość potrzebnych prac została już wykonana.
Model semantyczny programu jest trzymany w strukturach danych, które można modyfikować z interpretowanego kodu.
Proces kompilacji został zaprojektowany z uwzględnieniem zmian w reprezentacji pośredniej.

% todo: explain basic library
Aby skutecznie skorzystać z tej możliwości, należy jeszcze rozszerzyć bibliotekę podstawową języka.
Obecnie programista może zamieniać albo usuwać istniejące elementy modelu semantycznego.
Aby umożliwić użytkownikom pełne wykorzystanie potencjału C-=-1, do biblioteki podstawowej należy dodać możliwość dynamicznego tworzenia elementów modelu.
W przyszłosci, do języka trzeba będzie dodać funkcje do tworzenia nowych funkcji, instrukcji oraz wyrażeń.

\subsection{Ewaluacja stałych wyrażeń w trakcie kompilacji}
\label{compile_time_constant_evaluation}
Jednym z celów C-=-1 było umożliwienie ewaluacji dowolnie złożonego wyrażenia do stałej oraz wykonywania obliczeń i generowania dodaktowych plików w trakcie kompilacji.
Z tego powodu, większość funkcji można wykonać zarówno w czasie kompilacji jak i uruchomienia.

Wśród nich są też procedury, mające efekty uboczne, na przykład interakcja z systemem plików czy nawet modyfikacja zmiennej globalnej.
Programista musi wiedzieć w którym etapie życia programu, kompilacji czy uruchomienia, te funkcje zostaną wykonane.
Jednak Z punktu widzenia modelu semantycznego one w żaden sposób się nie wyróżniają.

Ewaluacja wyrażeń, zawierających taki kod, w czasie kompilacji, prowadziłaby do błednego zachowania programu.
W czasie kompilacji, dochodziłoby do zbędnych operacji na plikach, bądź nawet błędów kompilacji.
Natomiast w czasie uruchomienia, program pomijałby pewne operacje, operując na przykład na plikach dostępnych w czasie kompilowania.

Aby poprawnie zrealizować tą funkcjonalność, koniecznym jest wprowadzenie do języka wyróżnienia funkcji które wpływają na, bądź zależą od globalnego stanu.
W tym celu można zastosować koncepcję 'czystej' funkcji z funkcyjnych języków programowania \cite{McLoughlin1989ImperativeEF}.
Dzięki temu, kompilator będzie w stanie ustalić, które wyrażenia może ewaluować jako część optymalizacji a które muszą pozostać w skompilowanym programie.

Takie rozwiązanie omija problem powstaly z wprowadzeniem słowa kluczowego \lstinline{constexpr} w C++\cite{Klimiankou:contexpr_great_good_wrong_idea}.
Czystość funkcji niesie za sobą dużo więcej konsekwencji niż to że funkcja może być częścią procesu optymalizacji.
Oznacza to że programista nie musi być świadomy transformacji które wykonuje kompilator.

\subsection{Integracja ze środowiskiem programistycznym}
\label{IDE_integration}

\subsection{System typów a atrybuty}
\label{further:adnotated_type_system}

wychodzi na to że koncepcja adnotacji jest bardzo mocno powiązana z systemem typów. Może trzeba przemyśleć w ogóle system typów języka. Odnośnik do teorii typów/kategorii
\subsubsection{Równoważność atrybutów przy wybieraniu przeciążenia}
\label{further:adnotated_type_system:attribute_equivalence}
