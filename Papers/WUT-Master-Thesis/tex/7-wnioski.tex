\section{Wnioski}

C-=-1, z oczywistych powodów, pod wieloma wzgęldami jest gorszy od istniejących rozwiązań.
Wiele udogodnień, traktowanych już jako standard, takich jak dedukcja parametrów generycznych czy semantyka przenoszenia nie zostało zaimplementowanych z powodu braku czasu.
Ponadto wsparcie dla C-=-1 jest minimalne: nie istnieje edytor, który koloruje jego składnie, a kompilator oferuje w najlepszym wypadku bardzo podstawową diagnostykę.
Jego celem nie było jednak zastąpić C++, Rust czy C\#, a zbadać jak priorytetyzacja wsparcia dla statycznego metaprogramowania wpływa na język oraz pisany w nim kod. 
Pod tym względem, C-=-1 można uznać za sukces.



\subsection{Kategoryzacja języka}

C-=-1 od samego początku był definiowany jako kompilowany język programowania ze wsparciem dla statycznego metaprogramowania i wykonywania kodu w czasie kompilacji.
Oryginalny projekt implementacji zakładał więc, że kompilator będzie zawierał w sobie moduł interpretujacy reprezentację pośrednią programu.
Po wykonaniu transformacji na kodzie użytkownika, kompilator miał przejąć kontrolę i dokonać kompilacji do kodu maszynowego.

Jednak w wyniku zmian opisanych w rozdziale \ref{Backend_Interface}, kompilacja do kodu maszynowego została zaimplementowana w interpretowanym C-=-1.
Oznacza to, że kompilator składa się wyłącznie z frontendu oraz środowiska uruchomieniowego które, udostępnia intefejs biblioteki llvm.
Z tego powodu, w ogólnym przypadku, C-=-1 można zaklasyfikować jako silnie typowany język interpretowany.

Należy jednak pamiętać, że ten język zawiera w sobie rozróżnienie między kodem wykonywalnym w czasie uruchomienia a kompilacji.
Ponadto, jego projekt zawiera semantykę typową dla języków kompilowanych: jawną alokację obiektów na stercie, ręczne zarządzanie pamięcią oraz wskazania. 
Ten aspekt został opisany w rozdziale \ref{Language_desig}.
Tak więc najlepszym sposobem na zkategoryzowanie C-=-1 jest uznanie go za hybrydę pomiędzy językiem kompilowanym a interpretowanym.


\subsection{Rozszerzalność języka}

Umożliwienie programiście modyfikowanie kodu oraz tworzenie własnych adnotacji sprawia, że język można bardzo łatwo rozszerzyć o nowe mechanizmy.
W rozdziale \ref{Language_extensibility} opisano przykłady składni nie wspieranej przez C-=-1, a którą można dodać przy użyciu proponowanego mechanizmu atrybutów.

Wykorzystując go, programista może zawęzić zbiór programów akceptowanych przez kompilator, przy użyciu analizy statycznej.
W rozdziałach \ref{no_discard} oraz \ref{const} przedstawiono przykłady adnotacji istniejących w C++ jako część języka, które można dodać do C-=-1.
Po dodaniu mechanizmu wyjątków, trywialną będzie też implementacja atrybutu \lstinline{noexcept}, oznaczającego funkcje która nigdy nie rzuca.

\subsection{Możliwa krytyka}

Zaproponowane podejście niesie ze sobą pewne wady.
Być możne najpoważniejszą z nich jest utrudniony rozwój kompilatora.
Ponieważ model programu musi edytowalny przez interpretowany kod, należy go zbudować ze struktur danych interpretera.
Dokładne uzasadnienie i opis implementacji znajduje się w rozdziale \ref{backend_integration}.
Opisane tam wyzwania stanowią argument przeciwko konstruowaniu języków programowania w ten sposób. 
Chwilowo nie ma alternatywnej implementacji, która zachowuje silne typowanie w kodzie kompilatora.

Inny problem, dotyczy samej konstrukcji języka.
Zaproponowane w tej pracy mechanizmy metaprogramistyczne, mogą prowadzić do zwiększenia poziomu złożoności tworzonego kodu.
Ponieważ atrybuty w C-=-1 mogą nadpisywać kod użytkownika, nadużycie tego mechanizmu może prowadzić do trudno przewidywalnego zachowania.
Zasadność tej krytyki powinna być tematem przyszłych badań.

%todo: opisać problem z systemem typów ad-hoc
