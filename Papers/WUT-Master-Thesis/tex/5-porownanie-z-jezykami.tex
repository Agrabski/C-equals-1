\section{Porównanie z innymi językami}
Ponieważ C-=-1 jest językiem badawczym, nie można go porównywać z językami stosowanymi w przemyśle pod względem wygody użycia. Warto jednak przeanalizować, w jaki sposób zaproponowane mechanizmy wpływają na jego użyteczność.
\subsection{Wsparcie dla paradygmatów programowania}
Teoretycznie niemalże w każdym języku da się programować w każdym paradygmacie. Jednak w zależności od struktury tego języka, może być to zadanie prostsze lub trudniejsze. 
\subsubsection{Aspect oriented programming}
Nieoczekiwanym skutkiem zaproponowanych mechanizmów jest wsparcie dla AOP.
\subsubsection{Obiektowe}
C-=-1 został zaprojektowany z myślą o wsparciu stylu obiektowego. Użytkownik może deklarować typy, metody oraz interfejsy. W przeciwieństwie do C++ czy C\#, w C-=-1 nie ma jednak koncepcji dziedziczenia między konkretnymi klasami.

Klasy i interfejsy mogą implementować inne interfejsy. To jest jedyny mechanizm dynamicznego polimorfizmu w C-=-1. Ta decyzja została podjęta częściowo w celu uproszczenia języka, a częściowo ponieważ dziedziczenie między konkretnymi klasami jest uważane za złą praktykę.

\subsection{Walidacja kodu}
Podstawowym celem C-=-1 było zbadanie możliwości statycznej walidacji kodu, używając zaproponowanych mechanizmów meta-programowania. W kolejnych rozdziałach omówione zostaną ich konkretne zastosowania 
\subsubsection{Atrybut noDiscard}
\label{no_discard}
\subsubsection{Atrybut const}
\label{const}

\subsection{Generowanie kodu}

\subsubsection{Atrybut Flags}

\subsection{Testowanie kodu}

\subsection{Rozszerzalność języka}
\label{Language_extensibility}
Jedną z konsekwencji zaproponowanych mechanizmów jest możliwość rozszerzania języka bez modyfikacji kompilatora.
Oznacza to że pewne elementy składni, obecne w innych językach, stają się zbędnę w C-=-1.
\subsubsection{Symbole zewnętrzne}
Większość języków programowania zawiera mechanizm umożliwiający import symboli zewnętrznych.
Mogą one pochodzić z kodu napisanego w innym języku lub z już skompilowanej biblioteki.

W wypadku C/C++, wymaga to użycia słowa \lstinline{external}, do zdefiniowania symbolu zewnętrznego oraz przekazania odpowiedniego argumentu do linkera.
W C-=-1 ten sam rezultat można osiągnąć, przypisując funkcji odpowiednie metadane, które potem zostaną odczytane przez interfejs backendu.
Wbudowana wersja tego programu obsługuje wczytywanie definicji funkcji z plików lib oraz zastępowanie jej ciała kodem LLVM.
Mechanizm metadanych został dokładniej opisany w rozdziale \ref{final_solution}.

Listing \ref{lst:replaceWithLLVMAttribute} przedstawia atrybut \lstinline{replaceWithLLVMIR} z biblioteki standardowej C-=-1. 
Pobiera on kod LLVMIR z pliku wskazanego przez użytkownika, a następnie ustawia w definicji wybranej funkcji zmienną "llvm-representation".
Backend interface może potem odczytać tą wartość i wykorzystać ją do wygenerowania kodu dla backendu.


\begin{lstlisting}[
    numbers=left,
    firstnumber=0,
    caption={Atrybut zastępujący ciało funkcji kodem LLVM},
    aboveskip=0pt,
    label={lst:replaceWithLLVMAttribute}
]
public att<function> ReplaceWithLLVMIR
{
  private _filename: string;
  private _nameOftypeToReplace: string;
  private _nameToReplace: string;

  public fn construct(
    filename: string,
    nameOfTypeToReplace: string,
    nameToReplace: string)
  {
    self._filename = filename;
    self._nameOftypeToReplace = nameOfTypeToReplace;
    self._nameToReplace = nameToReplace
  }

  public fn attach(f: functionDescriptor)
  {
    let ir = read_all_file(self._filename);
    ir = ir.replace("$" + self._nameOftypeToReplace, self._nameToReplace);
    f.store("llvm-representation", ir);
  }
}

\end{lstlisting}
