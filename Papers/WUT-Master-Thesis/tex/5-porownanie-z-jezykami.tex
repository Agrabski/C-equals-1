\section{Porównanie z innymi językami}
\label{comparison}

Ponieważ C-=-1 jest językiem badawczym, nie można go porównywać z językami stosowanymi w przemyśle pod względem wygody użycia.
Warto jednak przeanalizować, w jaki sposób zaproponowane mechanizmy wpływają na jego użyteczność.

Rozdział \ref{programming_paradigms} zawiera opis użycia języka C-=-1 w różnych paradygmatach programowania i jak zaproponowane mechanizmy mogą w nich pomóc.
Charakterystyka C-=-1 sprawia, że można go zastosować z nietypowymi technikami programistycznymi
W rozdziale \ref{code_analysis} przedstawiono natomiast przykłady użycia atrybutów do statycznej analizy kodu.
Ten mechanizm umożliwia weryfikację poprawności programu, bez użycia dodatkowych narzędzi.
Rozdział \ref{code_generation} zawiera przykłady zastosowania modyfikacji programu na etapie kompilacji.
Rozdziały \ref{comparison:template_type_erasure}, \ref{Language_extensibility} oraz \ref{comparison:exporting_to_other_languages} zawierają natomiast dyskusję o interakcji C-=-1 z innymi językami programowania: jak importować symbole zewnętrzne oraz jak wygląda wynik kompilacji.

\subsection{Wsparcie dla paradygmatów programowania}
\label{programming_paradigms}
Teoretycznie niemalże w każdym języku da się programować w każdym paradygmacie. Jednak w zależności od jego, może być to zadanie prostsze lub trudniejsze.
W niektórych wypadkach wymaga to wręcz użycia zewnętrznego narzędzia, poza kompilatorem \cite{aop:cpp}.
W tym rozdziale zostaną przedstawione paradygmaty programowania wspierane w C-=-1, bez dodatkowych narzędzi i ingerencji w język.

\subsubsection{Aspect Oriented Programming}
Nieoczekiwanym skutkiem zaproponowanych mechanizmów jest wsparcie dla programowania aspektowego (Aspect Oriented Programming, AOP) \cite{aop}.
Ta możliwość wynika z możliwości pisania samomodyfikującego kodu.

Programowanie aspektowe rozszerza typowy model programowania obiektowego o aspekty.
Składniowo, są podobne do klas.
Poza metodami i polami aspekty mogą deklarować również rady, służące do modyfikowania kodu w tak zwanych punktach cięcia.
Rada może wykonać dowolne operacje przed, po oraz zamiast wywołania funkcji zdefiniowanej przez punkt cięcia.

C-=-1 teoretycznie umożliwia tworzenie oprogramowania w paradygmacie AOP.
Jedyna przeszkoda przed praktycznym użyciem programowania aspektowego to niezaimplementowana biblioteka do generacji kodu.
Listing \ref{lst:aop_in_cm1} przedstawia przykład zastosowania AOP w C-=-1, używając hipotetycznych funkcji do generacji kodu.
Atrybut \lstinline{LogCalls} ma za zadanie wypisać do konsoli informację o każdym wywołaniu przyłączonej funkcji.
Metoda \lstinline{onCall} jest wywoływana przy każdym użyciu adnotowanej procedury.
Wewnątrz niej, atrybut, pełniący funkcję rady z AOP, może podjąć decyzję czy modyfikować to konkretne użycie.
Zmiana programu, w tym przykładzie, odbywa się przez dodanie dodatkowej instrukcji wywołania, odnoszącej się do funkcji \lstinline{print}.

%todo: make example
%todo: describe example
\begin{minipage}{\linewidth}
  
  \begin{lstlisting}[
    numbers=left,
    firstnumber=0,
    caption={Przykład Programowania Aspektowego w C-=-1},
    aboveskip=0pt,
    label={lst:aop_in_cm1}
    ]
public att<function> LogCalls
{
  public fn onCall(f: functionCallExpression)
  {
    let statement = f.parentStatment();
    let function = statement.definingFunction();
    let args = <IExpression>[];
    args.push(new unique constantExpression(f.function.name()
      + " was called"));
    function.insertBefore(
      statement,
      new unique functionCallStatement(
        new unique functionCallExpression(
          functionof(print),
          args
        )
      ));
  }
}
\end{lstlisting}
\end{minipage}

Kod przedstawiony w listingu \ref{lst:aop_in_cm1} używa planowanych dodatków do języka C-=-1 oraz jego biblioteki standardowej i stanowi bardzo podstawowy przykład zastosowania AOP.
Atrybut jest w nim używany jako rada.
Model punktu przyłączenia (ang: join-point model) \cite{vidal2019jpiaspectz}, użyty w tym przykładzie, umożliwia obserwowanie jedynie jawnie oznaczonych funkcji i nie zapewnia filtrowania parametrów, ani kontekstu wywołania.
Na jego podstawie można jednak zbudować bardziej zaawansowaną bibliotekę.

W przyszłej wersji języka zostaną wprowadzone atrybuty aplikowane do całego pakietu.
Ich wprowadzenie umożliwi stworzenie join-point model, który jest bliższy tego z biblioteki AspectJ \cite{keselev2003aspect}.


\subsubsection{Programowanie Obiektowe}
C-=-1 został zaprojektowany z myślą o wsparciu stylu obiektowego. Użytkownik może deklarować typy, metody oraz interfejsy. W przeciwieństwie do C++ czy C\#, w C-=-1 nie ma jednak koncepcji dziedziczenia między konkretnymi klasami.

Klasy i interfejsy mogą implementować inne interfejsy.
To jest jedyny mechanizm dynamicznego polimorfizmu w C-=-1.
Ta decyzja została podjęta częściowo w celu uproszczenia języka, a częściowo, ponieważ dziedziczenie między konkretnymi klasami jest uważane za złą praktykę \cite{gang_of_four:design_patterns}.

\subsection{Analiza kodu}
\label{code_analysis}
Podstawowym celem C-=-1 było zbadanie możliwości statycznej analizy kodu, używając zaproponowanych mechanizmów metaprogramowania.
W kolejnych rozdziałach omówione zostaną przykładowe analizatory, które zostały zaimplementowane w ramach biblioteki standardowej bądź których implementacja jest możliwa.
Ma to zademonstrować praktyczne aplikacje udostępnienia programiście modelu semantycznego tworzonego programu.



\subsubsection{Atrybut noDiscard}
\label{no_discard}

Atrybut \lstinline{noDiscard} służy do zapewnienia, że wynik funkcji nie zostanie zignorowany.
Tego typu adnotacja istnieje w C++ od wersji 17\cite{ISO:cpp17}.
Ma on na celu wyeliminować błędy takie jak ignorowanie kodu błędu, czy niepoprawne wywołanie funkcji o mylącej nazwie.

W C-=-1, w przeciwieństwie do większości języków niskiego poziomu, stworzenie analizatora, który zapewnia taki niezmiennik, jest trywialne.
Listing \ref{lst:noDiscardCm1} zawiera kod atrybutu \lstinline{noDiscard} z biblioteki standardowej C-=-1.
Linia czwarta deklaruje specjalną funkcję \lstinline{onCall} (specjalne funkcje atrybutów zostały opisane w rozdziale \ref{Attributes_mechanism_cm1}), która reaguje na wywołanie funkcji, do której został zaaplikowany.

Wewnątrz tego podprogramu, atrybut sprawdza, czy bezpośrednim rodzicem tego wyrażenia jest wyrażenie, czy instrukcja.
Jeśli jest nim instrukcja, oznacza to że wynik wywołania jest ignorowany i trzeba zgłosić błąd, używając funkcji \lstinline{raiseError}.
W przeciwnym wypadku żadne akcje nie są konieczne.

Na Listingu \ref{lst:noDiscardUsageCm1} przedstawiono zastosowanie atrybut \lstinline{noDiscard}.
Zignorowanie wartości zwracanej przez funkcję oznaczoną tą adnotacją, tak jak w linii piątej, powoduje zgłoszenie błędu o kodzie i komunikacie zgodnym z linią siódmą listingu \ref{lst:noDiscardCm1}.
Ponadto, kompilator otrzymuje wskazanie do pliku źródłowego na punkt, który wywołał ten błąd.
Ta informacja może być potem użyta do przekazania programiście dokładniejszej diagnostyki bądź do podkreślenia kodu w zintegrowanym środowisku programistycznym (rozdział \ref{IDE_integration}).

% reference listing
Przykład, zawarty w listingu \ref{lst:noDiscardCm1}, dobrze ilustruje, że mając dostęp do modelu semantycznego programu, implementacja niektórych typów analizy staje się trywialna.
W większości języków programowania taka analiza wymaga modyfikacji kompilatora albo zewnętrznego narzędzia.

\begin{minipage}{\linewidth}
  
  \begin{lstlisting}[
    numbers=left,
    firstnumber=0,
    caption={Atrybut noDiscard w C-=-1},
    aboveskip=0pt,
    label={lst:noDiscardCm1}
    ]
public att<function> NoDiscard
{
  public fn attach(f: functionDescriptor)
  {}
  public fn onCall(call: functionCallExpression*)
  {
  if(call._parentStatment != null<IInstruction>())
    raiseError(
      &(call._pointerToSource), 
      "Return value of a no-discard function is not used",
      123
    );
  }
}
\end{lstlisting}
\end{minipage}


\begin{minipage}{\linewidth}
  
  \begin{lstlisting}[
    numbers=left,
    firstnumber=0,
    caption={Przykład użycia atrybut noDiscard w C-=-1},
    aboveskip=0pt,
    label={lst:noDiscardUsageCm1}
    ]
[noDiscard()]
fn noDiscardFunction() -> usize;

fn main() -> usize
{
  noDiscardFunction(); // error 123: Return value of
                       // a no-discard function is not use
  let x = noDiscardFunction();     // ok
  let y = x + noDiscardFunction(); // ok
  return noDiscardFunction();      // ok
}
\end{lstlisting}
\end{minipage}

\subsubsection{Atrybut const}
\label{const}
Modyfikator \lstinline{const} z języka C/C++ oznacza wartość jako stałą i jest nieodłączną częścią systemu typów.
Przy wyborze przeciążenia funkcji, wartości typu \lstinline{int *} oraz \lstinline{int const *} są traktowane jako dwa różne typy.
Napisanie atrybutu, zapewniającego podobne zachowanie, do C-=-1 jest ciekawą aplikacją nowych mechanizmów tego języka.
W obecnym stanie kompilatora oraz języka, możliwa jest implementacja tylko części tej analizy.
Zaimplementowanie modyfikatora \lstinline{const} w pełni, przy użyciu atrybutu, wymaga od C-=-1 następujących, dodatkowych, właściwości:
\begin{enumerate}
  \item \label{prop:Attribute_function_overloading} Atrybuty są brane pod uwagę w trakcie wyboru przeciążenia funkcji
  \item \label{prop:Generic_adnotations} Istnieje sposób na modyfikowanie zachowania zmiennych, pól i parametrów w generykach, przy tworzeniu ich instancji
  \item \label{prop:Reference_adnotations} Przy aplikowaniu atrybutu do zmiennej typu referencyjnego, istnieje sposób na rozróżnienie między stałym wskazaniem a wskazaniem na stałą.
\end{enumerate}

Właściwość \ref{prop:Attribute_function_overloading} jest częściowo zaplanowaną, ale niezrealizowaną częścią języka (rozdział \ref{Attributes_definition}).
Atrybuty powiązane z parametrami funkcji mają być traktowane jako wymagania.
Wartość użyta jako argument musi zawierać wszystkie adnotacje parametru, któremu odpowiada.
Przykładowo, deklaracja \lstinline{fn positive_parameter_fun([positive()] usize paramter)} wymusza na wołającym funckję \lstinline{positive_parameter_fun}, aby przekazywana jej wartość była adnotowana atrybutem \lstinline{positive}.
Otwartym pytaniem pozostaje też, w jaki sposób traktować atrybuty przyjmujące parametr w kontekście wyboru przeciążenia funkcji.
Ten problem został dokładniej opisany w rozdziale \ref{further:adnotated_type_system:attribute_equivalence}
Zapewnienie tej właściwości wymaga ponadto dalszego rozszerzenia systemu atrybutów, tak aby można było aplikować adnotacje do wartości zwracanych z funkcji.
Ta potrzeba wywodzi się z tego, że dokładne znaczenie danej operacji w jest zależne od modyfikatorów zaaplikowanych do używanych wartości.
Listing \ref{lst:cpp_const_example} zawiera przykład użycia modyfikatora \lstinline{const} w C++.
Przeciążenie metody \lstinline{member} jest wybierane, biorąc pod uwagę czy parametr \lstinline{this} jest \lstinline{const}.
Tak więc w linii dziewiątej, wybierane jest przeciążenie nie-\lstinline{const} z linii trzeciej a w linii dziesiątej, przeciążenie \lstinline{const} z linii czwartej.

\begin{minipage}{\linewidth}
  \begin{lstlisting}[
    numbers=left,
    firstnumber=0,
    caption={Przykład modyfikatora const w C++},
    aboveskip=0pt,
    label={lst:cpp_const_example}
    ]
  class Foo {
    int member_;
  public:
    int& member();
    int const& member() const;
  };
  int main() {
    Foo foo;
    Foo const constFoo;
    foo.member() = 3;     // ok
    constFoo.member() = 3;// error
  }
  \end{lstlisting}
\end{minipage}

Zapewnienie właściwości \ref{prop:Generic_adnotations} jest najtrudniejsze i wymaga dalej idącej pracy teoretycznej.
Zaaplikowanie systemu atrybutów do implementacji modyfikatora \lstinline{const} oznacza śledzenie wartości, nie koniecznie pól, zmiennych, typów czy funkcji.
Listing \ref{lst:const_problem_example} zawiera przykład problemu rozwiązywanego przez właściwość \ref{prop:Generic_adnotations}.
W linii drugiej deklarowana jest zmienna typu \lstinline{usize} z modyfikatorem \lstinline{const}.
Następnie, w linii czwartej, wskazanie na nią jest wstawiane do listy.
W tym miejscu analizator wygeneruje błąd kompilacji, ponieważ tworzymy nie-stałe wskazanie na stałą wartość, co może naruszyć chroniony niezmiennik.
Rozwiązaniem byłoby zadeklarowanie listy jako zawierającej wskazania na stałe.
W obecnej formie języka nie jest to jednak możliwe.

\begin{minipage}{\linewidth}
  
  \begin{lstlisting}[
    numbers=left,
    firstnumber=0,
    caption={Śledzenie wartości, używając atrybutu const},
    aboveskip=0pt,
    label={lst:const_problem_example}
    ]
fn main() -> usize
{
  [const()] let x = 3;
  let list = List<usize*>();
  list.push(&x);
  [const()] let pointer = &x;
}
\end{lstlisting}
\end{minipage}


Właściwość \ref{prop:Reference_adnotations} jest blisko powiązana z właściowścią \ref{prop:Generic_adnotations}.
Rozróżnienie pomiędzy aplikacją atrybutu do wskazania a wskazywanej wartości, jest konieczne, aby zapewnić nienaruszenie niezmienników dotyczących wartości.
Listing \ref{lst:const_problem_example} przedstawia ten problem.
W linii piątej, do zmiennej \lstinline{pointer} przypisywany jest adres zmiennej \lstinline{x}.
Pomimo zaaplikowania atrybutu \lstinline{const} do nowej zmiennej, analizator generuje błąd.
Adnotacja została tutaj zaaplikowana do wartości wskazania, nie do wskazywanej wartości.
W C-=-1 istnieją sposoby na obejście tego ograniczenia, jednak lepsze rozwiązanie powinno powstać w wyniku prac opisanych w rozdziale \ref{further:adnotated_type_system}.

Wszystkie te właściwości wskazują na silne powiązanie pomiędzy systemem adnotacji a typów.
Sposób na połączenie tych aspektów C-=-1 powinno zostać bardziej dogłębnie zbadane w przyszłej pracy.
Szczegóły tych badań zostały opisane dokładniej w rozdziale \ref{further:adnotated_type_system}.
W C/C++ modyfikator \lstinline{const} jest integralną częścią systemu typów.
Typ \lstinline{int*} jest różny od \lstinline{int const*} i konwersja między nimi istnieje tylko w jedną stronę.
Z tego powodu, wszystkie problemy wymienione w tym rozdziale nie mają odzwierciedlenia w C/C++.

\subsection{Generowanie kodu}
\label{code_generation}

Model semantyczny programu, w języku C-=-1, jest mutowalną strukturą danych.
Oznacza to, że kod atrybutów może ją modyfikować w trakcie kompilacji, dodając, usuwając bądź zmieniając kolejność instrukcji i wyrażeń.
Ten aspekt języka daje programiście szereg nowych możliwości.

\subsubsection{Atrybut Flags}

Jednym z najprostszych przykładów użycia możliwości generowania kodu, jest zadeklarowanie typu enumeracyjnego jako zbioru flag.
Aby to osiągnąć, wszystkie wartości enumeratora muszą być kolejnymi potęgami liczby 2, tak aby binarna alternatywa dowolnych dwóch opcji była unikatowo identyfikowalna.

W większości języków programowania nie ma możliwości automatycznego przypisywania wartości enumeratora w ten sposób.
W C++ nie istnieje też żaden analizator który wykrywałby błędną definicję takiej flagi.
Przydzielenie wartości oraz weryfikacja czy są one poprawne, jest opdowiedzialnością programisty.

W C\# istnieje atrybut \lstinline{Flags}.
Służy on do sprawdzania poprawności wartości enumeratora i generuje ostrzeżenia w wypadku wykrycia błędu, ale nie przypisuje ich sam.
Programista nadal musi samodzielnie je ustalić.
W C-=-1, w obecnej wersji, nie istnieje pojęcie typu wyliczeniowego.
Ich dodanie jest jednak w planach.

\begin{minipage}{\textwidth}
\begin{lstlisting}[
  numbers=left,
  firstnumber=0,
  caption={Hipotetyczna implementacja atrybutu \lstinline{flags} dla C-=-1},
  aboveskip=0pt,
  label={lst:enum_flags_attribute}
  ]
public att<enumValue> NoneFlag {}
public att<enum> Flags {
    public fn attach(e: enumDescriptor) {
      let index = 0;
      for(label in e.labels)
        if(label.hasAttribute<_att_ NoneFlag>())
          label.value = 0;
        else
          label.value = power(2, index++);
    }
}
[Flags()] enum Sample {
  [NoneFlag()] None,
  A,
  B,
  C = A | B
}
\end{lstlisting}
\end{minipage}

Listing \ref{lst:enum_flags_attribute} zawiera hipotetyczną implementację atrybutu \lstinline{flags}, automatycznie przypisującego poprawne wartości elementom typu wyliczeniowego.
Każdemu z nich odpowiada kolejna potęga dwójki, poza oznaczonymi adnotacją \lstinline{NoneFlag}: te mają wartość zero.
Ta implementacja ma za zadanie zademonstrowanie możliwości generowania kodu, bardziej praktyczna wersja powinna brać pod uwagę też możliwość ręcznego ustalania wartości elementów na kombinacje innych etykiet.
Przykład takiej sytuacji jest w linii osiemnastej listingu \ref{lst:enum_flags_attribute}.

Dodanie typów wyliczeniowych do języka wymaga też rozszerzenia modelu semantycznego.
Kod z listingu \ref{lst:enum_flags_attribute} korzysta z części tych dodatków.
Cele dla atrybutów (opisane w rozdziale \ref{Attributes_definition}) \lstinline{enum} i \lstinline{enumValue} są powiązane odpowiednio z typem wyliczeniowym oraz jego pojedynczą wartością.
Ponadto, należy dodać typy \lstinline{enumDescriptor} oraz \lstinline{enumValueDescriptor}.

\subsubsection{Testowanie kodu}

Dzięki możliwościom generowania kodu w C-=-1, można w nim łatwo skonstruować framework do pisania testów jednostkowych.
Przykładami takiego oprogramowania, istniejącego dla innych języków programowania są biblioteki Nunit dla C\#, Gtest dla C++ czy Jasmine dla javascript.
Testy jednostkowe są typowo pisane jako zbiór funkcji, z których każda przedstawia pewne założenie, które testowane oprogramowanie musi spełniać.

\begin{minipage}{\textwidth}
  \begin{lstlisting}[
    numbers=left,
    firstnumber=0,
    caption={Przykład testów jednostkowych w C-=-1},
    aboveskip=0pt,
    label={lst:cm1_tests}
    ]
[Test()]
fn test1() {
  assert_equal(1 + 1, 2)
}
[Test(1, 1, 2)]
[Test(2, 2, 4)]
fn test2(arg1: usize, arg2: usize, result: usize) {
  assert_equal(arg1 + arg2, result);
}

[TestMain()]
fn main() {
}
  \end{lstlisting}
\end{minipage}

Listing \ref{lst:cm1_tests} zawiera przykładowe użycie frameworku do testów jednostkowych, który można skonstruować w C-=-1.
Atrybut \lstinline{Test} oznacza funkcję jako test jednostkowy.
Jego konstruktor może przyjmować szeroką gamę list argumentów.
Wartości przekazywane przy konstruowaniu atrybutu, będą wykorzystane przy wywołaniu funkcji testowej.
Atrybut \lstinline{TestMain}, przyłączony do procedury stanowiącej punkt wejścia programu, zamieni jej ciało na sekwencje wywołań wszystkich metod testowych.


\subsection{Kompilacja szablonów}
\label{comparison:template_type_erasure}
Jedną z ciekawych możliwości, wynikającej z istnienia modelu semantycznego w C-=-1, zawierającego w sobie funkcje generyczne, jest ich bardziej efektywna kompilacja.
W C++ szablony są częstym źródłem rozrostu generowanego kodu maszynowego.
Listing \ref{lst:comparison:bloat_cpp_example:cpp} zawiera przykład programu, w którym występuje ten problem natomiast w listingu \ref{lst:comparison:bloat_cpp_example:assembly} znajduje się kod assemblera x86-64 generowany przez kompilator Clang \cite{lattner2008llvm}.
Przykład został przygotowany przy użyciu narzędzia Compiler Explorer \cite{godbolt}.

Szablon zdefiniowany w linii piątej, listingu \ref{lst:comparison:bloat_cpp_example:cpp} jest prostym przykładem generyka, którego kod jest całkowicie niezależny od typów użytych do stworzenia jego instancji.
Funkcja \lstinline{foo} przyjmuje parametr \lstinline{v} będący wskazaniem na typ \lstinline{T} i używa go wyłącznie po konwersji do liczby całkowitej lub odejmując od niego rozmiar \lstinline{T}.
Szablon procedury \lstinline{foo} można zaprezentować matematycznie, jako funkcję mapującą typ \(t\) na funkcję \(x -> size\_t\) czyli \( foo = t -> (x: t -> size\_t)\). Alternatywnie: \(foo : T_y -> F\), gdzie \(T_y\) to zbiór wszystkich typów, a \(F\) to zbiór wszystkich funkcji jednoargumentowych zwracających typ \(size\_t\).
Każdą z instancji \lstinline{foo}, da się natomiast sprowadzić do formy:\\
\(foo(t) = y(x: size\_t) -> size\_t = \begin{cases}
  bar(x), & \text{jeśli} x < 500 \\
  y(x - sizeof(t)) & \text{w przeciwnym wypadku}
\end{cases}\)\\
Tak więc dla danego zbioru typów \(A = \{t_1, t_2, ... t_n\}\), przeciwdziedzina funkcji \(foo\) : \(B = \{ f | f = foo(t) \land t \in A\}\) będzie miała rozmiar \(|B| = |\{y | y = sizeof(t) \land t \in A\}|\).
W wypadku kodu z listingu \ref{lst:comparison:bloat_cpp_example:cpp}, obydwie instancje szablonu \lstinline{foo} powstały na podstawie typów o tych samych rozmiarach, więc powinna istnieć tylko jedna wersja tej funkcji.

Listing \ref{lst:comparison:bloat_cpp_example:assembly} zawiera kod w assemblerze x86-64 wygenerowany przez narzędzie Clang dla źródła z listingu \ref{lst:comparison:bloat_cpp_example:cpp}.
W liniach zerowej oraz dziesiątej znajdują się etykiety powiązane z instancjami generyka \lstinline{foo}.
Pomimo tego, że kod obydwu procedur jest identyczny, zgodnie z rozważaniami z poprzedniego akapitu, kompilator wygenerował obydwie wersje.
To jest przykład rozrostu kodu, któremu można zapobiec w C-=-1.

Aby zredukować rozrost kodu, zademonstrowany na przykładzie z listingu \ref{lst:comparison:bloat_cpp_example:cpp}, możemy zastosować tak zwane kasowanie typów.
Jest to technika używana przez programistów do optymalizacji kodu C++ \cite{becker2007:type_erasure} oraz przez kompilator Javy \cite{nino2007:cost_type_erasure_java}.
Te dwa języki tworzą więc spektrum, pomiędzy językami, w których instancje generyków są niezależnymi od siebie bytami a tymi, w których generyki stanowią jedynie ''cukier składniowy''.
C\# oraz inne języki używające Common Language Runtime \cite{ecma:cli} jako środowisko uruchomieniowe, znajduje się pośrodku tej skali.
Generyki tego języka są bezpośrednio reprezentowane w Common Intermidiate Language.
W trakcie JIT, środowisko uruchomieniowe dynamicznie podejmuje decyzję o emisji nowej instancji generyka bądź użyciu już istniejącej przez kasowanie typów \cite{kennedy2001:design_clr_generics}.
W ten sposób C\# omija problemy wynikające z jawnego kasowania typów w Javie oraz rozrostu kodu z C++.
To rozwiązanie, w przeciwieństwie do C++ wymaga oczywiście środowiska uruchomieniowego.

\begin{minipage}{\textwidth}
  
  \begin{lstlisting}[
    numbers=left,
    firstnumber=0,
    caption={Przykładowy kod w C++ powodujący rozrost pliku wynikowego},
    aboveskip=0pt,
    label={lst:comparison:bloat_cpp_example:cpp}
    ]
  #include<vector>
  #include<iostream>
    
  extern size_t bar(size_t v);
  template<typename T>
  size_t foo(T const* v) {
      if((size_t)v > 500)
          return bar(foo(v - 1));
      return  bar((size_t)v);
  }
  int main() {
      int n = 3;
      float y = 5;
      std::cout<< foo(&n) + foo(&y);
  }
  \end{lstlisting}
\end{minipage}

Ponieważ model semantyczny, na którym operuje \emph{interfejs backendu} C-=-1, umożliwia identyfikację funkcji jako instancję generyka, można wykonać powiązane z tym optymalizację.
Nawet w obecnej, okrojonej wersji kompilatora, możliwe jest porównanie instancji funkcji generycznej i stwierdzenie czy istnieją między nimi różnice.
Dzięki temu dostatecznie zaawansowana implementacja \emph{interfejsu backendu} jest w stanie dokonywać kasowania typu, w sposób całkowicie ukryty przed programistą.
W ten sposób zachowana zostaje moc szablonów C++, unikając jednocześnie problemów Javy.
Ta optymalizacja nie została zaimplementowana z powodu jej dużej złożoności technicznej w porównaniu do wartości badawczej w kontekście tej pracy.
Jej wykonanie zostałoby znacząco ułatwione po wprowadzeniu zmian w modelu semantycznym, opisanym w rozdziale \ref{further:generic_intermidiate_representation}.

\begin{minipage}{\textwidth}
  
  \begin{lstlisting}[
    numbers=left,
    firstnumber=0,
    caption={Fragment assemblera x86-64 generowanego dla przykładu z listingu \ref{lst:comparison:bloat_cpp_example:cpp} przez kompilator Clang w wersji 13.0.0, z maksymalnym poziomem optymalizacji},
    aboveskip=0pt,
    breaklines=true,
    label={lst:comparison:bloat_cpp_example:assembly}
    ]
    unsigned long foo<int>(int const*):
    cmp     rdi, 500
    jbe     .LBB1_2
    push    rax
    add     rdi, -4
    call    unsigned long foo<int>(int const*)
    mov     rdi, rax
    add     rsp, 8
  .LBB1_2:
    jmp     bar(unsigned long)          # TAILCALL
  unsigned long foo<float>(float const*):
    cmp     rdi, 500
    jbe     .LBB2_2
    push    rax
    add     rdi, -4
    call    unsigned long foo<float>(float const*)
    mov     rdi, rax
    add     rsp, 8
  .LBB2_2:
    jmp     bar(unsigned long)          # TAILCALL
  \end{lstlisting}
  
  
\end{minipage}
\subsection{Symbole zewnętrzne}
\label{Language_extensibility}
\label{external_symbols}
Jedną z konsekwencji zaproponowanych mechanizmów języka C-=-1 jest możliwość rozszerzania języka bez modyfikacji kompilatora.
Oznacza to, że pewne elementy składni, obecne w innych językach, stają się zbędne w C-=-1.
Większość języków programowania zawiera mechanizm umożliwiający import symboli zewnętrznych.
Mogą one pochodzić z kodu napisanego w innym języku lub z już skompilowanej biblioteki.

W wypadku C/C++ wymaga to użycia słowa \lstinline{external}, do zdefiniowania symbolu zewnętrznego oraz przekazania odpowiedniego argumentu do linkera.
W C-=-1 nie ma słowa kluczowego \lstinline{extern}.
Deklaracja symbolu zewnętrznego wymaga stworzenia atrybutu, który zostanie potem obsłużony w interfejsie backendu.

Listing \ref{lst:extern_cm1} zawiera przykład atrybut oznaczającego funkcję jako symbol zewnętrzny.
Interfejs \lstinline{ISymbolNameOverride} ma zwiększyć separację między interfejsem backendu a kodem kontrolującym nazwę symbolu.
Dzięki temu, w przyszłości będzie go można zaimplementować na inne sposoby, umożliwiając import symboli wymagających specjalnej obsługi.

\begin{minipage}{\linewidth}
  
  \begin{lstlisting}[
    numbers=left,
    firstnumber=0,
    caption={Atrybut deklarujący funkcję jako symbol zewnętrzny},
    aboveskip=0pt,
    label={lst:extern_cm1}
    ]
public interface ISymbolNameOverride {
  public fn createSymbolName() -> string;
}
public att<function> MarkAsExternal : ISymbolNameOverride {
  private _symbolName : string;
  public fn construct(symbolName: string) {
    self._symbolName = symbolName;
  }
  public fn attach(f: functionDescriptor) {
    ignoreBody(f);
  }
  public fn createSymbolName() -> string {
    return self._symbolName;
  }
}
\end{lstlisting}
\end{minipage}



Atrybut \lstinline{MarkAsExternal} ma natomiast dwie role.
Po pierwsze, przechowuje nazwę symbolu, który jest importowany.
W ten sposób użytkownik nie jest przywiązany do nazwy symbolu zadeklarowanej w skompilowanym pliku bibliotecznym.
Umożliwia to też importowanie funkcji z C++, których nazwy zostały zmieszane i zawierają znaki nieakceptowalne w identyfikatorach.
Po drugie, \lstinline{MarkAsExternal} wywołuje procedurę \lstinline{ignoreBody}.
Ta funkcja informuje kompilator, że powinien zignorować ciało podprogramu i analizować wyłącznie nagłówek.
To wywołanie jest konieczne, ponieważ kompilator zapewnia poprawność programu, analizując zawartość funkcji.
Dlatego, jeśli funkcja zwracająca jakąś wartość, nie zawiera instrukcji \lstinline{return}, program zostanie odrzucony.
W wypadku podprogramów importowanych z zewnątrz, ich ciało jest zbędne.

Listing \ref{lst:selected_backend_interface} zawiera wybrane funkcje z domyślnego interfejsu backendu, powiązane z symbolami zewnętrznymi.
Funkcja \lstinline{getFunctionName} ma za zadanie wygenerować nazwę symbolu, kompatybilną z C.
Linie od drugiej do czwartej obsługują atrybuty implementujące \lstinline{ISymbolNameOverride}.
Sprawdzają one, czy z daną funkcją powiązany jest atrybut nadpisujący nazwę symbolu i zwraca wygenerowaną przez niego wartość.

Funkcja \lstinline{buildFunction} ma stworzyć reprezentację pośrednią LLVM funkcji przekazanej jako parametr.
Linie od dwudziestej dziewiątej do trzydziestej trzeciej zawierają kod obsługujący symbole zewnętrzne.
Te instrukcje sprawdzają, czy z funkcją powiązany jest atrybut implementujący \lstinline{ISymbolNameOverride}.
Jeśli tak, ciało procedury jest ignorowane.
W przeciwnym wypadku interfejs backendu buduje funkcję normalnie.
W LLVM, procedury niezawierające ciała, są uznawane za symbole zewnętrzne.

\begin{minipage}{\linewidth}
  
  \begin{lstlisting}[
    numbers=left,
    firstnumber=0,
    caption={Wybrane fragmenty interfejsu backendu},
    aboveskip=0pt,
    label={lst:selected_backend_interface}
    ]

private fn getFunctionName(f: functionDescriptor) -> string {
  let attribute = f.get_attribute<ISymbolNameOverride>();
  if(attribute != null<ISymbolNameOverride>())
    return attribute.createSymbolName();
  let baseName = f.qualifiedName();
  let params = f.parameters();
  for(i in enumerate(0, params.length()))
  {
    baseName = baseName + "__" + (params[i].type().toString());
  }
  return baseName
    .replace(":", "_")
    [...]
    .replace("@", "__at__");
}

private fn buildFunction(
  f: functionDescriptor,
  llvmF: llvmFunction,
  registry: packageRegistry*,
  mod: llvmModule)
{
  let variables = dictionary<variableDescriptor, llvmValue>();
  let params = f.parameters();
  for(i in enumerate(0, params.length()))
    variables.push(params[i], llvmF.getParameter(i));
  let builder = llvmF.getBuilder();
  let attribute = f.get_attribute<ISymbolNameOverride>();
  if(attribute != null<ISymbolNameOverride>())
  {
    let code = f.code();
    build_block(&code, &builder, &variables, registry);
  }
}

\end{lstlisting}
\end{minipage}

Zaprezentowana implementacja jest bardzo prosta, zmienia jedynie nazwę generowanego symbolu i ignoruje ciało budowanej funkcji.
Mogą jednak istnieć scenariusze, w których użycie symbolu zewnętrznego wymaga bardziej złożonej obsługi.
Przykładowo, jeśli wywołanie funkcji zdefiniowanej w innym języku wymaga konwersji (ang. marshalling) przekazywanych parametrów, bądź zwracanej wartości.
Po niewielkich modyfikacjach zaproponowanego interfejsu obsługa takich przypadków byłaby możliwa.

\subsection{Użycie skompilowanego kodu w innych językach programowania}
\label{comparison:exporting_to_other_languages}

Użycie funkcji z jednego języka programowania, w innym języku zwykle wymaga zadeklarowania jej jako symbolu zewnętrznego.
W rozdziale \ref{external_symbols} zaprezentowano jak automatycznie zaimportować taką procedurę do C-=-1.
Ponieważ C-=-1 umożliwia wykonywanie dowolnego kodu w trakcie kompilacji, możliwym jest automatyczne eksportowanie zawartości skompilowanego pliku do innych języków programowania.

W zestawie bibliotek zaimplementowanych w ramach tej pracy, znajduje się pakiet \lstinline{bindingsGenerator}.
Zawiera interfejs \lstinline{IBindingsGenerator} którego metody są wywoływane przez \emph{interfejs backendu} (opisany w rozdziale \ref{Backend_Interface}) w trakcie kompilacji do LLVMIR.
Interfejs \lstinline{IBindingsGenerator} jest implementowany przez atrybuty, dołączane do funkcji, mające generować pliki nagłówkowe dla innych jęzków.

Pakiet \lstinline{CBindings} zawiera atrybut generujący pliki nagłówkowe dla języka C (plik\\ \path{\Libraries\cBindings\src\cBindingsAttribute.cm}).
Udostępnienie funkcjie C-=-1 do użycia w C wymaga osiągnięcia trzech celów.
Po pierwsze, nazwa wygenerowanego symbolu musi być poprawnym identyfikatorem w języku konsumującym bibliotekę.
LLVM nie ma takich wymagań, więc \emph{interface backendu} może generować symbole zawierające niedozwolone znaki.
Atrybuty implementujące \lstinline{IBindingsGenerator}, mogą ją więc nadpisać.
Po drugie koniecznym jest wygenerowanie definicji typów, w języku C, odpowiadającym typom w C-=-1, używanymi w nagłówku funkcji.
Tyczy się to wyłącznie klas i struktur definiowanych przez użytkownika, typy wbudowane C-=-1 muszą zostać przetłumaczone na ich odpowiedniki C.
Po trzecie wygenerowane pliki C muszą zostać umieszczone w folderze wyjściowym kompilatora, razem z wygenerowaną biblioteką.

W ramach przeprowadzonych prac badawczych wyeksportowano funkcję \lstinline{enumerate} z biblioteki standardowej C-=-1 do C.
Wygenerowany kod zawierał bardzo długie i nieporęczne nazwy, co jest skutkiem przyjętego algorytmu mieszania nazw \cite{wen2006analysis}.
Kod napisany w języku C był jednak w stanie wywołać funkcję \lstinline{enumerate} oraz w pełni użyć zwróconej pary iteratorów.
W przyszłych wersjach tej biblioteki zostanie użyty inny algorytm generujący nazwę symbolu, tworzący bardziej czytelne nazwy.

Listing \ref{lst:c_interopt} zawiera kod, napisany w C, wywołujący funkcję \lstinline{enumerate} z biblioteki standardowej C-=-1, przy użyciu wygenerowanych nagłówków.
Ten tworzy obiekt reprezentujący zakres liczb od zera do pięćdziesięciu pięciu (linia 17), a następnie używa jego iteratorów do wypisania wszystkich elementów do konsoli.
Zdecydowana większość kodu aplikacji składa się z pomieszanych nazw typów.
W praktycznym użyciu generator powinien zostać zmodyfikowany, aby tworzyć bardziej użyteczne identyfikatory, lub powinno się przypisać tym typom aliasy.

\begin{minipage}{\linewidth}
  
  \begin{lstlisting}[
    numbers=left,
    firstnumber=0,
    caption={Przykładowy program napisany w C, używający funkcji \lstinline{enumerate} z biblioteki standardowej C-=-1},
    aboveskip=0pt,
    breakatwhitespace=false,
    breaklines=true,
    label={lst:c_interopt}
    ]
#include <stdio.h>
#include <stdlib.h>
#include "__enumerate__usize__usize.h"
#include "__atHocCollection_integerIterator__comma__integerIterator ___iterate__atHocCollection_integerIterator__comma__integerIterator ___star__.h"
#include "__enumeration_integerIterator__comma__integerIterator___end__ enumeration_ integerIterator__comma__integerIterator___star__.h"
#include "__enumeration_integerIterator__comma__integerIterator___begin__ enumeration_ integerIterator__comma__integerIterator___star__.h"
#include "__integerIterator__get__integerIterator__star__.h"
#include "__integerIterator__advance__integerIterator__star__.h"
#include "__operator___not_equals____integerIterator__integerIterator.h"
struct arr {
  void *data;
  int count;
};
extern struct arr __new_array_char___usize(int size);
int main() {
  struct arr k =  __new_array_char___usize(3);
  printf("%p\n%i\n", k.data, k.count);
  __atHocCollection_l_integerIterator_c_integerIterator_r_ collection = __enumerate__usize__usize(0, 55);
  __enumeration_l_integerIterator_c_integerIterator_r_ enume = __atHocCollection_integerIterator__comma__integerIterator___ iterate__ atHocCollection_integerIterator__comma__integerIterator___star__(&collection);
  for(__integerIterator current = __enumeration_integerIterator__comma__integerIterator___begin__ enumeration_ integerIterator__comma__integerIterator___star__(&collection); __operator___not_equals____integerIterator__integerIterator( current, __enumeration_integerIterator__comma__integerIterator___end__ enumeration_ integerIterator__comma__integerIterator___star__(&collection)); __integerIterator__advance__integerIterator__star__(&current))
  {
    printf("%i\n", __integerIterator__get__integerIterator__star__(&current));
  }
  printf("%i\t%i\t%i\t%i\t", collection._from._step, collection._from._value, collection._to._step, collection._to._value);
}
\end{lstlisting}
\end{minipage}
