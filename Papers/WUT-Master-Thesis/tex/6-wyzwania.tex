\section{Wyzwania}
\subsection{Integracja z back-endem}
Przy pierwszym podejściu dobudowy reprezentacji pośredniej, użyto dedykowanych struktur danych. Było to podejście najbardziej naturalne i dające najwięcej bezpieczeństwa dzięki silnemu typowaniu. Wszystkie struktury danych opisane w rozdziale \ref{reprezentacja_posrednia} miały powiązane ze sobą klasy C++.


To podejście tworzy jednak duży problem. Reprezentacja pośrednia musi zostać wyeksponowana użytkownikowi. Ponieważ nie istniała możliwość stworzenia binarnego interfejsu między kompilatorem a interpretowanym kodem, struktury danych CIR musiały być dodatkowo reprezentowane za pomocą struktur danych interpretera.


Użytkownik może dokonywać modyfikacji w CIR, co oznacza że może dojść do rozbieżności między strukturami danych interpretera a kompilatora. Utrzymywanie tych dwóch reprezentacji stanowiło poważne wyzwanie, dlatego postanowiono zmienić podejście. Użycie wyłącznie struktur danych interpretera do reprezentacji CIR usunęło ten problem, kosztem bezpieczeństwa kodu.


Ponieważ w nowym podejściu, z perspektywy C++ niemalże wszystkie obiekty miały ten sam typ, kompilator stracił możliwość statycznej weryfikacji kodu.
Aby zapewnić poprawność programu, koniecznym stało się dodawanie walidacji argumentów do wszystkich funkcji.

\subsection{Informacje dodatkowe o obiektach}
Na wczesnej fazie projektu koniecznym okazało się przechowywanie informacji które nie zostały uwzględnione w modelu semantycznym.
Potrzeba ta powstała przy implementacji operatora \mintinline{Cpp}{sizeof}.
Ponieważ rozmiar obiektów jest ustalany dopiero przez LLVM, koniecznym stało się zastąpienie kodu funkcji, reprezentacją pośrednią backendu.

Rozwiązanie tego problemu wymaga wzięcia pod uwagę kilku istotnych aspektów.
Zostały one opisane w rozdziałach \ref{frontend_responsibility}, \ref{compiler_extensibility} oraz \ref{frontend_backend_complexity}.
Rozdziały \ref{original_solution} i \ref{final_solution}

\subsubsection{Oryginalne rozwiązanie}\label{original_solution}

Naturalnym podejściem do problemu, było zawarcie metadanych o obiekcie, wewnątrz jego definicji.
W ten sposób frontend miał do nich łatwy dostęp.

\subsubsection{Zakres odpowiedzialności frontendu} \label{frontend_responsibility}

\subsubsection{Rozszerzalność kompilatora}\label{compiler_extensibility}

\subsubsection{Złożoność granicy frontend-backend interface} \label{frontend_backend_complexity}

\subsubsection{Ostateczne rozwiązanie}\label{final_solution}
Po przeanalizowaniu rozwiązania z rozdziału \ref{original_solution} pod kątem problemów wymienionych w rozdziałach \ref{frontend_responsibility}, \ref{compiler_extensibility} oraz \ref{frontend_backend_complexity}, postanowiono radykalnie zmienić podejście.
Zamiast definiować metadane obiektów po stronie struktur danych frontendu, postanowiono zdefiniować interfejs umożliwiający zapis i metadanych o elementach modelu semantycznego. 
Listing \ref{lst:IMetadataHolder} zawiera jego deklarację.
Metody store i get zdefiniowane na liniach 2 i 3 umożliwiają zapis i odczyt metadanych.

Ten interfejs został zaimplementowany po stronie frontendu.
Ponieważ w tym kontekście sprawdzanie typów jest opcjonalne, nie wymagało to wprowadzania dodatkowych mechanizmów do języka takich jak \mintinline{Cpp}{std::any}.

W porównaniu z rozwiązaniem z oryginalny rozwiązaniem, to podejście znacząco zmniejsza granice między frontendem a interfejsem backendu.
Ponadto, podział odpowiedzialności pomiędzy tymi komponentami jest dużo bardziej logiczny.
Frontend nie wiedzieć o tym jakie informacje są potrzebne przy kompilacji.
Ponadto, ponieważ metadane które mogą być przechowywane nie są zdefiniowane odgórnie, znacząco zwiększa też rozszerzalność kompilatora.

\begin{lstlisting}[
    language=Cm1,
    numbers=left,
    firstnumber=0,
    caption={Interfejs obiektu przechowującego metadane},
    aboveskip=0pt,
    label={lst:IMetadataHolder}
]
class IMetadataHolder
{
    public fn store<T>(key: string, value: T);
    public fn get<T>(key: string) -> T;
    public fn has(key: string) -> bool;
}

class functionDescriptor : IMetadataHolder;
class typeDescriptor : IMetadataHolder;
class fieldDescriptor : IMetadataHolder;
class variableDescriptor : IMetadataHolder;
\end{lstlisting}